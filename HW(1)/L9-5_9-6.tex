\documentclass[12pt]{book}
\usepackage[utf8]{inputenc}
\usepackage{color,soul,CJK,epic,tikz,array}
\usepackage{amsmath,amsthm,amssymb}
\setlength{\parindent}{0em}
\linespread{1.3}
\author{andersonwu2000}
\usepackage[margin=1cm]{geometry}
\pagestyle{empty}
\thispagestyle{empty} 

\newcounter{sect}

\newcounter{block}[sect]
\newenvironment{tblock}[1]
{\refstepcounter{block}\theblock.~\begin{minipage}[t]{\dimexpr\linewidth}#1\\}
{\end{minipage}\\}

\newenvironment{comm}
{\makebox[12pt][l]{$\bullet$}\begin{minipage}[t]{\dimexpr\linewidth}}
{\end{minipage}}

\begin{document}
\begin{CJK}{UTF8}{bsmi}

\hfill 章節 9-1 9-2 吳至堯 U10811023

% 64 65
% 73 76

64-65. \begin{minipage}[t]{\dimexpr\linewidth-2em}
Find a  singular value decomposition of $A$
\end{minipage}\\

64. \begin{minipage}[t]{\dimexpr\linewidth-2em}
$A=\begin{bmatrix}
4&0&3\\0&0&5
\end{bmatrix}\quad\Rightarrow\quad A^TA=\begin{bmatrix}
16 & 0 & 12 \\
0 & 0 & 0 \\
12 & 0 & 34
\end{bmatrix}\quad\Rightarrow\quad\det(A^TA-\lambda I)=\textbf{0}\quad\Rightarrow\quad\lambda=40, 10, 0$ \\
$\Rightarrow\sigma_1=2\sqrt{10}, \sigma_2=\sqrt{10}, \sigma_3=0\quad\Rightarrow\quad$If $\lambda=40\quad\Rightarrow\quad$eigenvector of $\lambda=40$ is $\begin{bmatrix}
1\\0\\2
\end{bmatrix}$ \\
If $\lambda=20\quad\Rightarrow\quad$eigenvector of $\lambda=20$ is $\begin{bmatrix}
-2\\0\\1
\end{bmatrix}\quad\Rightarrow\quad$If $\lambda=0\quad\Rightarrow\quad$eigenvector  of $\lambda=0$ is $\begin{bmatrix}
0\\1\\0
\end{bmatrix}$ \\
Let $\displaystyle\textbf{v}_1=\frac{1}{\sqrt{5}}\begin{bmatrix}
1\\0\\2
\end{bmatrix},\textbf{v}_2=\frac{1}{\sqrt{5}}\begin{bmatrix}
-2\\0\\1
\end{bmatrix},\textbf{v}_3=\begin{bmatrix}
0\\1\\0
\end{bmatrix}\quad\Rightarrow\quad\textbf{u}_1=\frac{1}{\sigma_1}A\textbf{v}_1=\frac{1}{2\sqrt{10}\sqrt{5}}\begin{bmatrix}
4&0&3\\0&0&5
\end{bmatrix}\begin{bmatrix}
1\\0\\2
\end{bmatrix}=\begin{bmatrix}
\frac{1}{\sqrt{2}} \\ \frac{1}{\sqrt{2}}
\end{bmatrix}$ \\
$\displaystyle\Rightarrow\textbf{u}_2=\frac{1}{\sigma_2}A\textbf{v}_2=\frac{1}{\sqrt{10}\sqrt{5}}\begin{bmatrix}
4&0&3\\0&0&5
\end{bmatrix}\begin{bmatrix}
-2\\0\\1
\end{bmatrix}=\begin{bmatrix}
-\frac{1}{\sqrt{2}} \\ \frac{1}{\sqrt{2}}
\end{bmatrix}$ \\
$\displaystyle\Rightarrow A=U\Sigma V^T=\begin{bmatrix}
\frac{1}{\sqrt{2}} & -\frac{1}{\sqrt{2}} \\ 
\frac{1}{\sqrt{2}} & \frac{1}{\sqrt{2}}
\end{bmatrix}\begin{bmatrix}
2\sqrt{10} & 0 & 0 \\
0 & \sqrt{10} & 0
\end{bmatrix}\begin{bmatrix}
\frac{1}{\sqrt{5}} & \frac{-2}{\sqrt{5}} & 0 \\
0 & 0 & 1 \\
\frac{2}{\sqrt{5}} & \frac{1}{\sqrt{5}} & 0
\end{bmatrix}^T=\begin{bmatrix}
4&0&3\\0&0&5
\end{bmatrix}$
\end{minipage}\\

65. \begin{minipage}[t]{\dimexpr\linewidth-2em}
$A=\begin{bmatrix}
-1&4\\-2&2\\-2&-4
\end{bmatrix}\quad\Rightarrow\quad A^TA=\begin{bmatrix}
9&0\\0&36
\end{bmatrix}\quad\Rightarrow\quad\det(A^TA-\lambda I)=\textbf{0}\quad\Rightarrow\quad\lambda=36, 9$ \\
$\Rightarrow\sigma_1=6, \sigma_2=3\quad\Rightarrow\quad$If $\lambda=36\quad\Rightarrow\quad$eigenvector of $\lambda=36$ is $\begin{bmatrix}
0\\1
\end{bmatrix}$ \\
If $\lambda=20\quad\Rightarrow\quad$eigenvector of $\lambda=3$ is $\begin{bmatrix}
1\\0
\end{bmatrix}\quad\Rightarrow\quad$Let $\displaystyle\textbf{v}_1=\begin{bmatrix}
0\\1
\end{bmatrix},\textbf{v}_2=\begin{bmatrix}
1\\0
\end{bmatrix}$ \\
$\displaystyle\textbf{u}_1=\frac{1}{\sigma_1}A\textbf{v}_1=\frac{1}{6}\begin{bmatrix}
-1&4\\-2&2\\-2&-4
\end{bmatrix}\begin{bmatrix}
0\\1
\end{bmatrix}=\begin{bmatrix}
\frac{2}{3} \\ \frac{1}{3} \\ \frac{-2}{3}
\end{bmatrix}\quad\Rightarrow\quad\textbf{u}_2=\frac{1}{\sigma_2}A\textbf{v}_2=\frac{1}{3}\begin{bmatrix}
-1&4\\-2&2\\-2&-4
\end{bmatrix}\begin{bmatrix}
1\\0
\end{bmatrix}=\begin{bmatrix}
\frac{-1}{3} \\ \frac{-2}{3} \\ \frac{-2}{3}
\end{bmatrix}$ \\
Let $\displaystyle U=\begin{bmatrix}
\frac{2}{3} & \frac{-1}{3} & \frac{2}{3} \\ 
\frac{1}{3} & \frac{-2}{3} & \frac{-2}{3}\\ 
\frac{-2}{3} & \frac{-2}{3} & \frac{1}{3}
\end{bmatrix}$ orthogonal $\displaystyle\quad\Rightarrow\quad
A=U\Sigma V^T=\begin{bmatrix}
\frac{2}{3} & \frac{-1}{3} & \frac{2}{3} \\ 
\frac{1}{3} & \frac{-2}{3} & \frac{-2}{3}\\ 
\frac{-2}{3} & \frac{-2}{3} & \frac{1}{3}
\end{bmatrix}\begin{bmatrix}
6&0\\0&3\\0&0
\end{bmatrix}\begin{bmatrix}
0&1\\1&0
\end{bmatrix}^T=\begin{bmatrix}
-1&4\\-2&2\\-2&-4
\end{bmatrix}$
\end{minipage}\\

73. \begin{minipage}[t]{\dimexpr\linewidth-2em}
Find a reduced singular value decomposition of $A=\begin{bmatrix}
4&0&3\\0&0&5
\end{bmatrix}$. [ Note : Each matrix appears in the previous section, where you were asked to find its (unreduced) singular value decomposition. ] \\
(according Ex.64)$\quad\displaystyle\Rightarrow\quad A=U\Sigma V^T=\begin{bmatrix}
\frac{1}{\sqrt{2}} & -\frac{1}{\sqrt{2}} \\ 
\frac{1}{\sqrt{2}} & \frac{1}{\sqrt{2}}
\end{bmatrix}\begin{bmatrix}
2\sqrt{10} & 0 & 0 \\
0 & \sqrt{10} & 0
\end{bmatrix}\begin{bmatrix}
\frac{1}{\sqrt{5}} & \frac{-2}{\sqrt{5}} & 0 \\
0 & 0 & 1 \\
\frac{2}{\sqrt{5}} & \frac{1}{\sqrt{5}} & 0
\end{bmatrix}^T=\begin{bmatrix}
4&0&3\\0&0&5
\end{bmatrix}$ \\
$\displaystyle\Rightarrow A=U_1\Sigma_1 V^T_1=\begin{bmatrix}
\frac{1}{\sqrt{2}} & -\frac{1}{\sqrt{2}} \\ 
\frac{1}{\sqrt{2}} & \frac{1}{\sqrt{2}}
\end{bmatrix}\begin{bmatrix}
2\sqrt{10} & 0 \\
0 & \sqrt{10}
\end{bmatrix}\begin{bmatrix}
\frac{1}{\sqrt{5}} & \frac{-2}{\sqrt{5}} & 0 \\
0 & 0 & 1
\end{bmatrix}^T=\begin{bmatrix}
4&0&3\\0&0&5
\end{bmatrix}$

\end{minipage}\\

76. \begin{minipage}[t]{\dimexpr\linewidth-2em}
Find a reduced singular value expansion of $A=\begin{bmatrix}
4&0&3\\0&0&5
\end{bmatrix}=\sigma_1\textbf{u}_1\textbf{v}_1^T+\sigma_2\textbf{u}_2\textbf{v}_2^T+\sigma_3\textbf{u}_3\textbf{v}_3^T$. \\
$\Rightarrow A=2\sqrt{10}\begin{bmatrix}
\frac{1}{\sqrt{2}} \\ \frac{1}{\sqrt{2}}
\end{bmatrix}\begin{bmatrix}
\frac{1}{\sqrt{5}} & 0 &\frac{2}{\sqrt{5}}
\end{bmatrix}+\sqrt{10}\begin{bmatrix}
-\frac{1}{\sqrt{2}} \\ \frac{1}{\sqrt{2}}
\end{bmatrix}\begin{bmatrix}
\frac{-2}{\sqrt{5}} & 0 &\frac{1}{\sqrt{5}}
\end{bmatrix}$
\end{minipage}\\

\end{CJK}
\end{document}