\documentclass[12pt]{book}
\usepackage[utf8]{inputenc}
\usepackage{color,soul,CJK,epic,tikz,array}
\usepackage{amsmath,amsthm,amssymb}
\setlength{\parindent}{0em}
\linespread{1.3}
\author{andersonwu2000}
\usepackage[margin=1cm]{geometry}
\pagestyle{empty}
\thispagestyle{empty} 

\newcounter{sect}

\newcounter{block}[sect]
\newenvironment{tblock}[1]
{\refstepcounter{block}\theblock.~\begin{minipage}[t]{\dimexpr\linewidth}#1\\}
{\end{minipage}\\}

\newenvironment{comm}
{\makebox[12pt][l]{$\bullet$}\begin{minipage}[t]{\dimexpr\linewidth}}
{\end{minipage}}

\begin{document}
\begin{CJK}{UTF8}{bsmi}

\hfill 章節 7-1~7-2 吳至堯 U10811023

%12 13 15 17 26 32 35ad

12. \begin{minipage}[t]{\dimexpr\linewidth-2em}
A rectangular $x''y''z''$-coordinate system is obtained by first rotating a rectangular $xyz$-coordinate system $60^\circ$ counterclockwise about the $z$-axis (looking down the positive $z$-axis) to obtain an $x'y'z'$-coordinate system, and then totating the $x'y'z'$-coordinate system $45^\circ$ counterclockwise about the $y'$-axis (looking along the positive $y'$-axis toward the origin). Find a matrix $A$ such that $\begin{bmatrix}
x'' \\ y'' \\ z''
\end{bmatrix}=A\begin{bmatrix}
x \\ y \\ z
\end{bmatrix}$ where $(x,y,z)$ and $(x'', y'', z'')$ are the $xyz$- and $x''y''z''$-coordinates of the same point. \\

Let $P=\begin{bmatrix}
\cos60^\circ & -\sin60^\circ & 0 \\
\sin60^\circ &  \cos60^\circ & 0 \\
0    & 0    & 1 
\end{bmatrix}=\begin{bmatrix}
\frac{1}{2} & -\frac{\sqrt{3}}{2} & 0 \\
\frac{\sqrt{3}}{2} & \frac{1}{2}  & 0 \\
0    & 0    & 1
\end{bmatrix}\quad\Rightarrow\quad P^T=\begin{bmatrix}
\frac{1}{2} &  \frac{\sqrt{3}}{2} & 0 \\
-\frac{\sqrt{3}}{2} & \frac{1}{2} & 0 \\
0    & 0    & 1
\end{bmatrix}$ \\
$\Rightarrow PP^T=\begin{bmatrix}
1 & 0 & 0 \\
0 & 1 & 0 \\
0 & 0 & 1
\end{bmatrix}\quad\Rightarrow\quad P^{-1}=P^T, P$ is orthogonal\quad$\Rightarrow\quad\begin{bmatrix}
x'' \\ y'' \\ z''
\end{bmatrix}=P^{-1}\begin{bmatrix}
x' \\ y' \\ z'
\end{bmatrix}$ \\
Let $Q=\begin{bmatrix}
\cos45^\circ & 0 & \sin45^\circ \\
0 & 1 & 0 \\
-\sin45^\circ & 0 & \cos45^\circ
\end{bmatrix}=\begin{bmatrix}
\frac{1}{\sqrt{2}} & 0 & \frac{1}{\sqrt{2}} \\
0 & 1 & 0 \\
-\frac{1}{\sqrt{2}} & 0 & \frac{1}{\sqrt{2}}
\end{bmatrix}\quad\Rightarrow\quad Q^T=\begin{bmatrix}
\frac{1}{\sqrt{2}} & 0 & -\frac{1}{\sqrt{2}} \\
0 & 1 & 0 \\
\frac{1}{\sqrt{2}} & 0 &  \frac{1}{\sqrt{2}}
\end{bmatrix}$ \\
$\Rightarrow QQ^T=\begin{bmatrix}
1 & 0 & 0 \\
0 & 1 & 0 \\
0 & 0 & 1
\end{bmatrix}\quad\Rightarrow\quad Q^{-1}=Q^T, Q$ is orthogonal\quad$\Rightarrow\quad\begin{bmatrix}
x' \\ y' \\ z'
\end{bmatrix}=Q^{-1}\begin{bmatrix}
x \\ y \\ z
\end{bmatrix}$ \\
$\begin{bmatrix}
x'' \\ y'' \\ z''
\end{bmatrix}=P^{-1}Q^{-1}\begin{bmatrix}
x \\ y \\ z
\end{bmatrix}=\begin{bmatrix}
\frac{1}{2} &  \frac{\sqrt{3}}{2} & 0 \\
-\frac{\sqrt{3}}{2} & \frac{1}{2} & 0 \\
0    & 0    & 1
\end{bmatrix}\begin{bmatrix}
\frac{1}{\sqrt{2}} & 0 & -\frac{1}{\sqrt{2}} \\
0 & 1 & 0 \\
\frac{1}{\sqrt{2}} & 0 &  \frac{1}{\sqrt{2}}
\end{bmatrix}\begin{bmatrix}
x \\ y \\ z
\end{bmatrix}=\begin{bmatrix}
\frac{1}{2\sqrt{2}} & \frac{\sqrt{3}}{2} & -\frac{1}{2\sqrt{2}} \\
-\frac{\sqrt{3}}{2\sqrt{2}} & \frac{1}{2} & \frac{\sqrt{3}}{2\sqrt{2}} \\
\frac{1}{\sqrt{2}} & 0 &  \frac{1}{\sqrt{2}}
\end{bmatrix}\begin{bmatrix}
x \\ y \\ z
\end{bmatrix}$
\end{minipage}\\

13. \begin{minipage}[t]{\dimexpr\linewidth}
What conditions must $a$ and $b$ satisfy for the matrix $\begin{bmatrix}
a+2b & 2b-a \\
a-2b & 2b+a
\end{bmatrix}$ to be orthogonal? \\
(Thm 7.1.2.)$\quad\det\left(\begin{bmatrix}
a+2b & 2b-a \\
a-2b & 2b+a
\end{bmatrix}\right)=\pm1\quad\Rightarrow\quad2a^2+8b^2=1$
\end{minipage}

15.a. \begin{minipage}[t]{\dimexpr\linewidth-2em}
Use the result in Exercise 14 to prove that multiplication by a $2\times2$ orthogonal matrix is either a rotation or a reflection followed by a rotation about the $x$-axis. \\
(Ex 14.)$\quad2\times2$ orthogonal matrix has only one of two possible forms $\begin{bmatrix}
\cos\theta & -\sin\theta \\
\sin\theta & \cos\theta
\end{bmatrix}$ or $\begin{bmatrix}
\cos\theta & \sin\theta \\
\sin\theta & -\cos\theta
\end{bmatrix}$ \\
$\begin{bmatrix}
\cos\theta & -\sin\theta \\
\sin\theta & \cos\theta
\end{bmatrix}$ is a rotation with angle $\theta$. \\
$\begin{bmatrix}
\cos\theta & \sin\theta \\
\sin\theta & -\cos\theta
\end{bmatrix}=\begin{bmatrix}
\cos\theta & -\sin\theta \\
\sin\theta & \cos\theta
\end{bmatrix}\begin{bmatrix}
1 & 0 \\
0 & -1
\end{bmatrix}$ is a reflection followed by a rotation.
\end{minipage}

15.b. \begin{minipage}[t]{\dimexpr\linewidth-2em}
Prove that multiplication by $A$ is a rotation if $\det(A)=1$ and is a reflection followed by a rotation if $\det(A)=-1$. \\
$\det\left(\begin{bmatrix}
\cos\theta & -\sin\theta \\
\sin\theta & \cos\theta
\end{bmatrix}\right)=\cos^2\theta+\sin^2\theta=1,\quad\det\left(\begin{bmatrix}
\cos\theta & \sin\theta \\
\sin\theta & -\cos\theta
\end{bmatrix}\right)=-\cos^2\theta-\sin^2\theta=-1$
\end{minipage}

17. \begin{minipage}[t]{\dimexpr\linewidth-2em}
Find $a$, $b$, and $c$ for which the matrix $\begin{bmatrix}
a & \frac{1}{\sqrt{2}} & -\frac{1}{\sqrt{2}} \\
b & \frac{1}{\sqrt{6}} & \frac{1}{\sqrt{6}} \\
c & \frac{1}{\sqrt{3}} & \frac{1}{\sqrt{3}}
\end{bmatrix}$ is orthogonal. Are the values of $a$, $b$, and $c$ unique? \\
$\begin{bmatrix}
a & \frac{1}{\sqrt{2}} & -\frac{1}{\sqrt{2}} \\
b & \frac{1}{\sqrt{6}} & \frac{1}{\sqrt{6}} \\
c & \frac{1}{\sqrt{3}} & \frac{1}{\sqrt{3}}
\end{bmatrix}\begin{bmatrix}
a & b & c \\
\frac{1}{\sqrt{2}} & \frac{1}{\sqrt{6}} & \frac{1}{\sqrt{3}} \\
-\frac{1}{\sqrt{2}} & \frac{1}{\sqrt{6}} & \frac{1}{\sqrt{3}}
\end{bmatrix}=\begin{bmatrix}
a^2+1 & ab & ac \\
ab & b^2+\frac{1}{3} & bc+\frac{\sqrt{2}}{3} \\
ac & bc+\frac{\sqrt{2}}{3} & c^2+\frac{2}{3}
\end{bmatrix}=\begin{bmatrix}
1 & 0 & 0 \\
0 & 1 & 0 \\
0 & 0 & 1
\end{bmatrix}\quad\Rightarrow\quad\left\{\begin{matrix}
a = 0 \\
b = \pm\sqrt{\frac{2}{3}} \\
c = \pm\frac{1}{\sqrt{3}}
\end{matrix}\right.$ \\
$\Rightarrow$ $a$, $b$, $c$ are not unique.
\end{minipage}

26. \begin{minipage}[t]{\dimexpr\linewidth}
Find a matrix $P$ that orthogonally diagonalizes $A=\begin{bmatrix}
3 & 0 & 1 \\
0 & 2 & 0 \\
1 & 0 & 3
\end{bmatrix}$, and determine $P^{-1}AP$. \\
$\det(\lambda I-A)=\left|\begin{matrix}
\lambda-3 & 0 & -1 \\
0 & \lambda-2 & 0 \\
-1 & 0 & \lambda-3
\end{matrix}\right|=-(\lambda-2)^2(\lambda-4)=0\quad\Rightarrow\quad\lambda=2, 2, 4\quad\Rightarrow\quad$ Let $\textbf{x}=\begin{bmatrix}
x_1 \\ x_2 \\ x_3
\end{bmatrix}$ \\
If $\lambda=2$, $(2I-A)\textbf{x}=\begin{bmatrix}
-1 & 0 & -1 \\
0 & 0 & 0 \\
-1 & 0 & -1
\end{bmatrix}\begin{bmatrix}
x_1 \\ x_2 \\ x_3
\end{bmatrix}=\textbf{0}\quad\Rightarrow\quad\textbf{x}=\begin{bmatrix}
x_1 \\ x_2 \\ -x_1
\end{bmatrix}=t\begin{bmatrix}
0 \\ 1 \\ 0
\end{bmatrix}+t\begin{bmatrix}
1 \\ 0 \\ -1
\end{bmatrix},\quad t,r\in\mathbb{R}$ \\
If $\lambda=4$, $(4I-A)\textbf{x}=\begin{bmatrix}
1 & 0 & -1 \\
0 & 2 & 0 \\
-1 & 0 & 1
\end{bmatrix}\begin{bmatrix}
x_1 \\ x_2 \\ x_3
\end{bmatrix}=\textbf{0}\quad\Rightarrow\quad\textbf{x}=\begin{bmatrix}
x_1 \\ 0 \\ x_1
\end{bmatrix}=t\begin{bmatrix}
1 \\ 0 \\ 1
\end{bmatrix},\quad t\in\mathbb{R}$ \\
The bases for the eigenspaces are $v_1=\begin{bmatrix}
0 \\ 1 \\ 0
\end{bmatrix}, v_2=\begin{bmatrix}
1 \\ 0 \\ -1
\end{bmatrix}, v_3=\begin{bmatrix}
1 \\ 0 \\ 1
\end{bmatrix}\quad\Rightarrow\quad v_1, v_2, v_3$ linearly independent \\
Let $\displaystyle\textbf{u}_1=\frac{v_1}{\parallel v_1\parallel}=\begin{bmatrix}
0 \\ 1 \\ 0
\end{bmatrix}, \textbf{u}_2=\frac{v_2}{\parallel v_2\parallel}=\begin{bmatrix}
\frac{1}{\sqrt{2}} \\ 0 \\ -\frac{1}{\sqrt{2}}
\end{bmatrix}, \textbf{u}_3=\frac{v_3}{\parallel v_3\parallel}=\begin{bmatrix}
\frac{1}{\sqrt{2}} \\ 0 \\ \frac{1}{\sqrt{2}}
\end{bmatrix}\quad\Rightarrow\quad$Let $P=\begin{bmatrix}
0 & \frac{1}{\sqrt{2}} & \frac{1}{\sqrt{2}} \\
1 & 0 & 0 \\
0 & -\frac{1}{\sqrt{2}} & \frac{1}{\sqrt{2}}
\end{bmatrix}$ \\
$\Rightarrow P^{-1}AP=P^TAP=\begin{bmatrix}
0 & 1 & 0 \\
\frac{1}{\sqrt{2}} & 0 & -\frac{1}{\sqrt{2}} \\
\frac{1}{\sqrt{2}} & 0 & \frac{1}{\sqrt{2}}
\end{bmatrix}\begin{bmatrix}
3 & 0 & 1 \\
0 & 2 & 0 \\
1 & 0 & 3
\end{bmatrix}\begin{bmatrix}
0 & \frac{1}{\sqrt{2}} & \frac{1}{\sqrt{2}} \\
1 & 0 & 0 \\
0 & -\frac{1}{\sqrt{2}} & \frac{1}{\sqrt{2}}
\end{bmatrix}=\begin{bmatrix}
2 & 0 & 0 \\
0 & 2 & 0 \\
0 & 0 & 4
\end{bmatrix}$
\end{minipage}

32. \begin{minipage}[t]{\dimexpr\linewidth}
Find the characteristic polynomial and the dimensions of the eigenspaces of the symmetric matrix \\
Let $A=\begin{bmatrix}
3 & 2 & 2 \\
2 & 3 & 2 \\
2 & 2 & 3
\end{bmatrix}\ \Rightarrow\ \det(\lambda I-A)=\left|\begin{matrix}
\lambda-3 & -2 & -2 \\
-2 & \lambda-3 & -2 \\
-2 & -2 & \lambda-3
\end{matrix}\right|=\lambda^3-9\lambda^2+15\lambda-7=(\lambda-1)^2(\lambda-7)=0$ \\
If $\lambda=1$, $(I-A)\textbf{x}=\begin{bmatrix}
-2 & -2 & -2 \\
-2 & -2 & -2 \\
-2 & -2 & -2
\end{bmatrix}\begin{bmatrix}
x_1 \\ x_2 \\ x_3
\end{bmatrix}=\textbf{0}\quad\Rightarrow\quad\textbf{x}=\begin{bmatrix}
x_1 \\ x_2 \\ -x_1-x_2
\end{bmatrix}=t\begin{bmatrix}
0 \\ 1 \\ -1
\end{bmatrix}+r\begin{bmatrix}
1 \\ 0 \\ -1
\end{bmatrix},\quad t,r\in\mathbb{R}$ \\
If $\lambda=7$, $(7I-A)\textbf{x}=\begin{bmatrix}
 4 & -2 & -2 \\
-2 &  4 & -2 \\
-2 & -2 &  4
\end{bmatrix}\begin{bmatrix}
x_1 \\ x_2 \\ x_3
\end{bmatrix}=\textbf{0}\quad\Rightarrow\quad\textbf{x}=\begin{bmatrix}
x_1 \\ x_1 \\ x_1
\end{bmatrix}=t\begin{bmatrix}
1 \\ 1 \\ 1
\end{bmatrix},\quad t\in\mathbb{R}$ \\
$\Rightarrow$ If $\lambda=1$, dimension of eigenspace is 2.\qquad If $\lambda=1$, dimension of eigenspace is 1.
\end{minipage} \\

35. \begin{minipage}[t]{\dimexpr\linewidth}
Find the spectral decomposition of each matrix. \\
a. \begin{minipage}[t]{\dimexpr\linewidth}
Let $A=\begin{bmatrix}
4 & 2 \\
2 & 4
\end{bmatrix}\quad\Rightarrow\quad\det(\lambda I-A)=\left|\begin{matrix}
\lambda-4 & -2 \\
-2 & \lambda-4
\end{matrix}\right|=(\lambda-2)(\lambda-6)=0$ \\
If $\lambda=2$, $(2I-A)\textbf{x}=\begin{bmatrix}
-2 & -2 \\
-2 & -2
\end{bmatrix}\begin{bmatrix}
x_1 \\ x_2
\end{bmatrix}=\textbf{0}\quad\Rightarrow\quad\textbf{x}=\begin{bmatrix}
x_1 \\ -x_1
\end{bmatrix}=t\begin{bmatrix}
1 \\ -1
\end{bmatrix},\quad t\in\mathbb{R}$ \\
If $\lambda=6$, $(6I-A)\textbf{x}=\begin{bmatrix}
2  & -2 \\
-2 & 2
\end{bmatrix}\begin{bmatrix}
x_1 \\ x_2
\end{bmatrix}=\textbf{0}\quad\Rightarrow\quad\textbf{x}=\begin{bmatrix}
x_1 \\ x_1
\end{bmatrix}=t\begin{bmatrix}
1 \\ 1
\end{bmatrix},\quad t\in\mathbb{R}$ \\ 
Let $v_1=\begin{bmatrix}
1 \\ -1
\end{bmatrix}, v_2=\begin{bmatrix}
1 \\ 1
\end{bmatrix}\quad\Rightarrow\quad v_1, v_2$ linearly independent$\quad\Rightarrow\quad$
Let $\displaystyle\textbf{u}_1=\begin{bmatrix}
\frac{1}{\sqrt{2}} \\ -\frac{1}{\sqrt{2}}
\end{bmatrix}, \textbf{u}_2=\begin{bmatrix}
\frac{1}{\sqrt{2}} \\ \frac{1}{\sqrt{2}}
\end{bmatrix}$ \\
$A=\lambda_1\textbf{u}_1\textbf{u}_1^T+\lambda_2\textbf{u}_2\textbf{u}_2^T
=2\begin{bmatrix}
\frac{1}{\sqrt{2}} \\ -\frac{1}{\sqrt{2}}
\end{bmatrix}\begin{bmatrix}
\frac{1}{\sqrt{2}} & -\frac{1}{\sqrt{2}}
\end{bmatrix}+6\begin{bmatrix}
\frac{1}{\sqrt{2}} \\ \frac{1}{\sqrt{2}}
\end{bmatrix}\begin{bmatrix}
\frac{1}{\sqrt{2}} & \frac{1}{\sqrt{2}}
\end{bmatrix}
=2\begin{bmatrix}
\frac{1}{2} & -\frac{1}{2} \\
-\frac{1}{2} & \frac{1}{2}
\end{bmatrix}+6\begin{bmatrix}
\frac{1}{2} & \frac{1}{2} \\
\frac{1}{2} & \frac{1}{2}
\end{bmatrix}$
\end{minipage} \\

d. \begin{minipage}[t]{\dimexpr\linewidth}
Let $A=\begin{bmatrix}
3 & 0 & 1 \\
0 & 2 & 0 \\
1 & 0 & 3
\end{bmatrix}\quad\Rightarrow\quad\det(\lambda I-A)=\left|\begin{matrix}
\lambda-3 & 0 & -1 \\
0 & \lambda-2 & 0 \\
-1 & 0 & \lambda-3
\end{matrix}\right|=(\lambda-2)^2(\lambda-4)=0$ \\
If $\lambda=2$, $(2I-A)\textbf{x}=\begin{bmatrix}
-1 & 0 & -1 \\
0 & 0 & 0 \\
-1 & 0 & -1
\end{bmatrix}\begin{bmatrix}
x_1 \\ x_2 \\ x_3
\end{bmatrix}=\textbf{0}\quad\Rightarrow\quad\textbf{x}=\begin{bmatrix}
x_1 \\ 0 \\ -x_1
\end{bmatrix}=t\begin{bmatrix}
0 \\ 1 \\ 0
\end{bmatrix}+r\begin{bmatrix}
1 \\ 0 \\ -1
\end{bmatrix},\quad t,r\in\mathbb{R}$ \\
If $\lambda=4$, $(4I-A)\textbf{x}=\begin{bmatrix}
1 & 0 & -1 \\
0 & 2 & 0 \\
-1 & 0 & 1
\end{bmatrix}\begin{bmatrix}
x_1 \\ x_2 \\ x_3
\end{bmatrix}=\textbf{0}\quad\Rightarrow\quad\textbf{x}=\begin{bmatrix}
x_1 \\ 0 \\ x_1
\end{bmatrix}=t\begin{bmatrix}
1 \\ 0 \\ 1
\end{bmatrix},\quad t\in\mathbb{R}$ \\
Let $v_1=\begin{bmatrix}
0 \\ 1 \\ 0
\end{bmatrix}, v_2=\begin{bmatrix}
1 \\ 0 \\ -1
\end{bmatrix}, v_3=\begin{bmatrix}
1 \\ 0 \\ 1
\end{bmatrix}$ linearly independent$\ \Rightarrow\ $
Let $\displaystyle\textbf{u}_1=\begin{bmatrix}
0 \\ 1 \\ 0
\end{bmatrix}, \textbf{u}_1=\begin{bmatrix}
\frac{1}{\sqrt{2}} \\ 0 \\ -\frac{1}{\sqrt{2}}
\end{bmatrix}, \textbf{u}_2=\begin{bmatrix}
\frac{1}{\sqrt{2}} \\ 0 \\ \frac{1}{\sqrt{2}}
\end{bmatrix}$ \\
$A=\lambda_1\textbf{u}_1\textbf{u}_1^T+\lambda_2\textbf{u}_2\textbf{u}_2^T+\lambda_3\textbf{u}_3\textbf{u}_3^T
=2\begin{bmatrix}
0 & 0 & 0 \\
0 & 1 & 0 \\
0 & 0 & 0
\end{bmatrix}+2\begin{bmatrix}
 \frac{1}{2} & 0 &-\frac{1}{2} \\
 0 & 0 & 0 \\
-\frac{1}{2} & 0 & \frac{1}{2}
\end{bmatrix}+4\begin{bmatrix}
 \frac{1}{2} & 0 & \frac{1}{2} \\
 0 & 0 & 0 \\
 \frac{1}{2} & 0 & \frac{1}{2}
\end{bmatrix}$
\end{minipage} \\
\end{minipage}

\end{CJK}
\end{document}