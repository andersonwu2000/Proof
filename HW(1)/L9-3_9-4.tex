\documentclass[12pt]{book}
\usepackage[utf8]{inputenc}
\usepackage{color,soul,CJK,epic,tikz,array}
\usepackage{amsmath,amsthm,amssymb}
\setlength{\parindent}{0em}
\linespread{1.3}
\author{andersonwu2000}
\usepackage[margin=1cm]{geometry}
\pagestyle{empty}
\thispagestyle{empty} 

\newcounter{sect}

\newcounter{block}[sect]
\newenvironment{tblock}[1]
{\refstepcounter{block}\theblock.~\begin{minipage}[t]{\dimexpr\linewidth}#1\\}
{\end{minipage}\\}

\newenvironment{comm}
{\makebox[12pt][l]{$\bullet$}\begin{minipage}[t]{\dimexpr\linewidth}}
{\end{minipage}}

\begin{document}
\begin{CJK}{UTF8}{bsmi}

\hfill 章節 9-1 9-2 吳至堯 U10811023

% 46 47
% 50 52 57

46-47. \begin{minipage}[t]{\dimexpr\linewidth-2em}
The adjacency matrix $A$ of an Internet search engine is given. Use the method of Example 3 of Section 9.3 to rank the sites in decreasing order of authority.
\end{minipage}\\

46. \begin{minipage}[t]{\dimexpr\linewidth-2em}
$\displaystyle A=\begin{bmatrix}
0 & 0 & 1 & 1 & 0 \\
0 & 0 & 1 & 1 & 1 \\
0 & 1 & 0 & 1 & 0 \\
0 & 0 & 0 & 1 & 0 \\
1 & 0 & 0 & 0 & 0
\end{bmatrix}\quad\Rightarrow\quad a_0=\begin{bmatrix}
1 \\ 1 \\ 2 \\ 4 \\ 1
\end{bmatrix}\quad\Rightarrow\quad a_1=\frac{A^TAa_0}{\parallel A^TAa_0\parallel}\approx\begin{bmatrix}
0.38 \\ 0.19 \\ 0.48 \\ 0.82 \\ 0.26
\end{bmatrix}\quad\Rightarrow\quad a_2=\frac{A^TAa_1}{\parallel A^TAa_1\parallel}\approx\begin{bmatrix}
0.01 \\ 0.17 \\ 0.49 \\ 0.81 \\ 0.27
\end{bmatrix}\quad\Rightarrow\quad a_3=\frac{A^TAa_2}{\parallel A^TAa_2\parallel}\approx\begin{bmatrix}
0 \\ 0.17 \\ 0.5 \\ 0.81 \\ 0.27
\end{bmatrix}\quad\Rightarrow\quad$ site4 $>$ site3 $>$ site5 $>$ site2 $>$ site1
\end{minipage}\\

47. \begin{minipage}[t]{\dimexpr\linewidth-2em}
$\displaystyle A=\begin{bmatrix}
0 & 0 & 1 & 1 & 1 & 0 & 0 & 0 & 0 & 0 \\
0 & 1 & 0 & 0 & 0 & 1 & 1 & 0 & 0 & 0 \\
1 & 0 & 0 & 1 & 0 & 0 & 0 & 1 & 0 & 0 \\
1 & 1 & 1 & 1 & 1 & 1 & 1 & 1 & 1 & 1 \\
0 & 0 & 0 & 0 & 1 & 0 & 0 & 0 & 1 & 0 \\
1 & 1 & 1 & 0 & 1 & 1 & 0 & 0 & 0 & 1 \\
0 & 0 & 1 & 0 & 0 & 0 & 1 & 0 & 1 & 1 \\
0 & 1 & 0 & 1 & 0 & 1 & 0 & 1 & 1 & 1 \\
0 & 0 & 0 & 0 & 1 & 0 & 1 & 0 & 1 & 1 \\
0 & 0 & 1 & 1 & 0 & 1 & 1 & 0 & 1 & 0
\end{bmatrix}\quad\Rightarrow\quad a_0=\begin{bmatrix}
3\\ 4\\ 5\\ 5\\ 5\\ 5\\ 5\\ 3\\ 6\\ 5
\end{bmatrix}\quad\Rightarrow\quad a_1=\frac{A^TAa_0}{\parallel A^TAa_0\parallel}\approx\begin{bmatrix}
0.21\\0.29\\0.34\\0.32\\0.3\\0.36\\0.32\\0.22\\0.39\\0.36
\end{bmatrix}\quad\Rightarrow\quad a_2=\frac{A^TAa_1}{\parallel A^TAa_1\parallel}\approx\begin{bmatrix}
0.21\\0.3\\0.34\\0.32\\0.3\\0.36\\0.32\\0.22\\0.38\\0.36
\end{bmatrix}\quad\Rightarrow\quad a_3=\frac{A^TAa_2}{\parallel A^TAa_2\parallel}\approx\begin{bmatrix}
0.21\\0.3\\0.34\\0.32\\0.3\\0.36\\0.32\\0.22\\0.38\\0.36
\end{bmatrix}\quad\Rightarrow\quad$ site9 $>$ site10 $> \cdots >$ site8 $>$ site1
\end{minipage}\\

50. \begin{minipage}[t]{\dimexpr\linewidth-2em}
The IBM Roadrunner computer can operate at speeds in excess of 1 petaflop per second (1 petaflop = $10^{15}$ flops). Use Table 1 to estimate the time required to perform the following operations of the invertible $100000\times100000$ matrix $A$.
\end{minipage}\\

50.a. \begin{minipage}[t]{\dimexpr\linewidth-2em}
Execute the forward phase of Gauss-Jordan elimination.$\displaystyle\quad\Rightarrow\quad\frac{2}{3}(10^5)^3\div10^{15}=\frac{2}{3}$ (sec.)
\end{minipage}\\

50.b. \begin{minipage}[t]{\dimexpr\linewidth-2em}
Execute the backward phase of Gauss-Jordan elimination.$\displaystyle\quad\Rightarrow\quad(10^5)^2\div10^{15}=10^{-5}$ (sec.)
\end{minipage}\\

50.c. \begin{minipage}[t]{\dimexpr\linewidth-2em}
$LU$-decomposition of $A$.$\displaystyle\quad\Rightarrow\quad\frac{2}{3}(10^5)^3\div10^{15}=\frac{2}{3}$ (sec.)
\end{minipage}\\

50.d. \begin{minipage}[t]{\dimexpr\linewidth-2em}
Find $A^{-1}$ by reducing $[A\mid I]$ to $[I\mid A^{-1}]$.$\displaystyle\quad\Rightarrow\quad2(10^5)^3\div10^{15}=2$ (sec.)
\end{minipage}\\

52. \begin{minipage}[t]{\dimexpr\linewidth-2em}
About how many teraflops per second must a computer be able to execute to find the inverse of a matrix of size $100000\times100000$ in less than 0.5s?(See Table 1 from this chapter.) \\
Let computer execute $x$ teraflops per second$\displaystyle\quad\Rightarrow\quad2(10^5)^3\div 10^{12}x<0.5\quad\Rightarrow\quad x>4(10)^{3}$
\end{minipage}\\

57. \begin{minipage}[t]{\dimexpr\linewidth-2em}
If $A$ and $B$ are upper triangular matrices of size $n$ how many flops are required to compute $AB$? \\
Let $A=\begin{bmatrix}
a_{11} & a_{12} & \cdots & a_{1n} \\
0      & a_{22} & \cdots & a_{2n} \\
\vdots & \vdots &        & \vdots \\
0      & 0      & \cdots & a_{nn} \\
\end{bmatrix},\ B=\begin{bmatrix}
b_{11} & b_{12} & \cdots & b_{1n} \\
0      & b_{22} & \cdots & b_{2n} \\
\vdots & \vdots &        & \vdots \\
0      & 0      & \cdots & b_{nn} \\
\end{bmatrix}$ are upper triangular \\
$\Rightarrow AB=\begin{bmatrix}
a_{11}b_{11} & \sum_{r=1}^2 a_{1r}b_{r2} & \cdots & \sum_{r=1}^n a_{1r}b_{rn} \\
0      & a_{22}b_{22} & \cdots & \sum_{r=2}^n a_{2r}b_{rn} \\
\vdots & \vdots &        & \vdots \\
0      & 0      & \cdots & a_{nn}b_{nn} \\
\end{bmatrix}$ \\
$\Rightarrow$ the number of flops : $\begin{bmatrix}
1+0 & 2+1 & \cdots & n+(n-1)     \\
0   & 1+0 & \cdots & (n-1)+(n-2) \\
\vdots & \vdots &        & \vdots \\
0      & 0      & \cdots & 1+0 \\
\end{bmatrix}=\begin{bmatrix}
1 & 3 & \cdots & 2n-1     \\
0   & 1 & \cdots & 2n-3 \\
\vdots & \vdots &        & \vdots \\
0      & 0      & \cdots & 1 \\
\end{bmatrix}\Rightarrow\begin{bmatrix}
n^2 \\ (n-1)^2 \\ \vdots \\ 1
\end{bmatrix}$ \\
$\Rightarrow$ the number of flops $\displaystyle=\frac{n(n+1)(2n+1)}{6}=\frac{1}{3}n^3+\frac{1}{2}n^2+\frac{1}{6}n\approx\frac{1}{3}n^3$
\end{minipage}\\

\end{CJK}
\end{document}