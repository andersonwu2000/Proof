\documentclass[12pt]{book}
\usepackage[utf8]{inputenc}
\usepackage{color,soul,CJK,epic,tikz,array}
\usepackage{amsmath,amsthm,amssymb}
\setlength{\parindent}{0em}
\linespread{1.3}
\author{andersonwu2000}
\usepackage[margin=1cm]{geometry}
\pagestyle{empty}
\thispagestyle{empty} 

\newcounter{sect}

\newcounter{block}[sect]
\newenvironment{tblock}[1]
{\refstepcounter{block}\theblock.~\begin{minipage}[t]{\dimexpr\linewidth}#1\\}
{\end{minipage}\\}

\newenvironment{comm}
{\makebox[12pt][l]{$\bullet$}\begin{minipage}[t]{\dimexpr\linewidth}}
{\end{minipage}}

\begin{document}
\begin{CJK}{UTF8}{bsmi}

\hfill 章節 7-5~8-1 吳至堯 U10811023

% 106 109 111 119 123 124
% 6 10 27 29 34

106. \begin{minipage}[t]{\dimexpr\linewidth-2em}
Show that $A$ is unitary, and find $A^{-1}$. \\
$\displaystyle A=\begin{bmatrix}
\frac{6}{7} & \frac{2}{7}+\frac{3i}{7} \\
-\frac{2}{7}+\frac{3i}{7} & \frac{6}{7}
\end{bmatrix}\quad\Rightarrow\quad A^*=\begin{bmatrix}
\frac{6}{7} & -\frac{2}{7}-\frac{3i}{7} \\
\frac{2}{7}-\frac{3i}{7} & \frac{6}{7}
\end{bmatrix}$ \\
$\displaystyle A^{-1}=\frac{1}{\frac{6}{7}\frac{6}{7}-\frac{3i+2}{7}\frac{3i-2}{7}}\begin{bmatrix}
\frac{6}{7} & -\frac{2}{7}-\frac{3i}{7} \\
\frac{2}{7}-\frac{3i}{7} & \frac{6}{7}
\end{bmatrix}=\begin{bmatrix}
\frac{6}{7} & -\frac{2}{7}-\frac{3i}{7} \\
\frac{2}{7}-\frac{3i}{7} & \frac{6}{7}
\end{bmatrix}=A^*\quad\Rightarrow\quad$ A is unitary
\end{minipage}\\

109,111. 
Find a unitary matrix $P$ that diagonalizes the Hermitian matrix $A$, and determine $P^{-1}AP$.

109. \begin{minipage}[t]{\dimexpr\linewidth-2em}
$A=\begin{bmatrix}
5 & 1+i \\
1-i & 6
\end{bmatrix}\quad\Rightarrow\quad\det(A-\lambda I)=\left|\begin{matrix}
5-\lambda & 1+i \\
1-i & 6-\lambda
\end{matrix}\right|=(\lambda-4)(\lambda-7)=0\quad\Rightarrow\quad\lambda=4, 7$ \\
If $\lambda=4,\ (A-4 I)\textbf{x}=\textbf{0}\quad\Rightarrow\quad\textbf{x}=t\begin{bmatrix}
-1-i \\ 1
\end{bmatrix},\quad$ If $\lambda=7,\ (A-7 I)\textbf{x}=\textbf{0}\quad\Rightarrow\quad\textbf{x}=t\begin{bmatrix}
1+i \\ 2
\end{bmatrix},\quad t\in\mathbb{R}$ \\
Let $\displaystyle\textbf{v}_1=\begin{bmatrix}
-1-i \\ 1
\end{bmatrix},\textbf{v}_2\begin{bmatrix}
1+i \\ 2
\end{bmatrix}\quad\Rightarrow\quad\textbf{p}_1=\frac{\textbf{v}_1}{\parallel\textbf{v}_1\parallel}=\begin{bmatrix}
\frac{-1-i}{\sqrt{3}} \\ \frac{1}{\sqrt{3}}
\end{bmatrix},\textbf{p}_2=\frac{\textbf{v}_2}{\parallel\textbf{v}_2\parallel}=\begin{bmatrix}
\frac{1+i}{\sqrt{6}} \\ \frac{2}{\sqrt{6}}
\end{bmatrix}$ \\
Let $\displaystyle P=\begin{bmatrix}
\frac{-1-i}{\sqrt{3}} & \frac{1+i}{\sqrt{6}} \\
\frac{1}{\sqrt{3}} & \frac{1}{\sqrt{6}}
\end{bmatrix}\quad\Rightarrow\quad P^{-1}=\begin{bmatrix}
\frac{-1+i}{\sqrt{3}} & \frac{1}{\sqrt{6}} \\
\frac{1-i}{\sqrt{3}} & \frac{1}{\sqrt{6}}
\end{bmatrix}\quad\Rightarrow\quad P^{-1}AP=\begin{bmatrix}
4 & 0 \\
0 & 7
\end{bmatrix}$
\end{minipage}\\

111. \begin{minipage}[t]{\dimexpr\linewidth-2em}
$A=\begin{bmatrix}
4 & 0 & 0 \\
0 & -4 & -2+i \\
0 & -2-i & 0
\end{bmatrix}\quad\Rightarrow\quad\det(A-\lambda I)=\left|\begin{matrix}
4-\lambda & 0 & 0 \\
0 & -4-\lambda & -2+i \\
0 & -2-i & -\lambda
\end{matrix}\right|=0\quad\Rightarrow\quad\lambda=1, -5, 4$ \\
If $\lambda=1,\ (A-I)\textbf{x}=0\quad\Rightarrow\quad\textbf{x}=t\begin{bmatrix}
0 \\ -2+i \\ 5
\end{bmatrix},\quad$ If $\lambda=-5,\ (A+5I)\textbf{x}=0\quad\Rightarrow\quad\textbf{x}=t\begin{bmatrix}
0 \\ 2-i \\ 1
\end{bmatrix},$ \\
If $\lambda=4,\ (A-4I)\textbf{x}=0\quad\Rightarrow\quad\textbf{x}=t\begin{bmatrix}
1 \\ 0 \\ 0
\end{bmatrix},\quad t\in\mathbb{R}$ \\
Let $\textbf{v}_1=\begin{bmatrix}
0 \\ -2+i \\ 5
\end{bmatrix},\textbf{v}_2=\begin{bmatrix}
0 \\ 2-i \\ 1
\end{bmatrix},\textbf{v}_3=\begin{bmatrix}
1 \\ 0 \\ 0
\end{bmatrix}\quad\Rightarrow\quad\textbf{p}_1=\begin{bmatrix}
0 \\ \frac{i-2}{\sqrt{30}} \\ \frac{5}{\sqrt{30}}
\end{bmatrix},\textbf{p}_2=\begin{bmatrix}
0 \\ \frac{2-i}{\sqrt{6}} \\ \frac{1}{\sqrt{6}}
\end{bmatrix},\textbf{p}_3=\begin{bmatrix}
1 \\ 0 \\ 0
\end{bmatrix}$ \\
Let $P=\begin{bmatrix}
0 & 0 & 1 \\ 
\frac{i-2}{\sqrt{30}} & \frac{2-i}{\sqrt{6}} & 0\\ 
\frac{5}{\sqrt{30}} & \frac{1}{\sqrt{6}} & 0
\end{bmatrix}\quad\Rightarrow\quad P^{-1}=\begin{bmatrix}
0 & \frac{-2-i}{\sqrt{30}} & \frac{5}{\sqrt{30}} \\ 
0 & \frac{2+i}{\sqrt{6}} & \frac{1}{\sqrt{6}} \\ 
1 & 0 & 0
\end{bmatrix}\quad\Rightarrow\quad P^{-1}AP=\begin{bmatrix}
1 & 0 & 0 \\
0 &-5 & 0 \\
0 & 0 & 4
\end{bmatrix}$
\end{minipage}\\

119. \begin{minipage}[t]{\dimexpr\linewidth-2em}
Prove that each entry on the main diagonal of a skew-Hermitian matrix is either zero or a pure imaginary number. \\
Let $A=\begin{bmatrix}a_{ij}+b_{ij}i\end{bmatrix}$ is a skew-Hermitian matrix$\quad\Rightarrow\quad A^*=\overline{A}^T=-A$ \\
$\Rightarrow\quad\overline{a_{ii}+b_{ii}i}=a_{ii}-b_{ii}i=-(a_{ii}+b_{ii}i)=-a_{ii}-b_{ii}i\quad\Rightarrow\quad a_{ii}=0$
\end{minipage}\\

123. \begin{minipage}[t]{\dimexpr\linewidth-2em}
Show that the eigenvalues of a skew-Hermitian matrix are either zero or purely imaginary. \\
Let $A$ is a skew-Hermitian matrix$\quad\Rightarrow\quad (A-\lambda I)\textbf{x}=0,\ \textbf{x}\ne\textbf{0}\quad\Rightarrow\quad A\textbf{x}=\lambda\textbf{x}$ \\
$\Rightarrow(\overline{A\textbf{x}})^T=\overline{\textbf{x}}^T\overline{A}^T=\overline{\textbf{x}}^T(-A)=(\overline{\lambda\textbf{x}})^T=\overline{\lambda}\overline{\textbf{x}}^T\quad\Rightarrow\quad\overline{\textbf{x}}^TA\textbf{x}=\overline{\textbf{x}}^T\lambda\textbf{x}=\lambda\overline{\textbf{x}}^T\textbf{x}=-\overline{\lambda}\overline{\textbf{x}}^T\textbf{x}$ \\
$\Rightarrow\lambda=-\overline{\lambda}\quad\Rightarrow\quad\lambda$ is purely imaginart or zero.
\end{minipage}\\

124. \begin{minipage}[t]{\dimexpr\linewidth-2em}
Show that the eigenvalues of a unitary matrix have modulus 1. \\
Let $A$ is a unitary matrix$\quad\Rightarrow\quad A\textbf{x}=\lambda\textbf{x},\ \textbf{x}\ne\textbf{0}$ \\
$\Rightarrow(\overline{A\textbf{x}})^T=\overline{\textbf{x}}^TA^*=\overline{\textbf{x}}^TA^{-1}=(\overline{\lambda\textbf{x}})^T=\overline{\lambda}\overline{\textbf{x}}^T\quad\Rightarrow\quad\overline{\textbf{x}}^TA^{-1}A\textbf{x}=\overline{\textbf{x}}^T\textbf{x}=\overline{\lambda}\overline{\textbf{x}}^TA\textbf{x}=\overline{\lambda}\overline{\textbf{x}}^T\lambda\textbf{x}$ \\
$\Rightarrow\overline{\lambda}\lambda=\mid\lambda\mid=1$
\end{minipage}\\

6. \begin{minipage}[t]{\dimexpr\linewidth-2em}
Determine whether the function $T:M_{22}\rightarrow R$ is a linear transformation.
\end{minipage}\\

6(a). \begin{minipage}[t]{\dimexpr\linewidth-2em}
$T\left(\begin{bmatrix}
a & b \\
c & d
\end{bmatrix}\right)=3a-4b+c-d$ is a linear transformation.\\
$T\left(k\begin{bmatrix}
a & b \\
c & d
\end{bmatrix}\right)=T\left(\begin{bmatrix}
ka & kb \\
kc & kd
\end{bmatrix}\right)=3ka-4kb+kc-kd=k(3a-4b+c-d)=kT\left(\begin{bmatrix}
a & b \\
c & d
\end{bmatrix}\right)$ \\
$T\left(\begin{bmatrix}
a & b \\
c & d
\end{bmatrix}+\begin{bmatrix}
e & f \\
g & h
\end{bmatrix}\right)$ \begin{minipage}[t]{\dimexpr\linewidth-2em}
$=T\left(\begin{bmatrix}
a+e & b+f \\
c+g & d+h
\end{bmatrix}\right)=3(a+e)-4(b+f)+(c+g)-(d+h)$ \\
$=(3a-4b+c-d)+(3e-4f+g-h)=T\left(\begin{bmatrix}
a & b \\
c & d
\end{bmatrix}\right)+T\left(\begin{bmatrix}
e & f \\
g & h
\end{bmatrix}\right)$
\end{minipage}
\end{minipage}\\

6(b). \begin{minipage}[t]{\dimexpr\linewidth-2em}
$T\left(\begin{bmatrix}
a & b \\
c & d
\end{bmatrix}\right)=a^2+b^2$ is not a linear transformation. \\
$T\left(k\begin{bmatrix}
a & b \\
c & d
\end{bmatrix}\right)=T\left(\begin{bmatrix}
ka & kb \\
kc & kd
\end{bmatrix}\right)=k^2a^2+k^2b^2=k^2(a^2+b^2)\ne kT\left(\begin{bmatrix}
a & b \\
c & d
\end{bmatrix}\right)$
\end{minipage}\\

10. \begin{minipage}[t]{\dimexpr\linewidth-2em}
Consider the basis $S=\{\textbf{v}_1, \textbf{v}_2\}$ for $R^2$, where $\textbf{v}_1=(2, 1)$ and $\textbf{v}_2=(-1, 3)$, and let $T:R^2\rightarrow R^3$ be the linear transformation such that $T(\textbf{v}_1)=(1, 2, 0)$ and $T(\textbf{v}_2)=(0, 3, 5)$. \\
Find a formula for $T(x_1, x_2)$ and use that formula to find $T(2, -3)$. \\
Let $c_1, c_2\in\mathbb{R}$ and $(x_1, x_2)\in\mathbb{R}^2$ s.t. $\displaystyle c_1\textbf{v}_1+c_2\textbf{v}_2=(x_1, x_2)\quad\Rightarrow\quad c_1=\frac{3x_1+x_2}{7}, c_2=\frac{-x_1+2x_2}{7}$ \\
$\displaystyle\Rightarrow T(x_1, x_2)=c_1T(\textbf{v}_1)+c_2T(\textbf{v}_2)=(c_1, 2c_1+3c_2, 5c_2)=\left(\frac{3x_2+x_2}{7}, \frac{3x_1+8x_2}{7}, \frac{-5x_1+10x_2}{7}\right)$ \\
$\displaystyle T(2, -3)=\left(\frac{6-3}{7}, \frac{6-24}{7}, \frac{-10-30}{7}\right)=\left(\frac{3}{7}, \frac{-18}{7}, \frac{-40}{7}\right)$
\end{minipage}\\

27(a). \begin{minipage}[t]{\dimexpr\linewidth-2em}
Describe the kernel and range of the orthogonal projection on the $xz$-plane. \\
Let $T(x, y, z)=(x, 0, z),\ (x, y, z)\in R^3$ is the orthogonal projection on the $xz$-plane \\
$T(0, y, 0)=(0, 0, 0)$ for all $(x, y, z)\in R^3\quad\Rightarrow\quad\ker(T)=(0, y, 0)$ for all $(0, y, 0)\in R^3$ \\
$T(x, y, z)=(x, 0, z)$ for all $(x, 0, z)\in R^3\quad\Rightarrow\quad R(T)=(x, 0, z)$ for all $(x, 0, z)\in R^3$
\end{minipage}\\

27(b). \begin{minipage}[t]{\dimexpr\linewidth-2em}
Describe the kernel and range of the orthogonal projection on the $yz$-plane. \\
Let $T(x, y, z)=(0, y, z),\ (x, y, z)\in R^3$ is the orthogonal projection on the $yz$-plane \\
$T(x, 0, 0)=(0, 0, 0)$ for all $(x, y, z)\in R^3\quad\Rightarrow\quad\ker(T)=(x, 0, 0)$ for all $(x, 0, 0)\in R^3$ \\
$T(x, y, z)=(0, y, z)$ for all $(0, y, z)\in R^3\quad\Rightarrow\quad R(T)=(0, y, z)$ for all $(0, y, z)\in R^3$
\end{minipage}\\

27(c). \begin{minipage}[t]{\dimexpr\linewidth-2em}
Describe the kernel and range of the orthogonal projection on the plane defined by the equation $y=x$. \\
Let $\displaystyle T(x, y, z)=(\frac{x+y}{2}, \frac{x+y}{2}, z),\ (x, y, z)\in R^3$ is the orthogonal projection on $y=x$ \\
$\displaystyle T(x, -x, 0)=(0, 0, 0)$ for all $(x, y, z)\in R^3\quad\Rightarrow\quad\ker(T)=(x, -x, 0)$ for all $(x, -x, 0)\in R^3$ \\
$T(x, y, z)=(\frac{x+y}{2}, \frac{x+y}{2}, z)$ for all $(\frac{x+y}{2}, \frac{x+y}{2}, z)\in R^3\quad\Rightarrow\quad R(T)=(\frac{x+y}{2}, \frac{x+y}{2}, z)$ for all $(\frac{x+y}{2}, \frac{x+y}{2}, z)\in R^3$
\end{minipage}\\

29. \begin{minipage}[t]{\dimexpr\linewidth-2em}
In each part, use the given information to find the nullity of the linear transformation $T$.
\end{minipage}\\

29(a). \begin{minipage}[t]{\dimexpr\linewidth-2em}
$T:R^4\rightarrow R^8$ has rank 2$\quad\Rightarrow\quad\mathrm{rank}(T)+\mathrm{nullity}(T)=\dim(R^4)=4\quad\Rightarrow\quad\mathrm{nullity}(T)=4-2=2$
\end{minipage}\\

29(b). \begin{minipage}[t]{\dimexpr\linewidth-2em}
$T:P_4\rightarrow P_5$ has rank 1$\quad\Rightarrow\quad\mathrm{rank}(T)+\mathrm{nullity}(T)=\dim(P_4)=5\quad\Rightarrow\quad\mathrm{nullity}(T)=5-1=4$
\end{minipage}\\

29(c). \begin{minipage}[t]{\dimexpr\linewidth-2em}
The range of $T:R^7\rightarrow R^2$ is $R^2\quad\Rightarrow\quad\mathrm{rank}(T)+\mathrm{nullity}(T)=\dim(R(T))+\mathrm{nullity}(T)=\dim(R^7)=7$ \\
$\Rightarrow\mathrm{nullity}(T)=7-\dim(R(T))=7-\dim(R^2)=7-2=5$
\end{minipage}\\

29(d). \begin{minipage}[t]{\dimexpr\linewidth-2em}
$T:M_{22}\rightarrow M_{22}$ has rank 2$\quad\Rightarrow\quad\mathrm{rank}(T)+\mathrm{nullity}(T)=\dim(M_{22})=4\quad\Rightarrow\quad\mathrm{nullity}(T)=4-2=2$
\end{minipage}\\

34. \begin{minipage}[t]{\dimexpr\linewidth-2em}
Let $T:R^3\rightarrow R^3$ be multiplication by $A=\begin{bmatrix}
1 & 3 & 4 \\
3 & 4 & 7 \\
-2& 2 & 0
\end{bmatrix}\quad\Rightarrow\quad$ Let $\textbf{x}=\begin{bmatrix}
x_1 \\ x_2 \\ x_3
\end{bmatrix},\quad T(\textbf{x})=A\textbf{x}$
\end{minipage}\\

34(a). \begin{minipage}[t]{\dimexpr\linewidth-2em}
Show that the kernel of $T$ is a line through the origin, and find parametric equations for it. \\
$\ker(T)=\{\textbf{x}\in R^3:T(\textbf{x})=A\textbf{x}=\textbf{0}\}\quad\rightarrow\quad A\textbf{x}=\begin{bmatrix}
1 & 3 & 4 \\
3 & 4 & 7 \\
-2& 2 & 0
\end{bmatrix}\begin{bmatrix}
x_1 \\ x_2 \\ x_3
\end{bmatrix}=\textbf{0}\quad\rightarrow\quad\begin{bmatrix}
1 & 0 & 1 \\
0 & 1 & 1 \\
0 & 0 & 0 
\end{bmatrix}\begin{bmatrix}
x_1 \\ x_2 \\ x_3
\end{bmatrix}=\textbf{0}$ \\
$\Rightarrow \ker(T)=t\begin{bmatrix}
-1 \\ -1 \\ 1
\end{bmatrix},\ t\in R\quad\Rightarrow\quad$If $t=0,\quad0\begin{bmatrix}
-1 \\ -1 \\ 1
\end{bmatrix}=\textbf{0}$
\end{minipage}\\

34(b). \begin{minipage}[t]{\dimexpr\linewidth-2em}
Show that the range of $T$ is a plane through the origin, and find an equation for it. \\
$R(T)=\{(x,y,z)\in R^3:(x,y,z)=A\textbf{x},\ \textbf{x}\in R^3\}\quad\Rightarrow\quad(x,y,z)=x_1\begin{bmatrix}
1 \\ 3 \\ -2
\end{bmatrix}+x_2\begin{bmatrix}
3 \\ 4 \\ 2
\end{bmatrix}+x_3\begin{bmatrix}
4 \\ 7 \\ 0
\end{bmatrix}$ \\
Since $\begin{bmatrix}
1 \\ 3 \\ -2
\end{bmatrix}+\begin{bmatrix}
3 \\ 4 \\ 2
\end{bmatrix}=\begin{bmatrix}
4 \\ 7 \\ 0
\end{bmatrix}$ and $\begin{bmatrix}
1 \\ 3 \\ -2
\end{bmatrix},\begin{bmatrix}
3 \\ 4 \\ 2
\end{bmatrix}$ linear independent$\quad\Rightarrow\quad R(T)=t\begin{bmatrix}
1 \\ 3 \\ -2
\end{bmatrix}+u\begin{bmatrix}
3 \\ 4 \\ 2
\end{bmatrix},\ t,u\in R$ \\
If $t=u=0,\quad0\begin{bmatrix}
1 \\ 3 \\ -2
\end{bmatrix}+0\begin{bmatrix}
3 \\ 4 \\ 2
\end{bmatrix}=\textbf{0}$
\end{minipage}\\


\end{CJK}
\end{document}