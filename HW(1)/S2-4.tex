\documentclass[12pt]{book}
\usepackage[utf8]{inputenc}
\usepackage{color,soul,CJK,epic,tikz,array}
\usepackage{amsmath,amsthm,amssymb}
\setlength{\parindent}{0em}
\linespread{1.3}
\author{andersonwu2000}
\pagestyle{empty}
\thispagestyle{empty} 

\newcounter{sect}

\newcounter{block}[sect]
\newenvironment{tblock}[1]
{\refstepcounter{block}\theblock.~\begin{minipage}[t]{\dimexpr\linewidth}#1\\}
{\end{minipage}\\}

\newenvironment{comm}
{\makebox[12pt][l]{$\bullet$}\begin{minipage}[t]{\dimexpr\linewidth}}
{\end{minipage}}

\begin{document}
\begin{CJK}{UTF8}{bsmi}

\hfill 章節 2-4 吳至堯 U10811023 \\

5. Use the PMI to prove the following for all natural numbers: \\\\
j. $3^n\leq1+2^n$ \\
\begin{minipage}[t]{\dimexpr\linewidth}
For $n=1$, $3^1=3\geq1+2^1=3$, statement is true. \\
Assume that $n=k\in\mathbb{N}$, $3^k\geq1+2^k$ \\
If $n=k+1$, $3^{k+1}=3\cdot3^k\geq3\cdot(1+2^k)\geq3+3\cdot2^k\geq1+2\cdot2^k=1+2^{k+1}$ \\
Hence $3^n\leq1+2^n$ for all $n\in\mathbb{N}$
\end{minipage} \\\\

o. $\displaystyle\frac{n^3}{3}+\frac{n^5}{5}+\frac{7n}{15}$ is an integer \\[5pt]
\begin{minipage}[t]{\dimexpr\linewidth}
For $n=1$, $\displaystyle\frac{1}{3}+\frac{1}{5}+\frac{7}{15}=\frac{5+3+7}{15}=1$, statement is true. \\[5pt]
Assume that $n=k\in\mathbb{N}$, $\displaystyle\frac{k^3}{3}+\frac{k^5}{5}+\frac{7k}{15}$ is an integer \\[5pt]
If $n=k+1$, $\displaystyle\frac{{(k+1)}^3}{3}+\frac{{(k+1)}^5}{5}+\frac{7(k+1)}{15}$ \\\hspace*{6em}$\displaystyle=\frac{3k^5+5k^3+7k}{15}+\frac{15k^4+30k^3+45k^2+30k+15}{15}$ \\[5pt]
$\displaystyle\because\frac{k^3}{3}+\frac{k^5}{5}+\frac{7k}{15}$ and $\displaystyle\frac{15k^4+30k^3+45k^2+30k+15}{15}$ are divisible by 15 \\
Thus the statement is true for $k+1$ \\[5pt]
Hence $\displaystyle\frac{n^3}{3}+\frac{n^5}{5}+\frac{7n}{15}$ is an integer
\end{minipage}

\end{CJK}
\end{document}