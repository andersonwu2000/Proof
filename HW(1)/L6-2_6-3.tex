\documentclass[12pt]{book}
\usepackage[utf8]{inputenc}
\usepackage{color,soul,CJK,epic,tikz,array}
\usepackage{amsmath,amsthm,amssymb}
\setlength{\parindent}{0em}
\linespread{1.3}
\author{andersonwu2000}
\usepackage[margin=1cm]{geometry}
\pagestyle{empty}
\thispagestyle{empty} 

\newcounter{sect}

\newcounter{block}[sect]
\newenvironment{tblock}[1]
{\refstepcounter{block}\theblock.~\begin{minipage}[t]{\dimexpr\linewidth}#1\\}
{\end{minipage}\\}

\newenvironment{comm}
{\makebox[12pt][l]{$\bullet$}\begin{minipage}[t]{\dimexpr\linewidth}}
{\end{minipage}}

\begin{document}
\begin{CJK}{UTF8}{bsmi}

\hfill 章節 6-2~6-3 吳至堯 U10811023

%34b 35b 42 47b 48c
%75 82a 86 95 96

34.b. \begin{minipage}[t]{\dimexpr\linewidth}
Let $P_2$ have the inner product in Example 7 in Section 6.1. \\
Find the cosine of the angle between \textbf{p} and \textbf{q}. \\
$\textbf{p}=2x+x^2,\quad\textbf{q}=1-x+2x^2\quad\Rightarrow\quad\left\langle \textbf{p},\textbf{q} \right\rangle=0\cdot1+2\cdot(-1)+1\cdot2=0$ \\
$\parallel\textbf{p}\parallel=\sqrt{\left\langle \textbf{p},\textbf{p} \right\rangle}=\sqrt{2\cdot2+1\cdot1}=\sqrt{5},\quad\parallel\textbf{q}\parallel=\sqrt{\left\langle \textbf{q},\textbf{q} \right\rangle}=\sqrt{1\cdot1+(-1)\cdot(-1)+2\cdot2}=\sqrt{6}$ \\
$\displaystyle\Rightarrow\cos\theta=\frac{\left\langle \textbf{p},\textbf{q} \right\rangle}{\parallel\textbf{p}\parallel\parallel\textbf{q}\parallel}=\frac{0}{\sqrt{5}\cdot\sqrt{6}}=0$
\end{minipage}\\

42. \begin{minipage}[t]{\dimexpr\linewidth}
Let $R^4$ have the Euclidean inner product. \\
Find two unit vectors that are orthogonal to all three of the vectors \\
$\textbf{u}=(1,1,-4,0),\quad\textbf{v}=(-1,1,2,-2),\quad\textbf{w}=(3,-2,5,4)$ \\
Let $\textbf{x}=(x_1,x_2,x_3,x_4),\quad\parallel\textbf{x}\parallel=1,\quad0=\left\langle \textbf{x},\textbf{u} \right\rangle=\left\langle \textbf{x},\textbf{v} \right\rangle=\left\langle \textbf{x},\textbf{w} \right\rangle$ \\
$\displaystyle\Rightarrow\quad\begin{matrix}
x_1  & +  x_2  & -  4x_3 &         & =0 \\
-x_1 & +  x_2  & +  2x_3 & -  2x_4 & =0 \\
3x_1 & -  2x_2 & +  5x_3 & +  4x_4 & =0
\end{matrix}\quad\rightarrow\quad\begin{bmatrix}
1 & 1 & -4 & 0 & 0 \\
-1 & 1 & 2 & -2 & 0 \\
3 & -2 & 5 & 4 & 0
\end{bmatrix}\quad\rightarrow\quad\begin{bmatrix}
1 & 0 & 0 & \frac{3}{4} & 0 \\
0 & 1 & 0 & -\frac{13}{12} & 0 \\
0 & 0 & 1 & -\frac{1}{12} & 0
\end{bmatrix}$ \\
Let $\displaystyle x_4=t,\quad\textbf{x}=(-\frac{3}{4}t,\frac{13}{12}t,\frac{1}{12}t,t),\quad t\in\mathbb{R}\quad\Rightarrow\quad\parallel\textbf{x}\parallel=(-\frac{3}{4}t)^2+(\frac{13}{12}t)^2+(\frac{1}{12}t)^2+t^2=1$ \\
$\displaystyle\Rightarrow t=\pm\frac{12}{\sqrt{395}}\quad\textbf{x}=(-\frac{9}{\sqrt{395}},\frac{13}{\sqrt{395}},\frac{1}{\sqrt{395}},\frac{12}{\sqrt{395}})$ or $\displaystyle(\frac{9}{\sqrt{395}},-\frac{13}{\sqrt{395}},-\frac{1}{\sqrt{395}},-\frac{12}{\sqrt{395}})$
\end{minipage} \\

48.c. \begin{minipage}[t]{\dimexpr\linewidth}
Find a basis for the orthogonal complement of the subspace of $R^n$ spanned by the vectors. \\
$\textbf{v}_1=(3,0,1,-2),\textbf{v}_2=(-1,-2,-2,1),\textbf{v}_3=(4,2,3,-3)$ \\
$\textbf{v}_1-\textbf{v}_2=\textbf{v}_3\quad\Rightarrow\quad\mathrm{\ Let\ }W=\mathrm{span}\{\textbf{v}_1,\textbf{v}_2,\textbf{v}_3\}=\mathrm{span}\{\textbf{v}_1,\textbf{v}_2\},\quad\textbf{v}=(x,y,z,w)\in W^\perp$ \\
$\Rightarrow\textbf{v}\cdot\textbf{v}_1=\textbf{v}\cdot\textbf{v}_2=0\quad\Rightarrow\quad\begin{matrix}
3x & & +z & -2w & =0 \\
-x & -2y & -2z & +w & =0
\end{matrix}\quad\rightarrow\quad\begin{bmatrix}
3  & 0  & 1  & -2 & 0 \\
-1 & -2 & -2 & 1  & 0
\end{bmatrix}$ \\
\end{minipage}

\end{CJK}
\end{document}