\documentclass[12pt]{book}
\usepackage[utf8]{inputenc}
\usepackage{color,soul,CJK,epic,tikz,array}
\usepackage{amsmath,amsthm,amssymb}
\usepackage{graphicx}
\usepackage{float}
\usepackage{subfigure}
\setlength{\parindent}{0em}
\linespread{1.3}
\author{andersonwu2000}
\usepackage[margin=1cm]{geometry}
\pagestyle{empty}
\thispagestyle{empty} 

\newcounter{sect}

\newcounter{block}[sect]
\newenvironment{tblock}[1]
{\refstepcounter{block}\theblock.sim\begin{minipage}[t]{\dimexpr\linewidth}#1\\}
{\end{minipage}\\}

\newenvironment{comm}
{\makebox[12pt][l]{$\bullet$}\begin{minipage}[t]{\dimexpr\linewidth}}
{\end{minipage}}

\begin{document}
\begin{CJK}{UTF8}{bsmi}

\hfill 章節 1 吳至堯 U10811023

\begin{minipage}[t]{\dimexpr\linewidth-2em}
找三種商標,判定群的種類 
\begin{figure}[H] 
\centering 
\subfigure[Audi : $D_2$]{
\includegraphics[width=0.3\textwidth]{HW(2)/Audi.png}}
\centering 
\subfigure[Mercedes-Benz : $D_3$]{
\includegraphics[width=0.3\textwidth]{HW(2)/Benz.png}}
\centering 
\subfigure[BP : $D_{18}$]{
\includegraphics[width=0.3\textwidth]{HW(2)/BP.png}}
\end{figure}
\end{minipage}\\

31. \begin{minipage}[t]{\dimexpr\linewidth-2em}
Prove that every group table is a Latin square; that is, each element of the group appears exactly once in each row and each column. \\
Suppose that $G$ is a group under the operator $*$, and $a, b, c\in G$ \\
Let $-c$ is the inverse of $c$, and $e$ is the identity of $G$ \\
Assume that $a*c=b*c\quad\Rightarrow\quad a*c$ appears twice in a column of the group table \\
$\Rightarrow\quad a*c*(-c)=b*c*(-c)\quad\Rightarrow\quad a*e=b*e=a=b$ \\
Assume that $c*a=c*b\quad\Rightarrow\quad c*a$ appears twice in a row of the group table \\
$\Rightarrow\quad (-c)*c*a=(-c)*c*b\quad\Rightarrow\quad e*a=e*b=a=b$ \\
Hence, each element of the group appears exactly once in each row and each column.
\end{minipage}\\

31. \begin{minipage}[t]{\dimexpr\linewidth-2em}
Construct a Cayley table for $U(12)$. \\
Since $U(12)=\{1,5,7,11\}$, the Cayley table for $U(12)$ is \\[5pt]
\begin{tabular}{c|c|c|c|c}
    $\mod 12$ & 1 & 5 & 7 & 11 \\\hline
    1 & 1 & 5 & 7 & 11 \\\hline
    5 & 5 & 1 & 11 & 7 \\\hline
    7 & 7 & 11 & 1 & 5 \\\hline
    11 & 11 & 7 & 5 & 1
\end{tabular}
\end{minipage}\\

\end{CJK}
\end{document}