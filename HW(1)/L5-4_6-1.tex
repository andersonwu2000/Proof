\documentclass[12pt]{book}
\usepackage[utf8]{inputenc}
\usepackage{color,soul,CJK,epic,tikz,array}
\usepackage{amsmath,amsthm,amssymb}
\setlength{\parindent}{0em}
\linespread{1.3}
\author{andersonwu2000}
\usepackage[margin=1cm]{geometry}
\pagestyle{empty}
\thispagestyle{empty} 

\newcounter{sect}

\newcounter{block}[sect]
\newenvironment{tblock}[1]
{\refstepcounter{block}\theblock.~\begin{minipage}[t]{\dimexpr\linewidth}#1\\}
{\end{minipage}\\}

\newenvironment{comm}
{\makebox[12pt][l]{$\bullet$}\begin{minipage}[t]{\dimexpr\linewidth}}
{\end{minipage}}

\begin{document}
\begin{CJK}{UTF8}{bsmi}

\hfill 章節 5-4~6-1 吳至堯 U10811023



105. \begin{minipage}[t]{\dimexpr\linewidth}
Use the procedure in Exercise 104 to solve $y''+y'-20y=0$ \\
Let $y_1=y,\ y_2=y'\quad\Rightarrow\quad y'_1=y',\ y'_2=y''\quad\Rightarrow\quad y'_1=y_2,\ y'_2=20y-y'=20y_1-y'_2$ \\
$\textbf{y}'=A\textbf{y}=\begin{bmatrix}
y'_1 \\ y'_2
\end{bmatrix}=\begin{bmatrix}
0 & 1 \\
20 & -1
\end{bmatrix}\begin{bmatrix}
y_1 \\ y_2
\end{bmatrix}\quad\Rightarrow\quad\det(\begin{bmatrix}
\lambda & -1 \\
-20 & \lambda+1
\end{bmatrix})=\lambda^2+\lambda-20=0\quad\Rightarrow\quad\lambda=4,-5$ \\
If $\lambda=4,\quad(\lambda I-A)\textbf{x}=\begin{bmatrix}
4 & -1 \\
-20 & 5
\end{bmatrix}\begin{bmatrix}
x_1 \\ x_2
\end{bmatrix}=\begin{bmatrix}
0 \\ 0
\end{bmatrix}\quad\Rightarrow\quad\begin{bmatrix}
x_1 \\ x_2
\end{bmatrix}=t\begin{bmatrix}
1 \\ 4
\end{bmatrix},\quad t\in\mathbb{R}$ \\
If $\lambda=-5,\quad(\lambda I-A)\textbf{x}=\begin{bmatrix}
-5 & -1 \\
-20 & -4
\end{bmatrix}\begin{bmatrix}
x_1 \\ x_2
\end{bmatrix}=\begin{bmatrix}
0 \\ 0
\end{bmatrix}\quad\Rightarrow\quad\begin{bmatrix}
x_1 \\ x_2
\end{bmatrix}=t\begin{bmatrix}
-1 \\ 5
\end{bmatrix},\quad t\in\mathbb{R}$ \\
Let $P=\begin{bmatrix}
1 & -1 \\
4 & 5
\end{bmatrix}\quad\Rightarrow\quad P^{-1}=\frac{1}{9}\begin{bmatrix}
5 & 1 \\
-4 & 1
\end{bmatrix}\quad\Rightarrow\quad D=P^{-1}AP=\begin{bmatrix}
4 & 0 \\
0 & -5
\end{bmatrix}$ \\
Let $\textbf{y}=P\textbf{u},\ \textbf{y}'=P\textbf{u}'\quad\Rightarrow\quad\textbf{u}'=P^{-1}A\textbf{y}=P^{-1}AP\textbf{u}\quad\Rightarrow\quad\begin{bmatrix}
u'_1 \\ u'_2
\end{bmatrix}=\begin{bmatrix}
4 & 0 \\
0 & -5
\end{bmatrix}\begin{bmatrix}
u_1 \\ u_2
\end{bmatrix}\quad\Rightarrow\quad\begin{matrix}
u'_1=4u_1 \\ u'_2=-5u_2
\end{matrix}$ \\
Let $\begin{matrix}
u_1=c_1 e^{4x} \\ u_2=c_2 e^{-5x}
\end{matrix},\ c\in\mathbb{R}\quad\Rightarrow\quad\begin{bmatrix}
y_1 \\ y_2
\end{bmatrix}=\begin{bmatrix}
1 & -1 \\
4 & 5
\end{bmatrix}\begin{bmatrix}
c_1 e^{4x} \\ c_2 e^{-5x}
\end{bmatrix}=\begin{bmatrix}
c_1 e^{4x}-c_2 e^{-5x} \\ 4c_1 e^{4x}+5c_2 e^{-5x}
\end{bmatrix}$
\end{minipage}\\

106. \begin{minipage}[t]{\dimexpr\linewidth}
Explain how you might use the procedure in Exercise 104 to solve $y'''-5y''+2y'+8y=0$. \\
Use your procedure to solve the equation. \\
Let $y_1=y,\ y_2=y',\ y_3=y''\quad\Rightarrow\quad y'_1=y',\ y'_2=y'',\ y'_3=y'''$ \\
$\textbf{y}'=A\textbf{y}=\begin{bmatrix}
y'_1 \\ y'_2 \\ y'_3
\end{bmatrix}=\begin{bmatrix}
0 & 1 & 0 \\%y'=y_2
0 & 0 & 1 \\%y''=y_3
-8 & -2 & 5 %y'''=5y''-2y'-8y
\end{bmatrix}\begin{bmatrix}
y_1 \\ y_2 \\ y_3
\end{bmatrix}\quad\Rightarrow\quad\det(\begin{bmatrix}
\lambda & -1 & 0 \\
0 & \lambda & -1 \\
8 & 2 & \lambda-5 
\end{bmatrix})=0\quad\Rightarrow\quad\lambda=2,4,-1$ \\
If $\lambda=2,\quad(\lambda I-A)\textbf{x}=0\quad\Rightarrow\quad \textbf{x}=t\begin{bmatrix}
1 & 2 & 4
\end{bmatrix}^T,\quad t\in\mathbb{R}$ \\
If $\lambda=4,\quad(\lambda I-A)\textbf{x}=0\quad\Rightarrow\quad \textbf{x}=t\begin{bmatrix}
1 & 4 & 16
\end{bmatrix}^T,\quad t\in\mathbb{R}$ \\
If $\lambda=-1,\quad(\lambda I-A)\textbf{x}=0\quad\Rightarrow\quad \textbf{x}=t\begin{bmatrix}
1 & -1 & 1
\end{bmatrix}^T,\quad t\in\mathbb{R}$ \\
Let $P=\begin{bmatrix}
1 & 1 & 1 \\
2 & 4 & -1 \\
4 & 16 & 1
\end{bmatrix}\quad\Rightarrow\quad P^{-1}=\displaystyle\frac{1}{30}\begin{bmatrix}
20 & 15 & -5 \\
-6 & -3 & 3 \\
16 & -12 & 2
\end{bmatrix}\quad\Rightarrow\quad D=P^{-1}AP=\begin{bmatrix}
2 & 0 & 0 \\
0 & 4 & 0 \\
0 & 0 & -1
\end{bmatrix}$ \\
Let $\textbf{y}=P\textbf{u},\ \textbf{y}'=P\textbf{u}'\quad\Rightarrow\quad\textbf{u}'=P^{-1}AP\textbf{u}\quad\Rightarrow\quad\begin{bmatrix}
u'_1 \\ u'_2 \\ u'_3
\end{bmatrix}=\begin{bmatrix}
2 & 0 & 0 \\
0 & 4 & 0 \\
0 & 0 & -1
\end{bmatrix}\begin{bmatrix}
u_1 \\ u_2 \\ u_3
\end{bmatrix}\quad\Rightarrow\quad\begin{matrix}
u'_1=2u_1 \\ u'_2=4u_2 \\ u'_3=-u_3
\end{matrix}$ \\
Let $\begin{matrix}
u_1=c_1 e^{2x} \\ u_2=c_2 e^{4x} \\ u_3=c_3 e^{-x}
\end{matrix},\ c\in\mathbb{R}\quad\Rightarrow\quad\begin{bmatrix}
y_1 \\ y_2 \\ y_3
\end{bmatrix}=\begin{bmatrix}
1 & 1 & 1 \\
2 & 4 & -1 \\
4 & 16 & 1
\end{bmatrix}\begin{bmatrix}
c_1 e^{2x} \\ c_2 e^{4x} \\ c_3 e^{-x}
\end{bmatrix}=\begin{bmatrix}
c_1 e^{2x}+c_2 e^{4x}+c_3 e^{-x} \\
2c_1 e^{2x}+4c_2 e^{4x}-c_3 e^{-x} \\
4c_1 e^{2x}+16c_2 e^{4x}+c_3 e^{-x}
\end{bmatrix}$
\end{minipage} \\

13.a. \begin{minipage}[t]{\dimexpr\linewidth}
Let $M_{22}$ have the inner product in Example 6 of Section 6.1. In each part, find $\parallel A\parallel$. \\
$A=\begin{bmatrix}
5 & 3 \\
2 & -6
\end{bmatrix}\quad\Rightarrow\quad\parallel A\parallel=\sqrt{\left\langle A,\ A \right\rangle}=\sqrt{\mathrm{tr}(A^TA)}=\sqrt{25+9+4+36}=\sqrt{74}$
\end{minipage} \\

15.a. \begin{minipage}[t]{\dimexpr\linewidth}
Let $M_{22}$ have the inner product in Example 7 of Section 6.1. Find $d(A,B)$. \\
$A=\begin{bmatrix}
4 & 2 \\
-3 & 0
\end{bmatrix},\quad B=\begin{bmatrix}
-4 & 3 \\
-2 & 5
\end{bmatrix}\quad\Rightarrow\quad A-B=\begin{bmatrix}
8 & -1 \\
-1 & -5
\end{bmatrix}$ \\
$\Rightarrow d(A,B)=\parallel A-B \parallel=\sqrt{\left\langle A-B,\ A-B \right\rangle}=\sqrt{\mathrm{tr}((A-B)^T(A-B))}=\sqrt{64+1+1+25}=\sqrt{91}$
\end{minipage}

18. \begin{minipage}[t]{\dimexpr\linewidth}
In each part, use the given inner product on $R^2$ to find $\parallel \textbf{w}\parallel$, where $\textbf{w}=(2,-5)$. \\
a. \begin{minipage}[t]{\dimexpr\linewidth}
the Euclidean inner product \\
$\parallel w\parallel=\sqrt{2^2+(-5)^2}=\sqrt{29}$
\end{minipage}
b. \begin{minipage}[t]{\dimexpr\linewidth}
the weighted Euclidean inner product \\
$\left\langle \textbf{u},\textbf{v} \right\rangle=2u_1v_1+3u_2v_2$, where $\textbf{u}=(u_1,u_2)$ and $\textbf{v}=(v_1,v_2)$ \\
$\left\langle \textbf{w},\textbf{w} \right\rangle=2\cdot2^2+3\cdot(-5)^2=83\quad\Rightarrow\quad\parallel\textbf{w}\parallel=\sqrt{\left\langle \textbf{w},\textbf{w} \right\rangle}=\sqrt{83}$
\end{minipage} \\
c. \begin{minipage}[t]{\dimexpr\linewidth}
the inner product generated by the matrix $A=\begin{bmatrix}
-1 & 4 \\
2 & 3
\end{bmatrix}$ \\[-5pt]
$A\textbf{w}=\begin{bmatrix}
-1 & 4 \\
2 & 3
\end{bmatrix}\begin{bmatrix}
2 \\ -5
\end{bmatrix}=\begin{bmatrix}
-22 \\ -11
\end{bmatrix}\quad\Rightarrow\quad\parallel\textbf{w}\parallel=\sqrt{\left\langle \textbf{w},\textbf{w} \right\rangle}=\sqrt{\mathrm{tr}((A\textbf{w})^T\cdot A\textbf{w})}=\sqrt{605}$
\end{minipage}
\end{minipage}

27. \begin{minipage}[t]{\dimexpr\linewidth}
Let $M_{22}$ have the inner product $(U,V)=\mathrm{tr}(U^TV)=\mathrm{tr}(V^TU)$ that was defined in Example 6 of Section 6.1. Describe the orthogonal complement of \\
a. \begin{minipage}[t]{\dimexpr\linewidth}
the subspace of all diagonal matrices. \\
Let $W=\begin{bmatrix}
w_1 & 0 \\
0 & w_4
\end{bmatrix},\quad A=\begin{bmatrix}
a_1 & 0 \\
0 & a_4
\end{bmatrix}\in W,\quad B=\begin{bmatrix}
b_1 & b_2 \\
b_3 & b_4
\end{bmatrix}\in W^\perp\quad\Rightarrow\quad\left\langle A, B \right\rangle=a_1b_1+a_4b_4=0$ \\
Take $a_1=b_1,\quad a_4=b_4\quad\Rightarrow\quad b_1^2+b_4^2=0=b_1=b_4\quad\Rightarrow\quad W^\perp=\left\{\begin{bmatrix}
0 & b_2 \\
b_3 & 0
\end{bmatrix}\mid b_2,b_3\in\mathbb{R}\right\}$
\end{minipage} \\
b. \begin{minipage}[t]{\dimexpr\linewidth}
the subspace of symmetric matrix. \\
Let $W=\begin{bmatrix}
w_1 & w_2 \\
w_2 & w_4
\end{bmatrix},\quad A=\begin{bmatrix}
a_1 & a_2 \\
a_2 & a_4
\end{bmatrix}\in W,\quad B=\begin{bmatrix}
b_1 & b_2 \\
b_3 & b_4
\end{bmatrix}\in W^\perp$ \\
$\Rightarrow\left\langle A, B \right\rangle=a_1b_1+a_2b_2+a_2b_3+a_4b_4=a_1b_1+a_2(b_2+b_3)+a_4b_4=0$ \\
Take $a_1=b_1,\quad a_2=b_2+b_3,\quad a_4=b_4\quad\Rightarrow\quad b_1^2+(b_2+b_3)^2+b_4^2=0=b_1=b_2+b_3=b_4$ \\
$\Rightarrow b_3=-b_2\quad\Rightarrow\quad W^\perp=\left\{\begin{bmatrix}
0 & b_2 \\
-b_2 & 0
\end{bmatrix}\mid b_2\in\mathbb{R}\right\}$
\end{minipage}
\end{minipage}

29. \begin{minipage}[t]{\dimexpr\linewidth}
(Calculus required) Use the inner product $\displaystyle\left\langle \textbf{p},\textbf{q} \right\rangle=\int_{-1}^1p(x)q(x)\ dx$ on $P_3$, to compute $\left\langle \textbf{p},\textbf{q} \right\rangle$. \\
a. \begin{minipage}[t]{\dimexpr\linewidth}
$\textbf{p}=1-x^2+3x^3,\quad\textbf{q}=2-x$ \\
$\displaystyle\left\langle \textbf{p},\textbf{q} \right\rangle=\int_{-1}^1p(x)q(x)\ dx=\int_{-1}^1 -3x^4-7x^3-2x^2-x+2\ dx=\frac{22}{15}$
\end{minipage} \\
b. \begin{minipage}[t]{\dimexpr\linewidth}
$\textbf{p}=x-5x^3,\quad\textbf{q}=2+8x^2$ \\
$\displaystyle\left\langle \textbf{p},\textbf{q} \right\rangle\int_{-1}^1p(x)q(x)\ dx=\int_{-1}^12x-2x^3-40x^5\ dx=0$
\end{minipage}
\end{minipage}

\end{CJK}
\end{document}