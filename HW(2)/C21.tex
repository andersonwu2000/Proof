\documentclass[12pt]{book}
\usepackage[utf8]{inputenc}
\usepackage{color,soul,CJK,epic,tikz,array}
\usepackage{amsmath,amsthm,amssymb}
\usepackage{graphicx}
\usepackage{float}
\usepackage{subfigure}
\setlength{\parindent}{0em}
\linespread{1.3}
\author{andersonwu2000}
\usepackage[margin=1cm]{geometry}
\pagestyle{empty}
\thispagestyle{empty} 

\newcounter{sect}

\newcounter{block}[sect]
\newenvironment{tblock}[1]
{\refstepcounter{block}\theblock.sim\begin{minipage}[t]{\dimexpr\linewidth}#1\\}
{\end{minipage}\\}

\newenvironment{comm}
{\makebox[12pt][l]{$\bullet$}\begin{minipage}[t]{\dimexpr\linewidth}}
{\end{minipage}}

\begin{document}
\begin{CJK}{UTF8}{bsmi}

\hfill 章節 21 吳至堯 U10811023

21.15. Prove that $\mathbb{Q}(\sqrt{2}, \sqrt{3})=\mathbb{Q}(\sqrt{2}+\sqrt{3})$ \\
It is clearly that $\mathbb{Q}(\sqrt{2}+\sqrt{3})\subseteq\mathbb{Q}(\sqrt{2}, \sqrt{3})$ \\
Since $\sqrt{2}+\sqrt{3}$ is a zero of $x^4-10x^2+1$, we have $-(x^3-10x)x=1$ \\
Hence $(\sqrt{2}+\sqrt{3})^{-1}=-((\sqrt{2}+\sqrt{3})^3-10(\sqrt{2}+\sqrt{3}))=\sqrt{3}-\sqrt{2}$ \\
It implies that $\sqrt{2}-\sqrt{3}\in\mathbb{Q}(\sqrt{2}+\sqrt{3})$ \\
Since $\frac{1}{2}((\sqrt{2}+\sqrt{3})+(\sqrt{2}-\sqrt{3}))=\sqrt{2}\in\mathbb{Q}(\sqrt{2}+\sqrt{3})$ and $\frac{1}{2}((\sqrt{2}+\sqrt{3})-(\sqrt{2}-\sqrt{3}))=\sqrt{3}\in\mathbb{Q}(\sqrt{2}+\sqrt{3})$ \\
Therefore, $\mathbb{Q}(\sqrt{2}+\sqrt{3})\supseteq\mathbb{Q}(\sqrt{2}, \sqrt{3})$ and $\mathbb{Q}(\sqrt{2}, \sqrt{3})=\mathbb{Q}(\sqrt{2}+\sqrt{3})$ \\

21.16.a. What is the minimal polynomials for $\sqrt{2}+\sqrt{3}$ \\
Let $x=\sqrt{2}+\sqrt{3}\quad\Rightarrow\quad x^4-10x^2+1=0$ \\
Since $x^4-10x^2+1$ is monic and irreducible over $\mathbb{Q}$ \\
Hence $x^4-10x^2+1$ is the minimal polynomials for $\sqrt{2}+\sqrt{3}$ \\

21.16.b. Express $\sqrt{3}-\sqrt{2}$ as a polynomial $p(x)$ with $x=\sqrt{3}+\sqrt{2}$ over $\mathbb{Q}$ \\
Since $-(x^3-10x)x=1$ and $(\sqrt{2}+\sqrt{3})^{-1}=-((\sqrt{2}+\sqrt{3})^3-10(\sqrt{2}+\sqrt{3}))=\sqrt{2}-\sqrt{3}$ \\
Hence $\sqrt{2}-\sqrt{3}=-x^3+10x$ \\

20.4. Let $\alpha$ be a zero of $f(x)=x^3+x^2+1$ in some extension of $\mathbb{Z}_2$, \\
\hspace*{2em} find the multiplicative inverse of $\alpha+1$ in $\mathbb{Z}_2[\alpha]$ \\
Since $\alpha^3+\alpha^2+1=\alpha^2(\alpha+1)+1=0\quad\Rightarrow\quad\alpha^2(\alpha+1)=1$ \\
Hence $\alpha^2$ is the multiplicative inverse of $\alpha+1$ in $\mathbb{Z}_2[\alpha]$ \\

20.5. Let $\alpha$ be a zero of $f(x)=x^3+x^2+1$ in some extension of $\mathbb{Z}_2$, \\
\hspace*{2em} solve the equation $(\alpha+1)x+\alpha=\alpha^2+1$ for $x$ \\
Since $\alpha^3+\alpha^2+1=0\quad\Rightarrow\quad\alpha^3=\alpha^2+1$ and $\alpha^4=\alpha^3+\alpha=\alpha^2+\alpha+1$ \\
Since $\alpha^2=(\alpha+1)^{-1}$ in $\mathbb{Z}_2[\alpha]\quad\Rightarrow\quad\alpha^2(\alpha+1)x=x=\alpha^2(\alpha^2+\alpha+1)=\alpha^4+\alpha^3+\alpha^2=\alpha^2+\alpha$ \\
Hence $x=\alpha^2+\alpha$

\end{CJK}
\end{document}