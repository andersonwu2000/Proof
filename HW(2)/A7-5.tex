\documentclass[12pt]{book}
\usepackage[utf8]{inputenc}
\usepackage{color,soul,CJK,epic,tikz,array}
\usepackage{amsmath,amsthm,amssymb}
\usepackage{graphicx}
\usepackage{float}
\usepackage{subfigure}
\setlength{\parindent}{0em}
\author{andersonwu2000}
\pagestyle{empty}
\thispagestyle{empty} 

\newcounter{sect}

\newcounter{block}[sect]
\newenvironment{tblock}[1]
{\refstepcounter{block}\theblock.sim\begin{minipage}[t]{\dimexpr\linewidth}#1\\}
{\end{minipage}\\}

\newenvironment{comm}
{\makebox[12pt][l]{$\bullet$}\begin{minipage}[t]{\dimexpr\linewidth}}
{\end{minipage}}

\begin{document}
\begin{CJK}{UTF8}{bsmi}

\hfill 2021/6/10 吳至堯 U10811023 \\

7.21. Let $S(x) = \sum_{k=0}^\infty a_k(x-x_0)^k$ be a power series centered at $x_0$. If $R = 1/\limsup_{k\rightarrow\infty} |a_k|^{1/k}$, with the convention that $1/\infty = 0$ and $1/0 = \infty$, then $R$ is the radius of convergence of $S$. In fact, \\
i. $S(x)$ converges absolutely for each $x\in(x_0-R, x_0+R)$. \\
ii. $S(x)$ converges uniformly on any closed interval $[a, b] \subset(x_0-R, x_0+R)$. \\
iii. $S(x)$ diverges for each $x\not\in[x_0-R, x_0+R]$ where $R$ is finite. \\
$Proof$. Let $x\in\mathbb{R}$ with $x\ne x_0$, $\rho = 1/\limsup_{k\rightarrow\infty} |a_k|^{1/k}$ and 
\[
    r(x) 
    = \limsup_{k\rightarrow\infty} |a_k(x-x_0)^k|^{1/k} 
    = |x-x_0|\cdot\limsup_{k\rightarrow\infty} |a_k|^{1/k}.
\]
If $\rho = 0$, then $\limsup_{k\rightarrow\infty} |a_k|^{1/k} = \infty$ and $r(x) = \infty > 1$. \\
By Root Test, $S(x) = \sum_{k=0}^\infty a_k(x-x_0)^k$ converges only when $x=x_0$. \\
Hence $R = 0 = \rho$. \\
If $\rho = \infty$, then $\limsup_{k\rightarrow\infty} |a_k|^{1/k} = 0$ and $r(x) = 0 < 1$. \\
By Root Test, $S(x)$ converges absolutely on $R$. \\
Hence $R = \infty = \rho$. \\
If $\rho\in(0, \infty)$, then $r(x)=|x-x_0|/\rho$. \\
Suppose that $x\in(x_0-\rho, x_0+\rho)$, then $|x-x_0| < \rho$ and $r(x)<1$. \\
By Root Test, $S(x)$ converges when $x\in(x_0-\rho, x_0+\rho)$. \\
Suppose that $x\not\in[x_0-\rho, x_0+\rho]$, then $|x-x_0| > \rho$ and $r(x)>1$. \\
By Root Test, $S(x)$ diverges when $x\not\in[x_0-\rho, x_0+\rho]$. \\
Hence $\rho$ is the radius of convergence of $S$. \\
Let $[a, b]\subset(x_0-R, x_0+R)$, $x_1\in(x_0-R, a)\cap(b, x_0+R)$, then $|x-x_0| \le |x_1-x_0|$. \\
Let $M_k = |a_k||x_1-x_0|^k$ for $k\in\mathbb{N}$. \\
Since $x_1\in(x_0-R, x_0+R)$, then $\sum_{k=0}^\infty M_k = \sum_{k=0}^\infty |a_k||x_1-x_0|^k$ converges. \\
Since $|a_k(x-x_0)^k|\le M_k$ for $x\in[a, b]$ and $k\in\mathbb{N}$, by Theorem 7.15, $S$ converges uniformly on $[a, b]$. \\
Hence the proof is complete. \\

7.22. If the limit 
\[
    R = \lim_{k\rightarrow\infty}\frac{|a_k|}{|a_{k+1}|}
\]
exists as an extended number, then $R$ is the radius of convergence of power series $S(x) = \sum_{k=0}^\infty a_k(x-x_0)^k$. \\
$Proof$. Let $x\in\mathbb{R}$ with $x\ne x_0$, $\rho = \lim_{k\rightarrow\infty}|a_k|/|a_{k+1}|$ and 
\[
    r(x) 
    = \lim_{k\rightarrow\infty} \frac{|a_{k+1}(x-x_0)^{k+1}|}{|a_k(x-x_0)^k|}
    = |x-x_0|\cdot\lim_{k\rightarrow\infty} \frac{|a_{k+1}|}{|a_k|}.
\]
If $\rho = 0$, then $\lim_{k\rightarrow\infty}|a_{k+1}|/|a_k| = \infty$ and $r(x) = \infty > 1$. \\
By Ratio Test, $S(x)$ converges only when $x=x_0$. \\
Hence $R = 0 = \rho$. \\
If $\rho = \infty$, then $\lim_{k\rightarrow\infty}|a_{k+1}|/|a_k| = 0$ and $r(x) = 0 < 1$. \\
By Ratio Test, $S(x)$ converges absolutely on $R$. \\
Hence $R = \infty = \rho$. \\
If $\rho\in(0, \infty)$, then $r(x) = |x-x_0|/\rho$. \\
Suppose that $x\in(x_0-\rho, x_0+\rho)$, then $|x-x_0| < \rho$ and $r(x)<1$. \\
By Ratio Test, $S(x)$ converges when $x\in(x_0-\rho, x_0+\rho)$. \\
Suppose that $x\not\in[x_0-\rho, x_0+\rho]$, then $|x-x_0| > \rho$ and $r(x)>1$. \\
By Ratio Test, $S(x)$ diverges when $x\not\in[x_0-\rho, x_0+\rho]$. \\
Hence $R$ is the radius of convergence of $S$. \\

7.26. If $f(x)=\sum_{k=0}^\infty a_k(x-x_0)^k$ is a power series with positive radius of convergence $R$, then $f$ is continuous on $(x_0-R, x_0+R)$. \\
$Proof$. Let $x_1\in(x_0-R, x_0+R)$, then $\exists\ a, b\in\mathbb{R}$ such that $x_0-R < a < x_1$ and $x_1 > b > x_0+R$. \\
By Theorem 7.21.i, f converges uniformly on $[a, b]$. \\
Since $a_k(x-x_0)^k$ is continuous at $x_1$ for all $k\in\mathbb{N}$, by Theorem 7.14.i, $f$ is continuous at $x_1$. \\
Hence $f$ is continuous on $(x_0-R, x_0+R)$. \\

\clearpage

7.27. Suppose that $[a, b]$ is nondegenerate. If $f(x) := \sum_{k=0}^\infty a_k(x-x_0)^k$ converges on $[a, b]$, then $f(x)$ is continuous and converges uniformly on $[a, b]$. \\
$Proof$. Let $\varepsilon>0$. \\
Since $[a, b]$ is nondegenerate and $f$ converges on $[a, b]$, then $\exists\ x_0\in\mathbb{R}$ and $R\in(0, \infty)$ such that $f$ converges on $(x_0-R, x_0+R)$ and diverges for $|x-x_0|>R$, it follows that $[a, b]\subset[x_0-R, x_0+R]$. \\
Let $x_1\in[a, b]$ with $x_1\ne x_0$ and
\[
    g(x) = \left\{\begin{matrix}
  b - x_0 & x_0 < x \\
  a - x_0 & x \le x_0.
\end{matrix}\right.
\]
Set $b_k = a_k(g(x_1))^k$, $c_k = (x_1-x_0)^k/(g(x_1))^k$ for $k\in\mathbb{N}$. \\
Since $f$ converges on $[a, b]$, then $\sum_{k=1}^\infty b_k = \sum_{k=1}^\infty a_k(g(x_1))^k$ converges and $\exists\ N\in\mathbb{N}$ with $N>1$ such that 
\[
    k>m\ge N\quad\text{imply}\quad
    \left| \sum_{j=m}^k b_j \right|
    < \varepsilon.
\]
If $x_0 < x_1$, then $0 < x_1-x_0 \le b-x_0 = g(x_1)$ and $c_k$ is decreasing. \\
If $x_1 < x_0$, then $a-x_0 = g(x_1) \le x_1-x_0 < 0$ and $c_k$ is decreasing. \\
By Abel's Formula, 
\begin{eqnarray*}
\left| \sum_{k=m}^n a_k(x_1-x_0)^k \right| 
    & = & \left| \sum_{k=m}^n b_k c_k \right| \\
    & = & \left| c_n\sum_{k=m}^n b_k - \sum_{k=m}^{n-1} (c_{k+1}-c_k)\sum_{j=m}^k b_j \right| \\
    & < & \left| c_n\varepsilon - \varepsilon\sum_{k=m}^{n-1} (c_{k+1}-c_k) \right| \\
    & = & c_n\varepsilon + (c_m - c_n)\varepsilon = c_m\varepsilon.
\end{eqnarray*}
Since $0 < c_m \le c_1 = (x_1-x_0)/g(x_1) \le 1$, then 
\[
    \left| \sum_{k=m}^n a_k(x_1-x_0)^k \right| < c_m\varepsilon \le \varepsilon.
\]
Since $\left| \sum_{k=m}^n a_k(x_0-x_0)^k \right| = 0 < \varepsilon$, then $f$ converges uniformly on $[a, b]$. \\
Hence the proof is complete. \\

\end{CJK}
\end{document}