\documentclass[12pt]{book}
\usepackage[utf8]{inputenc}
\usepackage{color,soul,CJK,epic,tikz,array}
\usepackage{amsmath,amsthm,amssymb}
\usepackage{graphicx}
\usepackage{float}
\usepackage{subfigure}
\setlength{\parindent}{0em}
\linespread{1.3}
\author{andersonwu2000}
\usepackage[margin=1cm]{geometry}
\pagestyle{empty}
\thispagestyle{empty} 

\newcounter{sect}

\newcounter{block}[sect]
\newenvironment{tblock}[1]
{\refstepcounter{block}\theblock.sim\begin{minipage}[t]{\dimexpr\linewidth}#1\\}
{\end{minipage}\\}

\newenvironment{comm}
{\makebox[12pt][l]{$\bullet$}\begin{minipage}[t]{\dimexpr\linewidth}}
{\end{minipage}}

\begin{document}
\begin{CJK}{UTF8}{bsmi}

\hfill 章節 5 吳至堯 U10811023

% 09, 23, 35, 58

5-9. \begin{minipage}[t]{\dimexpr\linewidth-2em}
What are the possible orders for the elements of $S_6$ and $A_6$? What about $A_7$? (This exercise is referred to in Chapter 25.) \\
Denote an $n$-cycle by $(\underline{n})$ \\
Arranging all possible disjoint cycle structures of elements of $S_6$ : \\
\begin{tabular}{lc|lc|lc|lc}
    $(\underline{6})$ & $\Rightarrow 6$ &
    $(\underline{5})(\underline{1})$ & $\Rightarrow 5$ &
    $(\underline{4})(\underline{2})$ & $\Rightarrow 4$ &
    $(\underline{4})(\underline{1})(\underline{1})$ & $\Rightarrow 4$ \\
    $(\underline{3})(\underline{3})$ & $\Rightarrow 3$ &
    $(\underline{3})(\underline{2})(\underline{1})$ & $\Rightarrow 6$ &
    $(\underline{3})(\underline{1})(\underline{1})(\underline{1})$ & $\Rightarrow 3$ &
    $(\underline{2})(\underline{2})(\underline{2})$ & $\Rightarrow 2$ \\
    $(\underline{2})(\underline{2})(\underline{1})(\underline{1})$ & $\Rightarrow 2$ &
    $(\underline{2})(\underline{1})(\underline{1})(\underline{1})(\underline{1})$ & $\Rightarrow 2$ &
    $(\underline{1})(\underline{1})(\underline{1})(\underline{1})(\underline{1})(\underline{1})$ & $\Rightarrow 1$ 
    \end{tabular} \\
The orders of the elements of $S_6$ are $6, 5, 4, 3, 2, 1$ \\\\
For each even permutation $\alpha=(a_1 a_2 \cdots a_k)=(a_1a_k)(a_1a_{k-1})\cdots(a_1a_2),\quad k-2+1$ is even \\
Therefore, $k$ is odd \\
Arranging all possible disjoint cycle structures of elements of $A_6$ : \\
\begin{tabular}{lc|lc|lc|lc}
    $(\underline{5})(\underline{1})$ & $\Rightarrow 5$ &
    $(\underline{4})(\underline{2})$ & $\Rightarrow 4$ &
    $(\underline{3})(\underline{3})$ & $\Rightarrow 3$ &
    $(\underline{3})(\underline{2})(\underline{1})$ & $\Rightarrow 6$ \\
    $(\underline{3})(\underline{1})(\underline{1})(\underline{1})$ & $\Rightarrow 3$ &
    $(\underline{2})(\underline{2})(\underline{2})$ & $\Rightarrow 2$ &
    $(\underline{2})(\underline{2})(\underline{1})(\underline{1})$ & $\Rightarrow 2$ &
    $(\underline{2})(\underline{1})(\underline{1})(\underline{1})(\underline{1})$ & $\Rightarrow 2$ \\
    $(\underline{1})(\underline{1})(\underline{1})(\underline{1})(\underline{1})(\underline{1})$ & $\Rightarrow 1$ 
    \end{tabular} \\
The orders of the elements of $A_6$ are $5, 4, 3, 2, 1$ \\
Arranging all possible disjoint cycle structures of elements of $A_7$ : \\
\begin{tabular}{lc|lc|lc|lc}
    $(\underline{7})$ & $\Rightarrow 7$ &
    $(\underline{5})(\underline{1})(\underline{1})$ & $\Rightarrow 5$ &
    $(\underline{4})(\underline{2})(\underline{1})$ & $\Rightarrow 4$ &
    $(\underline{3})(\underline{3})(\underline{1})$ & $\Rightarrow 3$ \\
    $(\underline{3})(\underline{2})(\underline{2})$ & $\Rightarrow 6$ &
    $(\underline{3})(\underline{1})(\underline{1})(\underline{1})(\underline{1})$ & $\Rightarrow 3$ &
    $(\underline{2})(\underline{2})(\underline{1})(\underline{1})(\underline{1})$ & $\Rightarrow 2$ &
    $(\underline{1})(\underline{1})(\underline{1})(\underline{1})(\underline{1})(\underline{1})(\underline{1})$ & $\Rightarrow 1$ 
    \end{tabular} \\
The orders of the elements of $A_6$ are $7, 6, 5, 4, 3, 2, 1$
\end{minipage} \\

5-23. \begin{minipage}[t]{\dimexpr\linewidth-2em}
Show that if $H$ is a subgroup of $S_n$, then either every member of $H$ is an even permutation or exactly half of the members are even. (This exercise is referred to in Chapter 25.) \\
If $\forall x\in H$, $x$ is even, then we are done \\
Assume that $\exists\ x\in H$ s.t. $x$ is odd \\
$\Rightarrow\quad\forall y\in H$ and $y$ is even, $xy$ is odd \\
$\Rightarrow\quad\forall z\in H$ and $z$ is odd, $xz$ is even \\
Hence, if $\exists\ x\in H$ s.t. $x$ is odd, then the number of even and odd permutation in $H$ is the same
\end{minipage} \\

5-35. \begin{minipage}[t]{\dimexpr\linewidth-2em}
Let $G$ be a group of permutations on a set $X$. Let $a \in X$ and define
$\mathrm{stab}(a) = \{\alpha \in G | \alpha(a) = a\}$. We call $\mathrm{stab}(a)$ the stabilizer of $a$ in $G$ (since it consists of all members of $G$ that leave $a$ fixed). Prove that $\mathrm{stab}(a)$ is a subgroup of $G$. (This subgroup was introduced by Galois in 1832.) This exercise is referred to in Chapter 7. \\
Let $a\in X,\quad\alpha, \beta\in\mathrm{stab}(a)\quad\Rightarrow\quad \alpha(a)=\beta(a)=a$ \\
Since $\beta\in G\quad\Rightarrow\quad \beta^{-1}\in G$ and $\beta^{-1}\beta=\varepsilon\in\mathrm{stab}(a)$ \\
Since $\varepsilon(a)=\beta^{-1}(\beta(a))=\beta^{-1}(a)=a\quad\Rightarrow\quad\beta^{-1}\in\mathrm{stab}(a)$ \\
Since $\alpha(a)=a\quad\Rightarrow\quad\alpha(\beta^{-1}(a))=\alpha(a)=a\quad\Rightarrow\quad\alpha\beta^{-1}\in\mathrm{stab}(a)$ \\
Hence, $\mathrm{stab}(a)\le G$
\end{minipage} \\

5-58. \begin{minipage}[t]{\dimexpr\linewidth-2em}
Show that for $n\ge3, Z(S_n)=\{\varepsilon\}$ \\
Since $\forall\gamma\in S_n,\ \varepsilon\gamma=\gamma\varepsilon\quad\Rightarrow\quad\varepsilon\in Z(S_n)\ne\varnothing$ \\
Suppose that $Z(S_n)\ne\{\varepsilon\}\quad\Rightarrow\quad\exists\ \alpha\in Z(S_n)$ s.t. $\alpha\ne\varepsilon$ \\
$\Rightarrow\quad\exists\ a_1, a_2\in\mathbb{N}_n$ and $a_1\ne a_2$ s.t. $\alpha(a_1)=a_2$ \\
Set $\beta\in S_n,\ a_3\in\mathbb{N}_n,\ a_3\notin\{a_1, a_2\}$ s.t. $\beta=(a_1)(a_2a_3)$ \\
$\Rightarrow\quad\alpha(\beta(a_1))=\alpha(a_1)=a_2\ne\beta(\alpha(a_1))=\beta(a_2)=a_3$ \\
Hence $Z(S_n)=\{\varepsilon\}$
\end{minipage} \\

\end{CJK}
\end{document}