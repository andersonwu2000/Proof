\documentclass[12pt]{book}
\usepackage[utf8]{inputenc}
\usepackage{color,soul,CJK,epic,tikz,array}
\usepackage{amsmath,amsthm,amssymb}
\usepackage{graphicx}
\usepackage{float}
\usepackage{subfigure}
\setlength{\parindent}{0em}
\linespread{1.3}
\author{andersonwu2000}
\usepackage[margin=1cm]{geometry}
\pagestyle{empty}
\thispagestyle{empty} 

\newcounter{sect}

\newcounter{block}[sect]
\newenvironment{tblock}[1]
{\refstepcounter{block}\theblock.sim\begin{minipage}[t]{\dimexpr\linewidth}#1\\}
{\end{minipage}\\}

\newenvironment{comm}
{\makebox[12pt][l]{$\bullet$}\begin{minipage}[t]{\dimexpr\linewidth}}
{\end{minipage}}

\begin{document}
\begin{CJK}{UTF8}{bsmi}

\hfill 章節 2, 3 吳至堯 U10811023

2-22. \begin{minipage}[t]{\dimexpr\linewidth-2em}
Let $G$ be a group with the property that for any $x, y, z$ in the group, $xy = zx$ implies $y = z$. Prove that $G$ is Abelian. \\
Suppose that $xy=zx\Rightarrow y=z$ for all $x,y,z\in$ group $G$ \\
Assume that $a,b,c\in G$ and $ab=ca\quad\Rightarrow\quad b=c\quad\Rightarrow\quad ba=ca$ \\
Since $ab=ca$ and $ba=ca\quad\Rightarrow\quad ab=ba$ \\
Hence $G$ is Abelian.
\end{minipage}\\

3-1. \begin{minipage}[t]{\dimexpr\linewidth-2em}
For each group in the following list, find the order of the group and the order of each element in the group. What relation do you see between the orders of the elements of a group and the order of the group?
\end{minipage}\\

3-1-a. \begin{minipage}[t]{\dimexpr\linewidth-2em}
$\mathbb{Z}_{12}=\{0,1,2,3,4,5,6,7,8,9,10,11\}\quad\Rightarrow\quad|\mathbb{Z}_{12}|=12$ \\
\begin{tabular}{ccc|ccc|ccc}
    $0\cdot1=0$ & $\Rightarrow$ & $|0|=1$ & $1\cdot12=0$ & $\Rightarrow$ & $|1|=12$ & $2\cdot6=0$ & $\Rightarrow$ & $|2|=6$ \\
    $3\cdot4=0$ & $\Rightarrow$ & $|3|=4$ & $4\cdot3=0$ & $\Rightarrow$ & $|4|=3$ & $5\cdot12=0$ & $\Rightarrow$ & $|5|=12$ \\
    $6\cdot2=0$ & $\Rightarrow$ & $|6|=2$ & $7\cdot12=0$ & $\Rightarrow$ & $|7|=12$ & $8\cdot3=0$ & $\Rightarrow$ & $|8|=3$ \\
    $9\cdot4=0$ & $\Rightarrow$ & $|9|=4$ & $10\cdot6=0$ & $\Rightarrow$ & $|10|=6$ & $11\cdot12=0$ & $\Rightarrow$ & $|11|=12$ \\
\end{tabular}
\end{minipage}\\

3-1-b. \begin{minipage}[t]{\dimexpr\linewidth-2em}
$U(10)=\{1,3,7,9\}\quad\Rightarrow\quad|U(10)|=4$ \\
$1^1=1 \Rightarrow |1|=1,\quad 3^4=1 \Rightarrow |3|=4,\quad 7^4=1 \Rightarrow |7|=4,\quad 9^2=1 \Rightarrow |9|=2$ \\
\end{minipage}\\

3-1-c. \begin{minipage}[t]{\dimexpr\linewidth-2em}
$U(12)=\{1,5,7,11\}\quad\Rightarrow\quad|U(12)|=4$ \\
$1^1=1 \Rightarrow |1|=1,\quad 5^2=1 \Rightarrow |5|=2,\quad 7^2=1 \Rightarrow |7|=2,\quad 11^2=1 \Rightarrow |11|=2$ \\
\end{minipage}\\

3-1-d. \begin{minipage}[t]{\dimexpr\linewidth-2em}
$U(20)=\{1,3,7,9,11,13,17,19\}\quad\Rightarrow\quad|U(20)|=8$ \\
$1^1=1 \Rightarrow |1|=1,\quad 3^4=1 \Rightarrow |3|=4,\quad 7^4=1 \Rightarrow |7|=4,\quad 9^2=1 \Rightarrow |9|=2$ \\
$11^2=1 \Rightarrow |11|=2,\quad 13^4=1 \Rightarrow |13|=4,\quad 17^4=1 \Rightarrow |17|=4,\quad 19^2=1 \Rightarrow |19|=2$ \\
\end{minipage}\\

3-1-e. \begin{minipage}[t]{\dimexpr\linewidth-2em}
$D_4=\{R_0, R_{90}, R_{180}, R_{270}, H, V, D, D'\}\quad\Rightarrow\quad|D_4|=8$ \\
$R_0^1=R_0 \Rightarrow |R_0|=1,\quad R_{90}^4=R_0 \Rightarrow |R_{90}|=4,\quad R_{180}^2=R_0 \Rightarrow |R_{180}|=2,\quad R_{270}^4=R_{0} \Rightarrow |R_{270}|=4$ \\
$H^2=R_0 \Rightarrow |H|=2,\quad V^2=R_0 \Rightarrow |V|=2,\quad D^2=R_0 \Rightarrow |D|=2,\quad {D'}^2=R_0 \Rightarrow |D'|=2$ \\
\end{minipage}\\

3-1-f. \begin{minipage}[t]{\dimexpr\linewidth-2em}
If $G$ is a group and $g\in G$, then $\exists\ a\in\mathbb{Z}$ such that $a|g|=|G|$
\end{minipage}\\

\clearpage

3-17. \begin{minipage}[t]{\dimexpr\linewidth-2em}
For each divisor $k>1$ of $n$, let $U_k(n)=\{x\in U(n)\mid x\mod k=1\}$. \\
List the elements of $U_4(20), U_5(20), U_5(30),$ and $U_{10}(30)$. \\
Prove that $U_k(n)$ is a subgroup of $U(n)$. \\
Let $H=\{x\in U(10)\mid x\mod 3=1\}$. Is
$H$ a subgroup of $U(10)$? \\
\end{minipage}\\

3-17-a. \begin{minipage}[t]{\dimexpr\linewidth-2em}
$U_4(20)=\{1,9,13,17\},\quad U_5(20)=\{1,11\},\quad U_5(30)=\{1,11\},\quad U_{10}(30)=\{1,11\}$ \\
\end{minipage}\\

3-17-b. \begin{minipage}[t]{\dimexpr\linewidth-2em}
Since $\gcd(1, n)=1$ and $(1\mod k)=1$ for all $n,k\in\mathbb{N}\quad\Rightarrow\quad 1\in U_k(n)\quad\Rightarrow\quad U_k(n)\ne\varnothing$ \\
Assume that $x\in U(n)\quad\Rightarrow\quad\gcd(x, n)=\gcd(x\cdot1, n)=1\quad\Rightarrow\quad 1$ is the identity of $U(n)$ and $U_k(n)$ \\
Let $a,b\in U_k(n)\quad\Rightarrow\quad\gcd(a,n)=\gcd(b,n)=1$ and $(a\mod k)=(b\mod k)=1$ \\
$\Rightarrow\quad\exists\ q,r\in\mathbb{Z}$ such that $ab^{-1}=qk+r,\ 0\le r<k\quad\Rightarrow\quad a=ab^{-1}b=bqk+br$ \\
Since $(b\mod k)=1\quad\Rightarrow\quad\exists\ p\in\mathbb{Z}$ such that $b=pk+1\quad\Rightarrow\quad br=rpk+r$ \\
$\Rightarrow\quad (a\mod k)=(bqk+br\mod k)=(br\mod k)=(rpk+r\mod k)=(r\mod k)=1$ \\
$\Rightarrow\quad\exists\ x\in\mathbb{Z}$ such that $r=xk+1\quad\Rightarrow\quad ab^{-1}=bqk+rpk+xk+1$ \\
$\Rightarrow\quad (ab^{-1}\mod k)=(bqk+rpk+xk+1\mod k)=1\quad\Rightarrow\quad ab^{-1}\in U_k(n)$
\end{minipage}\\

3-17-a. \begin{minipage}[t]{\dimexpr\linewidth-2em}
$U(10)=\{1,3,7,9\},\quad H=U_3(10)=\{1,7\}$ \\
Since $7\cdot7=49=9\notin U_3(10)$ \\
Hence $H$ is not a subgroup of $U(10)$.
\end{minipage}\\

\end{CJK}
\end{document}