\documentclass[12pt]{book}
\usepackage[utf8]{inputenc}
\usepackage{color,soul,CJK,epic,tikz,array}
\usepackage{amsmath,amsthm,amssymb}
\usepackage{graphicx}
\usepackage{float}
\usepackage{subfigure}
\setlength{\parindent}{0em}
\linespread{1.3}
\author{andersonwu2000}
\usepackage[margin=1cm]{geometry}
\pagestyle{empty}
\thispagestyle{empty} 

\newcounter{sect}

\newcounter{block}[sect]
\newenvironment{tblock}[1]
{\refstepcounter{block}\theblock.sim\begin{minipage}[t]{\dimexpr\linewidth}#1\\}
{\end{minipage}\\}

\newenvironment{comm}
{\makebox[12pt][l]{$\bullet$}\begin{minipage}[t]{\dimexpr\linewidth}}
{\end{minipage}}

\begin{document}
\begin{CJK}{UTF8}{bsmi}

\hfill 2021/5/27 吳至堯 U10811023

7.3. The pointwise limit of continuous (respectively, differentiable) functions is not necessarily continuous (respectively, differentiable). \\
$Proof$. Let $\varepsilon>0$, $f_n(x)=x^n$ and $\displaystyle f(x)=\left\{\begin{matrix}
  0 & 0\le x<1 \\
  1 & x=1
\end{matrix}\right.$ \\
Suppose that $x\in\left[0, 1\right)$, set $\displaystyle N=\max\{\left \lceil \frac{\ln{\varepsilon}}{\ln{x}} \right \rceil +1, 1\}\in\mathbb{N}$ \\
If $n\ge N\quad\Rightarrow\quad 0\le x^n\le x^N<x^{\ln{\varepsilon}/\ln{x}}=\varepsilon\quad\Rightarrow\quad|x^n|<\varepsilon$ \\
Suppose that $x=1\quad\Rightarrow\quad x^n=1$ for all $n\in\mathbb{N}$ \\
$\Rightarrow\quad f_n$ is converge pointwise to $f$ on $\left[0, 1\right]$ \\
Since $f_n$ is continuous and differentiable on $\left[0, 1\right]$ for all $n\in\mathbb{N}$ but $f$ is not \\
Hence the proof is complete \\

7.4. The pointwise limit of integrable functions is not necessarily integrable \\
$Proof$. Let $\varepsilon>0$, $\displaystyle f(x)=\left\{\begin{matrix}
  1 & x\in\mathbb{Q} \\
  0 & \text{otherwise}
\end{matrix}\right.$\quad for $n\in\mathbb{N}$ and \\
\[ f_n(x)=\left\{\begin{matrix}
  1 & x=p/m\in\mathbb{Q}\text{, written in reduced form, where }m\le n \\
  0 & \text{otherwise}
\end{matrix}\right.\quad\text{ for }n\in\mathbb{N}\]
Suppose that $x\in\left[0, 1\right]$ \\
If $x\in\mathbb{Q}\quad\Rightarrow\quad\exists\ p, m\in\mathbb{N}$ s.t. $x=p/m$ and $\gcd(p, m)=1$ \\
Set $N=m+1\in\mathbb{N}\quad\Rightarrow\quad m<m+1=N\le n$ if $n\ge N\quad\Rightarrow\quad f_n(x)=1=f(x)$ if $n\ge N$ \\
If $x\notin\mathbb{Q}\quad\Rightarrow\quad f_n(x)=0=f(x)$ \\
Therefore, $f_n\rightarrow f$ pointwise on $\left[0, 1\right]$ \\
Let $n\in\mathbb{N}$ and $E_n=\{x\in\left[0, 1\right]:f(x)=1\}$ \\
Since $E_n$ is finite $\quad\Rightarrow\quad f_n$ is integrable on $\left[0, 1\right]$ but $f$ is not  \\
Hence the proof is complete \\

7.5. There exist differentiable functions $f_n$ and $f$ such that $f_n\rightarrow f$ pointwise on $\left[0, 1\right]$ but 
\[
\lim_{n\rightarrow\infty}f'_n(x)\ne\left(\lim_{n\rightarrow\infty}f_n(x)\right)'\quad\text{for } x=1
\]
$Proof$. Let $\varepsilon>0$, $f_n(x)=x^n/n$ and $f(x)=0$, suppose that $x\in\left[0, 1\right]$ \\
Since $0<x^n\le1$ for all $n\in\mathbb{N}$ and $1/n\rightarrow0$ as $n\rightarrow\infty\quad\Rightarrow\quad f_n\rightarrow f$ pointwise on $\left[0, 1\right]$ \\
Since $f'_n(x)=x^{n-1}$ for all $n\in\mathbb{N}$ and $f'(x)=0\displaystyle\Rightarrow\quad\lim_{n\rightarrow\infty}f'_n(x)=\lim_{n\rightarrow\infty}1=1\ne\left(\lim_{n\rightarrow\infty}f_n(x)\right)'=0$ for $x=1$ \\
Hence the proof is complete \\

\clearpage

7.6. There exist continuous functions $f_n$ and $f$ such that $f_n\rightarrow f$ pointwise on $\left[0, 1\right]$ but \[\lim_{n\rightarrow\infty}\int_0^1f_n(x)\ dx\ne\int_0^1\left(\lim_{n\rightarrow\infty}f_n(x)\right)\ dx\]
$Proof$. Let $f(x)=0$, $f_1(x)=1$ and $\displaystyle f_n(x)=\left\{\begin{matrix}
  n^2x & 0\le x<1/n \\
  2n-n^2x & 1/n\le x<2/n \\
  0 & 2/n\le x\le1
\end{matrix}\right.$\quad for $n>1$ \\
Suppose that $x\in\left[0, 1\right]$, set $\displaystyle N=\left\lceil 2/x \right\rceil+1\in\mathbb{N}$ \\
If $n\ge N\quad\Rightarrow\quad 2/n\le2/N<2/(2/x)= x\quad\Rightarrow\quad x=0\quad\Rightarrow\quad|x|<\varepsilon$ for all $\varepsilon>0$ \\
Therefore, $f_n\rightarrow f$ pointwise on $\left[0, 1\right]$ \\
Since $\displaystyle\int_0^1f_n(x)\ dx=\int_0^{1/n}f_n(x)\ dx+\int_{1/n}^{2/n}f_n(x)\ dx=\left.2nx-\frac{1}{2}n^2x^2\right|_{1/n}^{2/n}+\frac{1}{2}=1$ for all $n\in\mathbb{N}$ \\
$\displaystyle\Rightarrow\quad\lim_{n\rightarrow\infty}\int_0^1f_n(x)\ dx=1\ne\int_0^1\left(\lim_{n\rightarrow\infty}f_n(x)\right)\ dx=\int_0^1f(x)\ dx=0$ \\
Hence the proof is complete \\

7.8. Prove that $x^n\rightarrow0$ uniformly on $\left[0, b\right]$ for any $b<1$, and pointwise, but not uniformly, on $\left[0, 1\right)$ \\
$Proof$. According to part of the proof of Remark 7.3, $x^n\rightarrow0$ pointwise on $\left[0, 1\right)$ \\
Let $0\le b<1$ and $\varepsilon>0$, set $\displaystyle N=\max\{\left \lceil \frac{\ln{\varepsilon}}{\ln{b}} \right \rceil +1, 1\}\in\mathbb{N}$ \\
If $n\ge N\quad\Rightarrow\quad 0\le b^n\le b^N<b^{\ln{\varepsilon}/\ln{b}}=\varepsilon\quad\Rightarrow\quad|x^n|\le b^n<\varepsilon$ if $x\in\left[0, b\right]$ \\
Therefore, $x^n\rightarrow 0$ uniformly on $\left[0, b\right]$ \\
Suppose that $x^n\rightarrow 0$ uniformly on $\left[0, 1\right)\quad\Rightarrow\quad\exists\ N\in\mathbb{N}$ s.t. $|x^n|<1/2$ for all $n\ge N$, $x\in\left[0, 1\right)$ \\
If $x=2^{-1/N}\quad\Rightarrow\quad x\in\left[0, 1\right)$ and $x^N=1/2\quad\rightarrow\leftarrow$ \\
Hence $x^n\rightarrow0$ uniformly on $\left[0, b\right]$ for any $b<1$, and pointwise, but not uniformly, on $\left[0, 1\right)$

\end{CJK}
\end{document}