\documentclass[12pt]{book}
\usepackage[utf8]{inputenc}
\usepackage{color,soul,CJK,epic,tikz,array}
\usepackage{amsmath,amsthm,amssymb}
\usepackage{graphicx}
\usepackage{float}
\usepackage{subfigure}
\setlength{\parindent}{0em}
\linespread{1.3}
\author{andersonwu2000}
\usepackage[margin=1cm]{geometry}
\pagestyle{empty}
\thispagestyle{empty} 

\newcounter{sect}

\newcounter{block}[sect]
\newenvironment{tblock}[1]
{\refstepcounter{block}\theblock.sim\begin{minipage}[t]{\dimexpr\linewidth}#1\\}
{\end{minipage}\\}

\newenvironment{comm}
{\makebox[12pt][l]{$\bullet$}\begin{minipage}[t]{\dimexpr\linewidth}}
{\end{minipage}}

\begin{document}
\begin{CJK}{UTF8}{bsmi}

\hfill 章節 15 吳至堯 U10811023

2. Is the mapping from $\mathbb{Z}_5$ to $\mathbb{Z}_{30}$ given by $x\rightarrow6x$ a ring homomorphism? \\
Let $x, y\in \mathbb{Z}_5,\  \phi(x):\mathbb{Z}_5\rightarrow\mathbb{Z}_{30}:\phi(x)=6x$ \\
Since $\phi(x+y)=6(x+y)=6x+6y=\phi(x)+\phi(y)$ and $\phi(xy)=6xy\equiv36xy=6x6y=\phi(x)\phi(y)$ \\
Hence $\phi$ is a ring homomorphism from $\mathbb{Z}_5$ to $\mathbb{Z}_{30}$ \\

3. Is the mapping from $\mathbb{Z}_{10}$ to $\mathbb{Z}_{10}$ given by $x\rightarrow2x$ a ring homomorphism? \\
Let $x, y\in \mathbb{Z}_{10},\  \phi(x):\mathbb{Z}_{10}\rightarrow\mathbb{Z}_{10}:\phi(x)=2x$ \\
Since $\phi(x\cdot y)=2xy\ne4xy=2x2y=\phi(x)\phi(y)$ \\
Hence $\phi$ is not a ring homomorphism from $\mathbb{Z}_{10}$ to $\mathbb{Z}_{10}$ \\

4.a. Determine all ring homomorphism from $\mathbb{Z}_6$ to $\mathbb{Z}_6$. \\
Since $\phi(1)=\phi(1)\phi(1)$ and $\forall x\in\mathbb{Z}_6, x^2=x$ iff $x\in\{0, 1, 3, 4\}\quad\Rightarrow\quad\phi(1)\in\{0, 1, 3, 4\}$ \\
Let $x, y\in\mathbb{Z}_6,\ \phi(1)\in\{0, 1, 3, 4\}$ \\
Since $\forall x\in\mathbb{Z}_6, \phi(x)=\phi(\underset{x}{\underbrace{1+1+\cdots+1}})=x\phi(1)$\hfill$(\phi(0)=\phi(0)+\phi(0)\quad\Rightarrow\quad\phi(0)=0=0\phi(1))$ \\
$\Rightarrow\quad\phi(x+y)=(x+y)\phi(1)=\phi(1)x+\phi(1)y=\phi(x)+\phi(y)$ and $\phi(xy)=xy\phi(1)=\phi(1)x\phi(1)y=\phi(x)\phi(y)$ \\
Hence all ring homomorphism from $\mathbb{Z}_6$ to $\mathbb{Z}_6$ are $\phi(x)=0,\ \phi(x)=x,\ \phi(x)=3x,\ \phi(x)=4x$ \\

4.b. Determine all ring homomorphism from $\mathbb{Z}_{20}$ to $\mathbb{Z}_{30}$. \\
Since $\phi(1)=\phi(1)\phi(1)$ and $\forall x\in\mathbb{Z}_{30}, x^2=x$ iff $x\in\{0, 1, 6, 10, 15, 16, 21, 25\}$ \\
$\Rightarrow\quad\phi(1)\in\{0, 1, 6, 10, 15, 16, 21, 25\}$ \\
Let $x, y\in\mathbb{Z}_{20},\ \phi(1)\in\{0, 1, 6, 10, 15, 16, 21, 25\}$, since $\forall x\in\mathbb{Z}_{20}, \phi(x)=\phi(\underset{x}{\underbrace{1+1+\cdots+1}})=x\phi(1)$ \\
$\Rightarrow\quad\phi(x+y)=(x+y)\phi(1)=\phi(1)x+\phi(1)y=\phi(x)+\phi(y)$ and $\phi(xy)=xy\phi(1)=\phi(1)x\phi(1)y=\phi(x)\phi(y)$ \\
Hence all ring homomorphism from $\mathbb{Z}_{20}$ to $\mathbb{Z}_{30}$ are \\
$\phi(x)=0,\ \phi(x)=x,\ \phi(x)=6x,\ \phi(x)=10x,\ \phi(x)=16x,\ \phi(x)=21x,\ \phi(x)=25x$ \\

4.c. Determine all ring homomorphism from $\mathbb{Z}$ to $\mathbb{Z}$. \\
Since $\phi(1)=\phi(1)\phi(1)$ and $\forall x\in\mathbb{Z}, x^2=x$ iff $x\in\{0, 1\}\quad\Rightarrow\quad\phi(1)\in\{0, 1\}$ \\
If $x\in\mathbb{Z},\ x>0\quad\Rightarrow\quad\phi(x)=x\phi(1)$, $\phi(0)=0$ and \\
$\phi(x)+\phi(-x)=x\phi(1)+x\phi(-1)=x\phi(1-1)=0\quad\Rightarrow\quad\phi(-x)=-x\phi(1)\quad\Rightarrow\quad\forall x\in\mathbb{Z},\ \phi(x)=x\phi(1)$ \\
Let $x, y\in\mathbb{Z},\ \phi(1)\in\{0, 1\}$ \\
$\Rightarrow\quad\phi(x+y)=(x+y)\phi(1)=\phi(1)x+\phi(1)y=\phi(x)+\phi(y)$ and $\phi(xy)=xy\phi(1)=\phi(1)x\phi(1)y=\phi(x)\phi(y)$ \\
Hence all ring homomorphism from $\mathbb{Z}$ to $\mathbb{Z}$ are $\phi(x)=0,\ \phi(x)=x$ \\

4.d. Determine all ring homomorphism from $\mathbb{Z}\oplus\mathbb{Z}$ into $\mathbb{Z}\oplus\mathbb{Z}$. \\
Since $\phi((0, 1))=\phi((0, 1))\phi((0, 1)),\ \phi((1, 0))=\phi((1, 0))\phi((1, 0))$ and \\
$\forall (m, n)\in\mathbb{Z}\oplus\mathbb{Z}, (m, n)^2=(m, n)$ iff $(m, n)\in\{(0, 0), (0, 1), (1, 0), (1, 1)\}$ \\
$\Rightarrow\quad\phi((0, 1)), \phi((1, 0))\in\{(0, 0), (0, 1), (1, 0), (1, 1)\}$ \\
Since $\forall (m, n)\in\mathbb{Z}\oplus\mathbb{Z},\ (m, n)=m(1, 0)+n(0, 1)\quad\Rightarrow\quad\forall (m, n)\in\mathbb{Z}\oplus\mathbb{Z},\ \phi((m, n))=m\phi((1, 0))+n\phi((0, 1))$ \\
Let $(m, n), (x, y)\in\mathbb{Z}\oplus\mathbb{Z}$, if $\phi((0, 1)), \phi((1, 0))\in\{(0, 0), (0, 1), (1, 0), (1, 1)\}$ \\
$\Rightarrow\quad\phi((m, n)+(x, y))=\phi((m+x, n+y))=(m+x)\phi((1, 0))+(n+y)\phi((0, 1))=\phi((m, n))+\phi((x, y))$ \\
Since $\phi((m, n)(x, y))=\phi((mx, ny))=mx\phi((1, 0))+ny\phi((0, 1))$ and \\
$\phi((m, n))\phi((x, y))=mx\phi((1, 0))+(my+nx)\phi((1, 0))\phi((0, 1))+ny\phi((0, 1))$ \\
$\Rightarrow\phi((1, 0))\phi((0, 1))=(0, 0)$ \\
Hence all ring homomorphism from $\mathbb{Z}\oplus\mathbb{Z}$ to $\mathbb{Z}\oplus\mathbb{Z}$ are \\
$\phi((m, n))=m(0, 0)+n(0, 0),\quad\phi((m, n))=m(0, 0)+n(0, 1),\quad\phi((m, n))=m(0, 0)+n(1, 0),$ \\
$\phi((m, n))=m(0, 0)+n(1, 1),\quad\phi((m, n))=m(0, 1)+n(0, 0),\quad\phi((m, n))=m(0, 1)+n(1, 0),$ \\
$\phi((m, n))=m(1, 0)+n(0, 0),\quad\phi((m, n))=m(1, 0)+n(0, 1),\quad\phi((m, n))=m(1, 1)+n(0, 0)$ \\

4.e. Determine all ring homomorphism from $\mathbb{Z}\oplus\mathbb{Z}$ to $\mathbb{Z}$. \\
Since $\phi((0, 1))=\phi((0, 1))\phi((0, 1)),\ \phi((1, 0))=\phi((1, 0))\phi((1, 0))$ and $\forall x\in\mathbb{Z}\oplus\mathbb{Z}, x^2=x$ iff $x\in\{0, 1\}$ \\
$\Rightarrow\quad\phi((0, 1)), \phi((1, 0))\in\{0, 1\}$ \\
Since $\forall (m, n)\in\mathbb{Z}\oplus\mathbb{Z},\ (m, n)=m(1, 0)+n(0, 1)\quad\Rightarrow\quad\forall (m, n)\in\mathbb{Z}\oplus\mathbb{Z},\ \phi((m, n))=m\phi((1, 0))+n\phi((0, 1))$ \\
Let $(m, n), (x, y)\in\mathbb{Z}\oplus\mathbb{Z}$, if $\phi((0, 1)), \phi((1, 0))\in\{0, 1\}$ \\
$\Rightarrow\quad\phi((m, n)+(x, y))=\phi((m+x, n+y))=(m+x)\phi((1, 0))+(n+y)\phi((0, 1))=\phi((m, n))+\phi((x, y))$ \\
Since $\phi((m, n)(x, y))=\phi((mx, ny))=mx\phi((1, 0))+ny\phi((0, 1))$ and \\
$\phi((m, n))\phi((x, y))=mx\phi((1, 0))+(my+nx)\phi((1, 0))\phi((0, 1))+ny\phi((0, 1))\quad\Rightarrow\phi((1, 0))\phi((0, 1))=0$ \\
Hence all ring homomorphism from $\mathbb{Z}\oplus\mathbb{Z}$ to $\mathbb{Z}$ are $\phi((m, n))=0,\ \phi((m, n))=m,\ \phi((m, n))=n$ \\

4.f. Determine all ring homomorphism from $\mathbb{Q}$ to $\mathbb{Q}$. \\
Since $\phi(1)=\phi(1)\phi(1)$ and $\forall x\in\mathbb{Q}\oplus\mathbb{Q}, x^2=x$ iff $x\in\{0, 1\}$ \\
$\Rightarrow\quad\phi(1)\in\{0, 1\}$ \\
Let $\displaystyle a, b\in\mathbb{Z},\ b\ne0,\quad\Rightarrow\quad\phi(1)=\phi(\frac{b}{b})=b\phi(\frac{1}{b})\quad\Rightarrow\quad\frac{1}{b}\phi(1)=\phi(\frac{1}{b})\quad\Rightarrow\quad\phi(\frac{a}{b})=\frac{a}{b}\phi(1)$ \\
Let $x, y\in\mathbb{Q},\ \phi(1)\in\{0, 1\}$ \\
$\Rightarrow\quad\phi(x+y)=(x+y)\phi(1)=\phi(1)x+\phi(1)y=\phi(x)+\phi(y)$ and $\phi(xy)=xy\phi(1)=\phi(1)x\phi(1)y=\phi(x)\phi(y)$ \\
Hence all ring homomorphism from $\mathbb{Q}$ to $\mathbb{Q}$ are $\phi(x)=0,\ \phi(x)=x$ \\

4.g. Determine all ring homomorphism from $\mathbb{R}$ to $\mathbb{R}$. \\
Let $\phi$ be a ring homomorphism from $\mathbb{R}$ to $\mathbb{R}\quad\Rightarrow\quad Ker(\phi)$ is an ideal of $\mathbb{R}\quad\Rightarrow\quad Ker(\phi)\in\{\{0\}, \mathbb{R}\}$ \\
If $Ker(\phi)=\mathbb{R}\quad\Rightarrow\quad\phi(r)=0, \forall r\in\mathbb{R}$ \\
If $Ker(\phi)=\{0\}\quad\Rightarrow\quad\phi$ is an automorphism $\quad\Rightarrow\quad\phi(r)=r, \forall r\in\mathbb{R}$ \\

5. Determine all ring isomorphisms from $\mathbb{Z}_n$ to itself. \\
Since $\phi(1)=\phi(1)\phi(1)$ and $\forall x\in\mathbb{Z}_n, x^2=x$ iff $n|x^2-x$ \\
Let $x, y\in\mathbb{Z}_n$ \\
Since $\forall x\in\mathbb{Z}_n, \phi(x)=x\phi(1)$ \\
$\Rightarrow\quad\phi(x+y)=(x+y)\phi(1)=\phi(1)x+\phi(1)y=\phi(x)+\phi(y)$ and $\phi(xy)=xy\phi(1)=\phi(1)x\phi(1)y=\phi(x)\phi(y)$ \\
Since $\phi$ is an isomorphism$\quad\Rightarrow\quad\phi(\phi(1))=\phi(1)\phi(1)=\phi(1)\quad\Rightarrow\quad\phi(1)=1$ \\
Hence all ring isomorphism from $\mathbb{Z}_n$ to $\mathbb{Z}_n$ are $\phi(x)=x$ \\

\end{CJK}
\end{document}