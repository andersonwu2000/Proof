\documentclass[12pt]{book}
\usepackage[utf8]{inputenc}
\usepackage{color,soul,CJK,epic,tikz,array}
\usepackage{amsmath,amsthm,amssymb}
\usepackage{graphicx}
\usepackage{float}
\usepackage{subfigure}
\setlength{\parindent}{0em}
\author{andersonwu2000}
\pagestyle{empty}
\thispagestyle{empty} 

\newcounter{sect}

\newcounter{block}[sect]
\newenvironment{tblock}[1]
{\refstepcounter{block}\theblock.sim\begin{minipage}[t]{\dimexpr\linewidth}#1\\}
{\end{minipage}\\}

\newenvironment{comm}
{\makebox[12pt][l]{$\bullet$}\begin{minipage}[t]{\dimexpr\linewidth}}
{\end{minipage}}

\begin{document}
\begin{CJK}{UTF8}{bsmi}

\hfill 2021/6/1 吳至堯 U10811023 \\

1. Find all the irreducible polynomials of degree $n\le5$ in $\mathbb{Z}_2[x]$. \\
$Solution$. Let $f(x) = a_n x^n + \cdots + a_1 x + a_0$. \\
For Degrees 1, $x$ and $x+1$ is irreducible over $\mathbb{Z}_2$. \\

For Degrees 2 and 3, if $a_0=0$, then $f(0)=0$. \\
Otherwise, $f(0)=1$ and $f(1)=a_n + \cdots + a_1 + a_0$. \\
Therefore, $x^2+x+1$, $x^3+x+1$ and $x^3+x^2+1$ are irreducible over $\mathbb{Z}_2$. \\

For Degrees 4 and 5, suppose that $f(x)=g(x)h(x)$, where both $g(x)$ and $h(x)$ belong to $\mathbb{Z}_2[x]$ and have degrees less than that of $f(x)$. \\
If $\deg g(x)\ne1\ne\deg h(x)$, then at least one of $g(x)$ and $h(x)$ has degree 2. \\
W.L.O.G., suppose $g(x)$ has degree 2, then $g(x)=x^2+x+1$. \\
Since $(x^2+x+1)(x^2+x+1)=x^4+x^2+1$, $(x^2+x+1)(x^3+x+1)=x^5+x^4+1$ and $(x^2+x+1)(x^3+x^2+1)=x^5+x+1$, hence $x^4+x^2+1$, $x^5+x^4+1$ and $x^5+x+1$ are not irreducible over $\mathbb{Z}_2$. \\
Otherwise, at least one of $g(x)$ and $h(x)$ has degree 1, and $f(x)$ is reducible over $\mathbb{Z}_2$ iff $f(x)$ has a zero in $\mathbb{Z}_2$. \\
If $a_0=0$, then $f(0)=0$. \\
If $a_0=1$, then $f(0)=1$ and $f(1)=a_n + \cdots + a_1 + a_0$. \\
Therefore, $x^4+x+1$, $x^4+x^3+1$, $x^4+x^3+x^2+x+1$, $x^5+x^2+1$, $x^5+x^3+1$, $x^5+x^3+x^2+x+1$, $x^5+x^4+x^2+x+1$, $x^5+x^4+x^3+x+1$ and $x^5+x^4+x^3+x^2+1$ are irreducible over $\mathbb{Z}_2$. \\

Hence all the irreducible polynomials of degree $n\le5$ in $\mathbb{Z}_2[x]$ are $x$, $x+1$, $x^2+x+1$, $x^3+x+1$, $x^3+x^2+1$, $x^4+x+1$, $x^4+x^3+1$, $x^4+x^3+x^2+x+1$, $x^5+x^2+1$, $x^5+x^3+1$, $x^5+x^3+x^2+x+1$, $x^5+x^4+x^2+x+1$, $x^5+x^4+x^3+x+1$ and $x^5+x^4+x^3+x^2+1$. \\

2. Find all the monic irreducible polynomials of degree $n\le3$ in $\mathbb{Z}_3[x]$. \\
$Solution$. For Degrees 1, $x$, $x+1$ and $x+2$ is irreducible over $\mathbb{Z}_3$. \\

For Degrees 2 and 3, if $a_0=0$, then $f(0)=0$. \\
Otherwise : \\
\begin{tabular}{l|l|l|l}
    f(x) & f(1) & f(2) & irreducible \\
    \hline
    $     x^2   +1$   & 2 & 2 & Yes \\
    $     x^2   +2$   & 0 & 0 & No \\
    $     x^2+ x+1$   & 0 & 1 & No \\
    $     x^2+ x+2$   & 1 & 2 & Yes \\
    $     x^2+2x+1$   & 1 & 0 & No \\
    $     x^2+2x+2$   & 2 & 1 & Yes \\
    $    2x^2   +1$   & 0 & 0 & No \\
    $    2x^2   +2$   & 1 & 1 & Yes \\
    $    2x^2+ x+1$   & 1 & 2 & Yes \\
    $    2x^2+ x+2$   & 2 & 0 & No \\
    $    2x^2+2x+1$   & 2 & 1 & Yes \\
    $    2x^2+2x+2$   & 0 & 2 & No \\
    $x^3        +1$   & 2 & 0 & No \\
    $x^3        +2$   & 0 & 1 & No \\
    $x^3     + x+1$   & 0 & 2 & No \\
    $x^3     + x+2$   & 1 & 0 & No \\
    $x^3     +2x+1$   & 1 & 1 & Yes \\
    $x^3     +2x+2$   & 2 & 2 & Yes \\
    $x^3+ x^2   +1$   & 0 & 1 & No \\
    $x^3+ x^2   +2$   & 1 & 2 & Yes \\
    $x^3+ x^2+ x+1$   & 1 & 0 & No \\
    $x^3+ x^2+ x+2$   & 2 & 1 & Yes \\
    $x^3+ x^2+2x+1$   & 2 & 2 & Yes \\
    $x^3+ x^2+2x+2$   & 0 & 0 & No \\
    $x^3+2x^2   +1$   & 1 & 2 & Yes \\
    $x^3+2x^2   +2$   & 2 & 0 & No \\
    $x^3+2x^2+ x+1$   & 2 & 1 & Yes \\
    $x^3+2x^2+ x+2$   & 0 & 2 & No \\
    $x^3+2x^2+2x+1$   & 0 & 0 & No \\
    $x^3+2x^2+2x+2$   & 1 & 1 & Yes 
\end{tabular} \\

Hence all the monic irreducible polynomials of degree $n\le3$ in $\mathbb{Z}_3[x]$ are $x$, $x+1$, $x+2$, $x^2+1$, $x^2+x+2$, $x^2+2x+2$, $2x^2+2$, $2x^2+x+1$, $2x^2+2x+1$, $x^3+2x+1$, $x^3+2x+2$, $x^3+x^2+2$, $x^3+x^2+x+2$, $x^3+x^2+2x+1$, $x^3+2x^2+1$, $x^3+2x^2+x+1$, $x^3+2x^2+2x+2$. \\

\clearpage

3. Find a primitive element $a$ such that $\mathbb{Q}(a)=\mathbb{Q}(\sqrt{2}, \sqrt{3}, \sqrt{5})$. \\
$Solution$. It is clearly that $\mathbb{Q}(\sqrt{2}+\sqrt{3}+\sqrt{5})\subseteq\mathbb{Q}(\sqrt{2}, \sqrt{3}, \sqrt{5})$. \\
Let $\alpha=\sqrt{2}+\sqrt{3}+\sqrt{5}$, then $\alpha^3=26\sqrt{2}+24\sqrt{3}+20\sqrt{5}+6\sqrt{30}$, $12/\alpha=3\sqrt{2}+2\sqrt{3}-\sqrt{30}$ and $72/\alpha^3=18\sqrt{2}+13\sqrt{3}-9\sqrt{5}-5\sqrt{30}\ \in\ \mathbb{Q}(\sqrt{2}+\sqrt{3}+\sqrt{5})$. \\
Try to use a linear combination of $\alpha$, $\alpha^3$, $12/\alpha$ and $72/\alpha^3$ to make $\sqrt{2}$, $\sqrt{3}$ and $\sqrt{5}$. \\
Let 
\[
A = 
\begin{bmatrix}
    1 & 1 & 1 & 0 \\
    26 & 24 & 20 & 6 \\
    3 & 2 & 0 & -30 \\
    18 & 13 & -9 & -5
\end{bmatrix},
\]
then
\[
A^{-1} = 
\begin{bmatrix}
\frac{-3547}{198} & \frac{325}{396} & \frac{19}{99} & \frac{-1}{6} \\
\frac{2947}{132} & \frac{-265}{264} & \frac{-8}{33} & \frac{1}{4} \\
\frac{-1351}{396} & \frac{145}{792} & \frac{5}{99} & \frac{-1}{12} \\
\frac{-10}{33} & \frac{1}{66} & \frac{-1}{33} & 0
\end{bmatrix}
\]
Since $AA^{-1}=I$, that is
\begin{eqnarray*}
\frac{-3547}{198}\alpha + 
\frac{2947}{132}\alpha^3 + 
\frac{-1351}{396}\frac{12}{\alpha} + \frac{-1}{12}\frac{72}{\alpha^3} = 
\sqrt{2}, \\
\frac{325}{396}\alpha + 
\frac{-265}{264}\alpha^3 + 
\frac{145}{792}\frac{12}{\alpha} +
\frac{1}{66}\frac{72}{\alpha^3} = 
\sqrt{3}, \\
\frac{19}{99}\alpha + 
\frac{-8}{33}\alpha^3 + 
\frac{5}{99}\frac{12}{\alpha} +
\frac{-1}{33}\frac{72}{\alpha^3} = 
\sqrt{5}. 
\end{eqnarray*}
Therefore, $\sqrt{2}$, $\sqrt{3}$, $\sqrt{5}\in\mathbb{Q}(\sqrt{2}+\sqrt{3}+\sqrt{5})$ and $\mathbb{Q}(\sqrt{2}+\sqrt{3}+\sqrt{5})\supseteq\mathbb{Q}(\sqrt{2}, \sqrt{3}, \sqrt{5})$. \\
Hence $\sqrt{2}+\sqrt{3}+\sqrt{5}$ is a primitive element of $\mathbb{Q}(\sqrt{2}, \sqrt{3}, \sqrt{5})$. \\

\clearpage

4. Find a primitive element $a$ such that $\mathbb{Q}(a)=\mathbb{Q}(\sqrt{2}, \sqrt[3]{3}, \sqrt[5]{5})$. \\
$Solution$. It is clearly that $\mathbb{Q}(\sqrt{2}\sqrt[3]{3}\sqrt[5]{5})\subseteq\mathbb{Q}(\sqrt{2}, \sqrt[3]{3}, \sqrt[5]{5})$. \\
Since 
\begin{eqnarray*}
(\sqrt{2}\sqrt[3]{3}\sqrt[5]{5})^{15}\ /\ 3888000 = \sqrt{2}, \\
(\sqrt{2}\sqrt[3]{3}\sqrt[5]{5})^{10}\ /\ 21600 = \sqrt[3]{3}, \\
(\sqrt{2}\sqrt[3]{3}\sqrt[5]{5})^{6} \ /\ 360 = \sqrt[5]{5}, 
\end{eqnarray*}
therefore, $\mathbb{Q}(\sqrt{2}\sqrt[3]{3}\sqrt[5]{5})\supseteq\mathbb{Q}(\sqrt{2}, \sqrt[3]{3}, \sqrt[5]{5})$. \\
Hence $\sqrt{2}\sqrt[3]{3}\sqrt[5]{5}$ is a primitive element of $\mathbb{Q}(\sqrt{2}, \sqrt[3]{3}, \sqrt[5]{5})$. \\

5. Give a least three suggestions for this online algebra course. \\

5.a. 因為我不一定能馬上回想起以前的內容,所以我覺得老師講的證明,例如定理引用和某些關鍵步驟的部分,速度可能有些偏快;我想,就證明中較抽象的部分,可以拆解得慢一點點,如有引用之前定理的部分,可以稍加提點,描述或展示出該定理的內容;若因此而有進度上的壓力,可以考慮以作業來代替上課的複習,將實體課用於複習的時間轉用於進度。 \\

5.b. 老師投影片上的定理編號好像跟課本的不太一樣,在證明過程中若有引用到過去的部分,要回去找課本可能會造成一些麻煩,可以考慮在投影片標註課本上對應的定理編號。 \\

5.c. 若有需要的話,可以在網路上看到別人的教學過程,借鑑後應該也能對教學的方法有更多認識 \\
雖然下面的影片我大多沒看過,更難以保證他們教得更好,不過這些是我已知的資源,可以提供參考 : \\
https://www.youtube.com/playlist?list=PL6839449936471E0C \\
https://www.youtube.com/playlist?app=desktop\&list=PLF379B0552AD17780 \\
https://youtube.com/playlist?list=PLVJXJebpO4Pi0DFfCrGHGrgh26xszeLch \\
https://youtube.com/playlist?list=PLVJXJebpO4PgHMNwSEmHYTScj7qqXp-QG \\
https://ocw.nthu.edu.tw/ocw/index.php?page=course\&cid=33\& \\
http://www.math.ncu.edu.tw/~cchsiao/OCW/05.html 

\end{CJK}
\end{document}