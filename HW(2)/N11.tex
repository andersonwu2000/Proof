\documentclass[12pt]{book}
\usepackage[utf8]{inputenc}
\usepackage{color,soul,CJK,epic,tikz,array}
\usepackage{amsmath,amsthm,amssymb}
\setlength{\parindent}{0em}
\linespread{1.3}
\author{andersonwu2000}
\usepackage[margin=2cm]{geometry}
\pagestyle{empty}
\thispagestyle{empty} 

\newcounter{sect}

\newcounter{block}[sect]
\newenvironment{tblock}[1]
{\refstepcounter{block}\theblock.sim\begin{minipage}[t]{\dimexpr\linewidth}#1\\}
{\end{minipage}\\}

\newenvironment{comm}
{\makebox[12pt][l]{$\bullet$}\begin{minipage}[t]{\dimexpr\linewidth}}
{\end{minipage}}

\begin{document}
\begin{CJK}{UTF8}{bsmi}

\hfill 吳至堯 U10811023

1. $\displaystyle \sqrt{23^2*1}=\sqrt{528}=22+\frac{1}{\displaystyle1+\frac{1}{44+\frac{1}{1+\frac{1}{44+\frac{1}{...}}}}}$

$50=2\cdot5^2$ \\
ord$_52=$ ord$_53=4\quad\Rightarrow\quad 2,\ 3$ are the primitive root modulo 5. \\
ord$_{25}2=$ ord$_{25}3=20\quad\Rightarrow\quad 2,\ 3$ are the primitive root modulo 25. \\
$2+25=27,\ 3$ are the primitive root modulo 50. \\

$U(50)=\{1,3,7,9,11,13,17,19,21,23,27,29,31,33,37,39,41,43,47,49\}$ \\

\begin{tabular}{c|c|c|c|c|c|c|c|c|c|c|c|c|c|c|c|c|c|c|c|c}
    power & 1 & 2 & 3 & 4 & 5 & 6 & 7 & 8 & 9 & 10 & 11 & 12 & 13 & 14 & 15 & 16 & 17 & 18 & 19 & 20 \\\hline
    $\left \langle 3 \right \rangle$ & 3 & 9 & 27 & 31 & 43 & 29 & 37 & 11 & 33 & 49 & 47 & 41 & 23 & 19 & 7 & 21 & 13 & 39 & 17 & 1
\end{tabular} \\

2. \begin{minipage}[t]{\dimexpr\linewidth-2em}
To solve $23x^{17}\equiv43\mod50$ \\
$\log_323+17\log_3x\equiv\log_343\mod\phi(50)=20$ \\
$17\log_3x\equiv\log_343-\log_323\equiv5-13\equiv-8\mod\phi20$ \\
$\log_3x\equiv16\mod20$ \\
$x\equiv21\mod50$
\end{minipage} \\

3. \begin{minipage}[t]{\dimexpr\linewidth-2em}
To solve $23x^{16}\equiv43\mod50$ \\
$\log_323+16\log_3x\equiv\log_343\mod\phi(50)=20$ \\
$16\log_3x\equiv\log_343-\log_323\equiv5-13\equiv-8\mod20$ \\
$\log_3x\equiv2, 7, 12, 17\mod20$ \\
$x\equiv9, 37, 41, 13\mod50$
\end{minipage} \\

4. \begin{minipage}[t]{\dimexpr\linewidth-2em}
To solve $23x^{15}\equiv43\mod50$ \\
$\log_323+15\log_3x\equiv\log_343\mod\phi(50)=20$ \\
$15\log_3x\equiv\log_343-\log_323\equiv5-13\equiv-8\mod20$ \\
$\exists\ r\in\mathbb{N}$ s.t. $15r+8\equiv0\mod20\quad\rightarrow\leftarrow$ \\
There has no any solutions.
\end{minipage}

\end{CJK}
\end{document}