\documentclass[12pt]{book}
\usepackage[utf8]{inputenc}
\usepackage{color,soul,CJK,epic,tikz,array}
\usepackage{amsmath,amsthm,amssymb}
\setlength{\parindent}{0em}
\linespread{1.3}
\author{andersonwu2000}
\usepackage[margin=1cm]{geometry}
\pagestyle{empty}
\thispagestyle{empty} 

\newcounter{sect}

\newcounter{block}[sect]
\newenvironment{tblock}[1]
{\refstepcounter{block}\theblock.sim\begin{minipage}[t]{\dimexpr\linewidth}#1\\}
{\end{minipage}\\}

\newenvironment{comm}
{\makebox[12pt][l]{$\bullet$}\begin{minipage}[t]{\dimexpr\linewidth}}
{\end{minipage}}

\begin{document}
\begin{CJK}{UTF8}{bsmi}

\hfill 章節 0 吳至堯 U10811023

% 32 34

32. \begin{minipage}[t]{\dimexpr\linewidth-2em}
What is the largest bet that cannot be made with chips worth \$7.00 and \$9.00? Verify that your answer is correct with both forms of induction. \\
1. \begin{minipage}[t]{\dimexpr\linewidth}
    Let $P(n)=\exists\ a, b\in\mathbb{N}_0$ s.t. $n=7a+9b$ \\
    Since $62=7\times5+9\times3\quad\Rightarrow\quad P(62)$ is true \\
    Suppose that $P(n)$ is true for all $n\ge62\quad\Rightarrow\quad\exists\ r, s\in\mathbb{N}_0$ s.t. $n=7r+9s$ \\
    Since $7\times4-9\times3=9\times4-7\times5=1\quad\Rightarrow\quad n+1=7(r+4)+9(s-3)=7(r-5)+9(s+4)$ \\
    $\Rightarrow\quad P(n+1)$ is true for all $r\ge5$ or $s\ge3$ \\
    If $r<5$ and $s<3\quad\Rightarrow\quad 7\times r+9\times s<7\times5+9\times3=62$ \\
    Since $n\ge62\quad\Rightarrow\quad r\ge5$ or $s\ge3\quad\Rightarrow\quad P(n+1)$ is true \\
    Hence $P(n)$ is true for all $n\ge62$
\end{minipage} \\[5pt]
2. \begin{minipage}[t]{\dimexpr\linewidth}
    Let $S=\{n\in\mathbb{N}\mid\exists\ a, b\in\mathbb{N}_0$ s.t. $n=7a+9b\}\quad\Rightarrow\quad 62\in S$\\
    Suppose that $k\in S$ for all $62\le k< n\quad\Rightarrow\quad n-1\in S\quad\Rightarrow\quad\exists\ r,s\in\mathbb{N}_0$ s.t. $n-1=7r+9s$ \\
    Since $n-1\ge62\quad\Rightarrow\quad r\ge5$ or $s\ge3\quad\Rightarrow\quad n=7r+9s+1=7(r+4)+9(s-3)=7(r-5)+9(s+4)$ \\
    Since $r+4, s-3\in\mathbb{N}_0$ or $r-5, s+4\in\mathbb{N}_0\quad\Rightarrow\quad n\in S$
\end{minipage} \\
Since \\
\begin{tabular}[t]{ccccc}
    $61=7\times1+9\times6$ & $60=7\times6+9\times2$ & $59=7\times2+9\times5$ & $58=7\times7+9\times1$ & $57=7\times3+9\times4$ \\ 
    $56=7\times8+9\times0$ & $55=7\times4+9\times3$ & $54=7\times0+9\times6$ & $53=7\times5+9\times2$ &
    $52=7\times1+9\times5$ \\
    $51=7\times6+9\times1$ & $50=7\times2+9\times4$ &
    $49=7\times7+9\times0$ & $48=7\times3+9\times3$ &
    and \\
    $47-9\times0\not|\ 7$ & $47-9\times1\not|\ 7$ &
    $47-9\times2\not|\ 7$ & $47-9\times3\not|\ 7$ &
    $47-9\times4\not|\ 7$ \\
    $47-9\times5\not|\ 7$ & $9\times6>47$
\end{tabular} \\
Hence $P(47)$ is False and $P(n)$ is True for all $n>47, n\in\mathbb{N}$
\end{minipage}\\[5pt]

32. \begin{minipage}[t]{\dimexpr\linewidth-2em}
What is the largest bet that cannot be made with chips worth \$7.00 and \$9.00? Verify that your answer is correct with both forms of induction. \\
Let $S=\{n\in\mathbb{N}\mid\exists\ a, b\in\mathbb{N}_0$ s.t. $n=7a+9b\}\quad\Rightarrow\quad 48=7\cdot3+9\cdot3\in S$\\
Suppose that $n\in S$ and $n\ge48\quad\Rightarrow\quad\exists\ r,s\in\mathbb{N}_0$ s.t. $n=7r+9s$ \\
Since $7\cdot4-9\cdot3=1\quad\Rightarrow\quad n+1=7r+9s+1=7(r+4)+9(s-3)\quad\Rightarrow\quad n+1\in S$ for $s\ge3$ \\
For $0\le s\le 2$, since $7r+18\ge n=7r+9s\ge48\quad\Rightarrow\quad 7r\ge30\quad\Rightarrow\quad r\ge5$ \\
If $0\le s\le 2$, since $r\ge5$ and $7\cdot(-5)+9\cdot4=1\quad\Rightarrow\quad n+1=7(r-5)+9(s+4)\in S$
\end{minipage}\\

34. \begin{minipage}[t]{\dimexpr\linewidth-2em}
The Fibonacci numbers are 1,1,2,3,.... In general, the Fibonacci numbers are defined by $f_1=1$, $f_2=1$, and for $n\ge3$, $f_n=f_{n-1}+f_{n-2}$. Prove that the $n$th Fibonacci number $f_n$ satisfies $f_n<2^n$. \\
Let $S=\{n\in\mathbb{N}\mid f_n<2^n\}$ \\
Since $f_1=1<2^1=2\quad\Rightarrow\quad 1\in S$ \\
Since $f_2=1<2^2=4\quad\Rightarrow\quad 2\in S$ \\
Suppose that $3\le k<n$ and $k\in S\quad\Rightarrow\quad n-1, n-2\in S$ and $f_{n-1}<2^{n-1}, f_{n-2}<2^{n-2}$ \\
$\Rightarrow\quad f_n=f_{n-1}+f_{n-2}<2^{n-1}+2^{n-2}<2^n\quad\Rightarrow\quad f_n\in S$ \\
Hence $f_n<2^n$ for all $n\in\mathbb{N}$
\end{minipage}\\

\end{CJK}
\end{document}