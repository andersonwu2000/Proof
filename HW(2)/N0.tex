\documentclass[12pt]{book}
\usepackage[utf8]{inputenc}
\usepackage{color,soul,CJK,epic,tikz,array}
\usepackage{amsmath,amsthm,amssymb}
\setlength{\parindent}{0em}
\linespread{1.3}
\author{andersonwu2000}
\usepackage[margin=2cm]{geometry}
\pagestyle{empty}
\thispagestyle{empty} 

\newcounter{sect}

\newcounter{block}[sect]
\newenvironment{tblock}[1]
{\refstepcounter{block}\theblock.sim\begin{minipage}[t]{\dimexpr\linewidth}#1\\}
{\end{minipage}\\}

\newenvironment{comm}
{\makebox[12pt][l]{$\bullet$}\begin{minipage}[t]{\dimexpr\linewidth}}
{\end{minipage}}

\begin{document}
\begin{CJK}{UTF8}{bsmi}

\hfill 吳至堯 U10811023

作業 : 利用 1-9 、 9-1 (順序不變) 及加減運算符號,排出答案等於學號後兩碼的算式\\
$1+2+3+4-5-6+7+8+9=23$ \\
$1-23+45-6+7+8-9=23$ \\
$1-23+45+6-7-8+9=23$ \\
$1+2-3-4+5-67+89=23$ \\
$1-2+3+4-5-67+89=23$ \\
$9+8+7-6-5+4+3+2+1=23$ \\
$98-76+5-4-3+2+1=23$ \\
$98-76-5+4+3-2+1=23$ \\
$9-8-7+6+54-32+1=23$ \\
$-9+8+7-6+54-32+1=23$ \\

討論 : 訂一個可以將自然數分為兩類的規則 \\
對於自然數 $n$,存在兩個質數 $p_1, p_2$ 使得 $p_1^2+p_2^2=n^2$ \\
我想把滿足這個條件的自然數分為一類,不滿足的分為一類 \\

其他 : \\
用 1,2,3,4 造出 1 到 100 所有整數的部分 \\
我印象中在 ppt 上,16 等式右方的指數有格式問題 \\
其他部分,指數有顯示上的問題 \\\\
\begin{minipage}{18em}
$18 = 4^2+(3-1)$ \\
$19 = 4^2+(3\times1)$ \\
$20 = 4^2+3+1$ \\
$22 = (4+1)^2-3$ \\
$23 = 3^2+14$ \\
$27 = 3^2\times(4-1)$ \\
$32 = 4^{(3-1)}\times2$ \\
$49 = ((4\times1)+3)^2$ \\
$50 = 41+3^2$ \\
$51 = (12\times4)+3$, or $(4^2+1)\times3$ \\
$52 = 4^3-12$ \\
$60 = 3^4-21$ \\
$61 = 4^3-(1+2)$
\end{minipage}
\begin{minipage}{\dimexpr\linewidth}
$62 = 43-(1\times2)$ \\
$63 = 4^3-(2-1)$ \\
$64 = (2-1)\times4^3$ \\
$65 = (2-1)+4^3$ \\
$66 = (2\times1)+4^3$ \\
$70 = 4^3+(1+2)!$ \\
$75 = (4+1)^2\times3$ \\
$79 = 3^4-(2\times1)$ \\
$80 = 3^4-(2-1)$ \\
$82 = 3^4+(2-1)$ \\
$83 = 3^4+(2\times 1)$ \\
$84 = 3^4+2+1$ \\
$95 = 3!\times2^4-1$
\end{minipage}

\end{CJK}
\end{document}