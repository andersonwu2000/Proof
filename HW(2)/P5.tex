\documentclass[12pt]{book}
\usepackage[utf8]{inputenc}
\usepackage{color,soul,CJK,epic,tikz,array}
\usepackage{amsmath,amsthm,amssymb}
\usepackage{graphicx}
\usepackage{float}
\usepackage{subfigure}
\setlength{\parindent}{0em}
\linespread{1.3}
\author{andersonwu2000}
\usepackage[margin=1cm]{geometry}
\pagestyle{empty}
\thispagestyle{empty} 

\newcounter{sect}

\newcounter{block}[sect]
\newenvironment{tblock}[1]
{\refstepcounter{block}\theblock.sim\begin{minipage}[t]{\dimexpr\linewidth}#1\\}
{\end{minipage}\\}

\newenvironment{comm}
{\makebox[12pt][l]{$\bullet$}\begin{minipage}[t]{\dimexpr\linewidth}}
{\end{minipage}}

\begin{document}
\begin{CJK}{UTF8}{bsmi}

\hfill 章節 2 吳至堯 U10811023

5.7-4. Let $X$ equal the number out of $n = 48$ mature
aster seeds that will germinate when $p = 0.75$ is the probability that a particular seed germinates. Approximate
$P(35 \le X \le 40)$. \\

Let $X$ be $b(48, 0.75)$ \\
$\Rightarrow \mu=48\times0.75=36,\quad\sigma^2=48\times0.75\times(1-0.75)=9$ \\
$\displaystyle\Rightarrow 
P(35 \le X \le 40)
=P(34.5 \le X \le 40.5)
=P(\frac{34.5-36}{\sqrt{9}}\le \frac{X-36}{\sqrt{9}} \le \frac{40.5-36}{\sqrt{9}})
\approx\Phi(1.5)-\Phi(-0.5)
\approx0.9332 - 1 + 0.6915
=0.6247$ \\

5.7-6. In adults, the pneumococcus bacterium causes $70\%$ of pneumonia cases. In a random sample of $n = 84$ adults who have pneumonia, let $X$ equal the number whose pneumonia was caused by the pneumococcus bacterium. Use the normal distribution to find $P(X \le 52)$, approximately. \\

Let $X$ be $b(84, 0.7)$ \\
$\Rightarrow \mu=84\times0.7=58.8,\quad\sigma^2=84\times0.7\times(1-0.7)=17.64$ \\
$\displaystyle\Rightarrow
P(X\le52)
=P(X\le52.5)
=P(\frac{X-58.8}{\sqrt{17.64}}\le\frac{52.5-58.8}{\sqrt{17.64}})
\approx\phi(-1.5)
\approx1-0.9332
=0.0668$ \\

5.7-8. A candy maker produces mints that have a label weight of 20.4 grams. Assume that the distribution of the weights of these mints is $N(21.37, 0.16)$. \\

(a) Let $X$ denote the weight of a single mint selected at random from the production line. Find $P(X < 20.857)$. \\
$\displaystyle P(X < 20.857)
=P(\frac{X-21.37}{\sqrt{0.16}} \le \frac{20.857-21.37}{\sqrt{0.16}})
\approx\Phi(-1.28)
\approx1-0.8997
=0.1003$ \\

(b) During a particular shift, 100 mints are selected at random and weighed. Let Y equal the number of these mints that weigh less than 20.857 grams. Approximate $P(Y \le 5)$. \\
Let $Y$ be $b(100, 0.1)$ \\
$\Rightarrow \mu=100\times0.1=10,\quad\sigma^2=100\times0.1\times(1-0.1)=9$ \\
$\displaystyle\Rightarrow
P(Y\le5)
=P(Y\le5.5)
=P(\frac{Y-10}{\sqrt{9}}\le\frac{5.5-10}{\sqrt{9}})
\approx\phi(-1.5)
\approx1-0.9332
=0.0668$ \\

(c) Let $\Bar{X}$ equal the sample mean of the 100 mints selected and weighed on a particular shift. Find $P(21.31 \le \Bar{X} \le 21.39)$. \\
$\displaystyle
P(21.31 \le \Bar{X} \le 21.39)
=P(\frac{21.31-21.37}{\sqrt{0.16}/\sqrt{100}} \le \frac{\Bar{X}}{\sqrt{0.16}/\sqrt{100}} \le \frac{21.39-21.37}{\sqrt{0.16}/\sqrt{100}})
\approx\Phi(0.5)-\Phi(-1.5)
\approx0.6915-1+0.9332
=0.6247$ \\

5.7-12. If $X$ is $b(100, 0.1)$, find the approximate value of $P(12 \le X \le 14)$, using \\
(a) The normal approximation. \\
$\mu=100*0.1=10,\quad\sigma^2=100\times0.1\times(1-0.1)=9$ \\
$\displaystyle
P(12 \le X \le 14)
=P(11.5\le X\le 14.5)
=P(\frac{11.5-10}{\sqrt{9}}\le\frac{X-10}{\sqrt{9}}\le\frac{14.5-10}{\sqrt{9}})
\approx\Phi(1.5)-\Phi(0.5)
\approx0.9332-0.6915
=0.2417$ \\

(b) The Poisson approximation. \\
$\displaystyle
P(12 \le X \le 14)
=P(X=12) + P(X=13) + P(X=14)
=\frac{10^{12}e^{-10}}{12!}+\frac{10^{13}e^{-10}}{13!}+\frac{10^{14}e^{-10}}{14!}
\approx0.0948+0.0729+0.0521
=0.2198$ \\

(c) The binomial. \\
$\displaystyle
P(12 \le X \le 14)
=P(X=12) + P(X=13) + P(X=14) \\
={\begin{pmatrix}100\\12\end{pmatrix}}0.1^{12} (1-0.1)^{100-12}+{\begin{pmatrix}100\\13\end{pmatrix}}0.1^{13} (1-0.1)^{100-13}+{\begin{pmatrix}100\\14\end{pmatrix}}0.1^{14} (1-0.1)^{100-14} \\[5pt]
\approx0.0988+0.0743+0.0513
=0.2244$ \\

\end{CJK}
\end{document}