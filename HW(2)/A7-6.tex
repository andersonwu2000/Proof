\documentclass[12pt]{book}
\usepackage[utf8]{inputenc}
\usepackage{color,soul,CJK,epic,tikz,array}
\usepackage{amsmath,amsthm,amssymb}
\usepackage{graphicx}
\usepackage{float}
\usepackage{subfigure}
\setlength{\parindent}{0em}
\author{andersonwu2000}
\pagestyle{empty}
\thispagestyle{empty} 

\newcounter{sect}

\newcounter{block}[sect]
\newenvironment{tblock}[1]
{\refstepcounter{block}\theblock.sim\begin{minipage}[t]{\dimexpr\linewidth}#1\\}
{\end{minipage}\\}

\newenvironment{comm}
{\makebox[12pt][l]{$\bullet$}\begin{minipage}[t]{\dimexpr\linewidth}}
{\end{minipage}}

\begin{document}
\begin{CJK}{UTF8}{bsmi}

\hfill 2021/6/15 吳至堯 U10811023 \\

7.30. If $f(x) = \sum_{k=0}^\infty a_k(x-x_0)^k$ is a power series with positive radius of convergence $R$, then $f'(x) = \sum_{k=1}^\infty k a_k(x-x_0)^{k-1}$ for $x\in(x_0-R, x_0+R)$. \\
$Proof$. Let $I=(x_0-R, x_0+R)$, $x\in I$ and $g(x) = \sum_{k=0}^\infty k a_k (x-x_0)^k$, then $\exists\ a, b\in\mathbb{R}$ such that $x\in[a, b]$ with $[a, b]\subset I$. \\
Since $R=1/\limsup_{k\rightarrow\infty}|a_k|^{1/k}$ is the radius of convergence of $f$, by Lemma 7.29, $R$ is the radius of convergence of $g$. \\
By Theorem 7.21, $g$ converges absolutely on $I$ and uniformly on $[a, b]$. \\
Let $S(x) = \sum_{k=1}^\infty k a_k (x-x_0)^{k-1}$. \\
Since $S(x) = g(x)/(x-x_0)$ if $x\ne x_0$, then $S(x)$ converges absolutely. \\
Since $S(x_0) = k a_k$, then $S(x_0)$ converges absolutely. \\
Therefore, $S$ converges absolutely on $I$. \\
By Theorem 7.21, $S$ converges uniformly on $[a, b]$. \\
By Theorem 7.14, f is differentiable on $[a, b]$ and
\[
    f'(x)
    = \sum_{k=0}^\infty (a_k(x-x_0)^k)'
    = \sum_{k=1}^\infty k a_k (x-x_0)^{k-1}
    = S(x).
\]
Hence $f'(x) = \sum_{k=1}^\infty k a_k(x-x_0)^{k-1}$ for $x\in(x_0-R, x_0+R)$. \\

7.32. Let $f(x) = \sum_{k=0}^\infty a_k(x-x_0)^k$ be a power series and let $a, b\in\mathbb{R}$ with $a<b$. \\
i. If $f(x)$ converges on $[a, b]$, then $f$ is integrable on $[a, b]$ and
\[
    \int_a^b f(x) dx 
    = \sum_{k=0}^\infty a_k \int_a^b (x-x_0)^k dx.
\]
ii. If $f(x)$ converges on $[a, b)$ and if $\sum_{k=0}^\infty a_k(b-x_0)^{k+1}/(k+1)$ converges, then $f$ is improperly integrable on $[a, b)$ and
\[
    \int_a^b f(x) dx 
    = \sum_{k=0}^\infty a_k \int_a^b (x-x_0)^k dx.
\]
$Proof$. By Theorem 7.27, $f$ converges uniformly on $[a, b]$. \\
Since $a_k(x-x_0)^k$ integrable on $[a, b]$ for all $k\in\mathbb{N}$, by Theorem 7.14.ii, $f$ is integrable on $[a, b]$ and
\[
    \int_a^b f(x) dx 
    = \int_a^b \sum_{k=0}^\infty a_k(x-x_0)^k dx
    = \sum_{k=0}^\infty a_k \int_a^b (x-x_0)^k dx.
\]
Let $t\in(a, b)$ and $A = \sum_{k=0}^\infty a_k(a-x_0)^{k+1}/(k+1)$, then $f$ converges on $[a, t]$ and
\[
    \int_a^t f(x) dx 
    = \sum_{k=0}^\infty a_k \int_a^t (x-x_0)^k dx
    = \sum_{k=0}^\infty \frac{a_k}{k+1}(t-x_0)^{k+1}-A.
\]
Since $\sum_{k=0}^\infty a_k(b-x_0)^{k+1}/(k+1)$ converges, then
\begin{eqnarray*}
\int_a^b f(x) dx 
    & = & \lim_{t\rightarrow b-} \int_a^t f(x) dx \\
    & = & \lim_{t\rightarrow b-} \sum_{k=0}^\infty \frac{a_k}{k+1}(t-x_0)^{k+1}-A \\
    & = & \sum_{k=0}^\infty \frac{a_k}{k+1}(b-x_0)^{k+1}-A
    = \sum_{k=0}^\infty a_k \int_a^b (x-x_0)^k dx.
\end{eqnarray*}
Hence the proof is complete. \\

7.33. If $f(x) = \sum_{k=0}^\infty a_k x^k$ and $g(x) = \sum_{k=0}^\infty b_k x^k$ converge on $(-r, r)$ and
\[
    c_k = \sum_{j=0}^k a_j b_{k-j},\quad k = 0, 1, \cdots,
\]
then $\sum_{k=0}^\infty c_k x^k$ converges on $(-r, r)$ and converges to $f(x)g(x)$. \\
$Proof$. Let $x\in(-r, r)$, $\varepsilon>0$
\[
    f_n(x) = \sum_{k=0}^n a_k x^k,\quad
    g_n(x) = \sum_{k=0}^n b_k x^k,\quad\text{and}\quad
    h_n(x) = \sum_{k=0}^n c_k x^k
\]
for all $n\in\mathbb{N}$, then
\begin{eqnarray*}
h_n(x) 
    & = & \sum_{k=0}^n \sum_{j=0}^k a_j b_{k-j} x^k
    = \sum_{k=0}^n a_j x^j \sum_{j=0}^k b_{k-j} x^{k-j} \\
    & = & \sum_{k=0}^n a_j x^j g_{n-j}(x) 
    = g(x)f_n(x) + \sum_{j=0}^n a_jx^j (g_{n-j}(x)-g(x)).
\end{eqnarray*}
 Since $f(x)$ converge absolutely, then $\exists\ M_1>0$ such that $\sum_{k=0}^\infty |a_k x^k|<M_1$. \\
 Since $g(x)$ and $g_n(x)$ converges as $n\rightarrow\infty$, then $\exists\ M_2>0$, $N_1\in\mathbb{N}$ such that $|g_{n-j}(x)-g(x)|\le M_2$ for all $n>j>N_1$. \\
 Take $M=\max{M_1, M_2}$, then $\exists N_2, N_3\in\mathbb{N}$ such that
\[
    l\ge N_2
    \quad\text{imply}\quad
    |g_l(x)-g(x)| < \frac{\varepsilon}{2M}
    \quad\text{and}\quad
    \sum_{j=N_3+1}^\infty |a_j x^j|<\frac{\varepsilon}{2M}.
\]
Let $N=\max\{N_1, N_2, N_3\}$ and $n\ge N$, then
\begin{eqnarray*}
\left| \sum_{j=0}^n a_j x^j (g_{n-j}(x)-g(x)) \right|
    & = & \left| \sum_{j=0}^N a_j x^j (g_{n-j}(x)-g(x)) +
    \sum_{j=N+1}^n a_j x^j (g_{n-j}(x)-g(x))
    \right| \\
    & < & \frac{\varepsilon}{2M}\sum_{j=0}^N |a_j x^j| +
    M\sum_{j=N+1}^n |a_j x^j| \\
    & < & \frac{\varepsilon}{2} + \frac{\varepsilon}{2} = \varepsilon.
\end{eqnarray*}
Hence the proof is complete. \\

7.36. Find a closed form of the power series
\[
    f(x) = \sum_{k=1}^\infty k x^k.
\]
$Solution$. Since $R = 1/\limsup_{k\rightarrow\infty}|k|^{1/k} = 1$ and $f(1)$, $f(-1)$ diverge, then $f$ converge on $(-1, 1)$. \\
Let $x\in(-1, 1)$. \\
Since $f(x)/x = \sum_{k=1}^\infty k x^{k-1}$ converge with the convention that $f(x)/0 = 1$, by Theorem 7.32, 
\[
    \int_0^x \frac{f(t)}{t} dt
    = \sum_{k=1}^\infty k \int_0^x t^{k-1} dt
    = \sum_{k=1}^\infty x^k
    = \frac{x}{1-x}.
\]
By the Fundamental Theorem of Calculus, 
\[
    \frac{f(x)}{x}
    = \left( \frac{x}{1-x} \right)'
    = \frac{1}{(1-x)^2}.
\]
Hence $f(x) = x/(1-x)^2$ for $x\in(-1, 1)$. \\

7.37.Find a closed form of the power series
\[
    g(x) = \sum_{k=0}^\infty \frac{x^k}{k+1}.
\]
$Solution$. Since $R = 1/\limsup_{k\rightarrow\infty}|1/(k+1)|^{1/k} = 1$, $f(1)$ diverge and $f(-1)$ converge, then $f$ converge on $[-1, 1)$. \\
Since $R\ne0$, by Theorem 7.30, 
\[
    (x g(x))'
    = \sum_{k=0}^\infty \left( \frac{x^{k+1}}{k+1} \right)'
    = \sum_{k=0}^\infty x^k
    = \frac{1}{1-x}
\]
for $x\in(-1, 1)$. \\
By the Fundamental Theorem of Calculus, 
\[
    x g(x)
    = \int_0^x \frac{dt}{1-t}
    = -\ln(1-x)
\]
for $x\in(-1, 1)$. \\
Since $g(-1)$ converge, by Theorem 7.27, $g(-1)$ is continuous. \\
Hence 
\[
    g(x)
    = \left\{\begin{matrix}
        -\ln(1-x)/x & x\in[-1, 1)\setminus\{0\} \\
        1 & x=0.
    \end{matrix}\right.
\]

\end{CJK}
\end{document}