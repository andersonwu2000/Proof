\documentclass[12pt]{book}
\usepackage[utf8]{inputenc}
\usepackage{color,soul,CJK,epic,tikz,array}
\usepackage{amsmath,amsthm,amssymb}
\usepackage{graphicx}
\usepackage{float}
\usepackage{subfigure}
\setlength{\parindent}{0em}
\author{andersonwu2000}
\pagestyle{empty}
\thispagestyle{empty} 

\newcounter{sect}

\newcounter{block}[sect]
\newenvironment{tblock}[1]
{\refstepcounter{block}\theblock.sim\begin{minipage}[t]{\dimexpr\linewidth}#1\\}
{\end{minipage}\\}

\newenvironment{comm}
{\makebox[12pt][l]{$\bullet$}\begin{minipage}[t]{\dimexpr\linewidth}}
{\end{minipage}}

\begin{document}
\begin{CJK}{UTF8}{bsmi}

\hfill 2021/6/17 吳至堯 U10811023 \\

7.39. Let $c, d$ be extended real numbers with $c<d$, let $x_0\in(c, d)$, and suppose that $f:(c, d)\rightarrow\mathbb{R}$. If $f(x) = \sum_{k=0}^\infty a_k (x-x_0)^k$ for $x\in(c, d)$, then $f\in C^\infty (c, d)$ and
\[
    a_k = \frac{f^{(k)}(x_0)}{k!},\quad k=0, 1, \cdots.
\]
$Proof$. Since $f(x_0) = a_0(x_0-x_0)^0$, then $a_0 = f(x_0)$. \\
Let $k\in\mathbb{N}$ and $R$ be the radius of convergence of $\sum_{k=0}^\infty a_k (x-x_0)^k$. \\
Since $\sum_{k=0}^\infty a_k (x-x_0)^k\in\mathbb{R}$ for $x\in(c, d)$, then $(c, d)\subseteq(x_0-R, x_0+R)$. \\
By Corollary 7.31, $f\in C^\infty(c, d)$ and
\[
    f^{(k)}(x)
    = \sum_{n=k}^\infty \frac{n!}{(n-k)!} a_n (x-x_0)^{n-k}
\]
for $x\in(c, d)$, then
\[
    f^{(k)}(x_0)
    = \sum_{n=k}^\infty \frac{n!}{(n-k)!} a_n (x_0-x_0)^{n-k}
    = \frac{k!}{(k-k)!} a_k (x_0-x_0)^0
    = k! a_k.
\]
Hence $a_k = f^{(k)}(x_0) / k!$ for $k\in\mathbb{N}_0$. \\

7.41. The function
\[
    f(x) = \left\{\begin{matrix}
        e^{-1/x^2} & x\ne0 \\
        0 & x=0
    \end{matrix}\right.
\]
belongs to $C^\infty(-\infty, \infty)$ but is not analytic on any interval which contains $x=0$. \\
$Proof$. By Exercise 4.4.7, $f\in C^\infty(-\infty, \infty)$ and $f^{(k)}(0)=0$ for $k\in\mathbb{N}$. \\
Since the Maclaurin expansion of $f$ is
\[
    \sum_{k=0}^\infty \frac{f^{(k)}(0)}{k!} x^k
    = \sum_{k=0}^\infty 0 x^k
    = 0
\]
for all $x\in\mathbb{R}$ and $f(x)\ne0$ for all $x\in\mathbb{R}/\{0\}$, then $f$ is not analytic on any interval which contains $x=0$. \\
Hence the proof is complete. \\

7.43. Let $f\in C^\infty(a, b)$. If there is an $M>0$ such that $|f^{(n)}(x)| \le M^n$ for all $x\in(a, b)$ and $n\in\mathbb{N}$, then $f$ is analytic on $(a, b)$. In fact, for each $x_0\in(a, b)$,
\[
    f(x)
    = \sum_{k=0}^\infty \frac{f^{(k)}(x_0)}{k!} (x-x_0)^k
\]
holds for all $x\in(a, b)$. \\
$Proof$. Let $x_0\in(a, b)$ and $C=\max\{M|a-x_0|, M|b-x_0|\}$. \\
By Taylor's Formula, for all $x\in(a, b)$ and $n\in\mathbb{N}$, $\exists\ x_1$ between $x_0$ and $x$ such that
\[
    |R^{f, x_0}_n(x)|
    = \frac{|f^{(n)}(x_1)|}{n!} |x-x_0|^n
    \le \frac{M^n}{n!} |x-x_0|^n
    \le \frac{C^n}{n!}.
\]
Since $\lim_{n\rightarrow\infty} C^n/n! = 0$ and $C^n/n!>0$ for all $n\in\mathbb{N}$, by  Squeeze Theorem, $\lim_{n\rightarrow\infty} R^{f, x_0}_n (x) = 0$. \\
Hence by Theorem 7.39, $f$ is analytic on $(a, b)$ and 
\[
    f(x)
    = \sum_{k=0}^\infty \frac{f^{(k)}(x_0)}{k!} (x-x_0)^k.
\]

7.46. Suppose that $I$ is an open interval centered at $c$ and that
\[
    f(x)
    = \sum_{k=0}^\infty a_k (x-c)^k,\quad x\in I.
\]
If $x_0\in I$ and $r>0$ satisfy $(x_0-r, x_0+r)\subseteq I$, then
\[
    f(x)
    = \sum_{k=0}^\infty \frac{f^{(k)}(x_0)}{k!} (x-x_0)^k
\]
for all $x\in(x_0-r, x_0+r)$. In particular, if $f$ is a $C^\infty$ function whose Taylor series expansion converges to $f$ on some open interval $I$, then $f$ is analytic on $I$. \\
$Proof$. Suppose that $c=0$ and $I=(-R, R)$, then $f(x) = \sum_{k=0}^\infty a_k x^k$ for all $x\in I$. \\
Assume that $(x_0-r, x_0+r)\subseteq I$ and $x\in(x_0-r, x_0+r)$. \\
By Binomial Formula,
\[
    f(x)
    = \sum_{k=0}^\infty a_k x^k
    = \sum_{k=0}^\infty a_k \sum_{j=0}^k \left( \begin{matrix}
        k \\ j
    \end{matrix}\right) x^{k-j}_0 (x-x_0)^j.
\]
Since $\sum_{k=0}^\infty a_k (|x-x_0|+|x_0|)^k$ converges absolutely and $|x-x_0|+|x_0|<R$, then
\begin{eqnarray*}
\sum_{k=0}^\infty \left| a_k \sum_{j=0}^k \left( \begin{matrix}
    k \\ j
\end{matrix}\right) x^{k-j}_0 (x-x_0)^j\right|
    &\le& \sum_{k=0}^\infty |a_k| \sum_{j=0}^k \left( \begin{matrix}
        k \\ j
    \end{matrix}\right) |x_0|^{k-j} |x-x_0|^j \\
    & = & \sum_{k=0}^\infty |a_k| (|x-x_0|+|x_0|)^k < \infty. 
\end{eqnarray*}
By Theorem 7.18 and Corollary 7.31,
\begin{eqnarray*}
f(x)
    & = & \sum_{k=0}^\infty a_k \sum_{j=0}^k \left( \begin{matrix}
        k \\ j
    \end{matrix}\right) x^{k-j}_0 (x-x_0)^j \\
    & = & \sum_{j=0}^\infty \left( \sum_{k=j}^\infty \left( \begin{matrix}
        k \\ j
    \end{matrix}\right) a_k x^{k-j}_0 \right) (x-x_0)^j \\
    & = & \sum_{j=0}^\infty \left( \sum_{k=j}^\infty \frac{k!}{(k-j)!}a_k x^{k-j}_0 \right) \frac{(x-x_0)^j}{j!}
    = \sum_{j=0}^\infty \frac{f^{(j)}(x_0)}{j!} (x-x_0)^j. 
\end{eqnarray*}
Hence the proof is complete by making the change of variables $w = x-c$. \\

7.50. Let $n\in\mathbb{N}$. If $f\in C^n(a, b)$, then
\[
    R_n(x) := R_n^{f, x_0}(x)
    = \frac{1}{(n-1)!} \int_{x_0}^x (x-t)^{n-1} f^{(n)} (t)\ dt
\]
for all $x, x_0\in(a, b)$. \\
$Proof$. Let $x, x_0\in(a, b)$. \\
By the Fundamental Theorem of Calculus, 
\[
    R_1(x) 
    = \int_{x_0}^x f^{(1)} (t)\ dt
    = f(x) - f(x_0)
    = f(x) - \frac{f^{(0)}(x_0)}{0!} (x-x_0)^0.
\]
Suppose that $\exists\ n\in\mathbb{N}$ such that
\[
    R_n(x) 
    = \frac{1}{(n-1)!} \int_{x_0}^x (x-t)^{n-1} f^{(n)} (t)\ dt.
\]
Since 
\[
    R_{n+1}(x) 
    = f(x) - \sum_{k=0}^n \frac{f^{(k)}(x_0)}{k!} (x-x_0)^k
    = R_n(x) - \frac{f^{(n)}(x_0)}{n!} (x-x_0)^n
\]
and
\[
    \frac{(x-x_0)^n}{n!}
    = \frac{1}{(n-1)!} \int_{x_0}^x (x-t)^{n-1}\ dt,
\]
then
\[
    R_{n+1}(x) 
    = \frac{1}{(n-1)!} \int_{x_0}^x (x-t)^{n-1} \left( f^{(n)}(t)-f^{(n)}(x_0) \right)\ dt.
\]
Let $u(t) = f^{(n)}(t)-f^{(n)}(x_0)$ and $v(t) = -(x-t)^n/n$, by Theorem 5.31, 
\[
    R_{n+1}(x) 
    = \frac{1}{(n-1)!} \left( 
        u(x)v(x) - u(x_0)v(x_0) - \int_{x_0}^x u'(t) v(t)\ dt 
    \right).
\]
Since $u(x_0)=0$ and $v(x)=0$, then
\[
    R_{n+1}(x) 
    = \frac{-1}{(n-1)!} \int_{x_0}^x u'(t) v(t)\ dt 
    = \frac{1}{n!} \int_{x_0}^x (x-t)^{n} f^{(n+1)} (t)\ dt.
\]
Hence the proof is complete. \\

\end{CJK}
\end{document}