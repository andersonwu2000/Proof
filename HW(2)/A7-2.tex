\documentclass[12pt]{book}
\usepackage[utf8]{inputenc}
\usepackage{color,soul,CJK,epic,tikz,array}
\usepackage{amsmath,amsthm,amssymb}
\usepackage{graphicx}
\usepackage{float}
\usepackage{subfigure}
\setlength{\parindent}{0em}
\author{andersonwu2000}
\pagestyle{empty}
\thispagestyle{empty} 

\newcounter{sect}

\newcounter{block}[sect]
\newenvironment{tblock}[1]
{\refstepcounter{block}\theblock.sim\begin{minipage}[t]{\dimexpr\linewidth}#1\\}
{\end{minipage}\\}

\newenvironment{comm}
{\makebox[12pt][l]{$\bullet$}\begin{minipage}[t]{\dimexpr\linewidth}}
{\end{minipage}}

\begin{document}
\begin{CJK}{UTF8}{bsmi}

\hfill 2021/6/1 吳至堯 U10811023 \\

7.9. Let $E$ be a nonempty subset of $\mathbb{R}$ and suppose that $f_n\rightarrow f$ uniformly on $E$, as $n\rightarrow\infty$. If each $f_n$ is continuous at some $x_0\in E$, then $f$ is continuous at $x_0\in E$. \\
$Proof$. Let $\varepsilon>0$. \\
Since $f_n\rightarrow f$ uniformly on $E$ \\
$\Rightarrow\quad\exists\ N\in\mathbb{N}$ such that $|f_n(x)-f(x)|<\varepsilon/3$ for all $n\ge N$ and $x\in E$. \\
Since $f_N$ is continuous at some $x_0$ \\
$\Rightarrow\quad\exists\ \delta>0$ such that $|x-x_0|<\delta$ and $x\in E$ imply $|f_N(x)-f_N(x_0)|<\varepsilon/3$. \\
Suppose that $|x-x_0|<\delta$ and $x\in E$ \\
$\Rightarrow\quad|f(x)-f(x_0)|
\le|f(x)-f_N(x)|+|f_N(x)-f_N(x_0)|+|f_N(x_0)-f(x_0)|
<\varepsilon$.  \\
Hence $f$ is continuous at $x_0\in E$. \\

7.10. Suppose that $f_n\rightarrow f$ uniformly on a closed interval $\left[ a, b \right]$. If each $f_n$ is integrable on $\left[ a, b \right]$, then so is $f$ and
\[
\lim_{n\rightarrow\infty}\int_a^b f_n(x)\ dx
=\int_a^b\left( \lim_{n\rightarrow\infty}f_n(x) \right)\ dx
\]
In fact, $\lim_{n\rightarrow\infty}\int_a^x f_n(t)\ dt=\int_a^x f(t)\ dt$ uniformly for $x\in\left[ a, b \right]$. \\
$Proof$. Claim : $f$ is bounded on $\left[ a, b \right]$.\\
Since $f_n\rightarrow f$ uniformly on $\left[ a, b \right]$ \\
$\Rightarrow\quad\exists\ N\in\mathbb{N}$ such that $n\ge N$ and $x\in\left[ a, b \right]$ imply $|f_n(x)-f(x)|<1$. \\
Since each $f_n$ is integrable, $f_n$ is bounded for all $n\in\mathbb{N}$. \\
$\Rightarrow\quad\exists\ M>0$ such that $|f_N(x)|\le M$ \\
$\Rightarrow\quad |f(x)|
=|f(x)-f_N(x)+f_N(x)|
\le|f(x)-f_N(x)|+|f_N(x)|
<1+M$. \\
Hence $f$ is bounded on $\left[ a, b \right]$. \\
Let $\varepsilon>0$. \\
Since $f_n\rightarrow f$ uniformly on $\left[ a, b \right]$ \\
$\Rightarrow\quad\exists\ N\in\mathbb{N}$ such that $n\ge N$ and $x\in\left[ a, b \right]$ imply $|f_n(x)-f(x)|<\frac{\varepsilon}{3(b-a)}$. \\
It follows that
\[
U(f-f_N, P)
=\sum_{j=1}^n M_j(f-f_N)\Delta x_j
<\sum_{j=1}^n\frac{\varepsilon}{3(b-a)}\Delta x_j
=\frac{\varepsilon}{3}
\]
and
\[
L(f-f_N, P)
=\sum_{j=1}^n m_j(f-f_N)\Delta x_j
>\sum_{j=1}^n\frac{-\varepsilon}{3(b-a)}\Delta x_j
=-\frac{\varepsilon}{3}
\]
for each $P=\{x_0, x_1, \cdots, x_n\}\in\mathcal{P}\left[ a, b \right]$. \\
Since $f_N$ is integrable \\
$\Rightarrow\quad\exists\ P=\{x_0, x_1, \cdots, x_n\}\in\mathcal{P}\left[ a, b \right]$ such that $U(f_N, P)-L(f_N, P)<\varepsilon/3$. \\
Since 
\begin{eqnarray*}
U(f, P) - L(f, P)
    & = & \sum_{j=1}^n \left( M_j(f)-m_j(f) \right)\Delta x_j \\
    & = & \sum_{j=1}^n \left( M_j(f)-M_j(f_N)+M_j(f_N)\right. \\
    &   & \left. -m_j(f)+m_j(f_N)-m_j(f_N) \right)\Delta x_j \\
    & = & U(f-f_N, P)+U(f_N, P)-L(f-f_N, P)-L(f_N, P) \\
    & < & \frac{\varepsilon}{3} + \frac{\varepsilon}{3} + \frac{\varepsilon}{3} = \varepsilon
\end{eqnarray*}
$\Rightarrow\quad f$ is integrable on $\left[ a, b \right]$. \\
Since 
\[
\left|
\int_a^x f_n(t)\ dt - \int_a^x f(t)\ dt
\right|
\le \int_a^x \left| f_n(t) - f(t) \right|\ dt
\le \int_a^x \frac{\varepsilon}{3(b-a)}\ dt
<\varepsilon
\]
for all $x\in\left[ a, b \right]$ and $n\ge N$ \\
$\Rightarrow\quad\lim_{n\rightarrow\infty}\int_a^x f_n(t)\ dt=\int_a^x f(t)\ dt$ uniformly for $x\in\left[ a, b \right]$ and
\[
\lim_{n\rightarrow\infty}\int_a^b f_n(x)\ dx
=\int_a^b\left( \lim_{n\rightarrow\infty}f_n(x) \right)\ dx.
\]
Hence the proof is complete. \\

7.12. Let $(a, b)$ be a bounded interval and suppose that $f_n$ is a sequence of functions which converges at some $x_0\in(a, b)$. If each $f_n$ is differentiable on $(a, b)$, and $f'_n$ converges uniformly on $(a, b)$ as $n\rightarrow\infty$, then $f_n$ converges uniformly on $(a, b)$ and
\[
\lim_{n\rightarrow\infty} f'_n(x) = \left( \lim_{n\rightarrow\infty} f_n(x) \right)'
\]
for each $x\in(a, b)$. \\
$Proof$. Let $\varepsilon>0$, $c\in(a, b)$ and 
\[
g_n(x)=\left\{\begin{matrix}\displaystyle
  \frac{f_n(x)-f_n(c)}{x-c} & x\ne c \\
  f'_n(c) & x=c
\end{matrix}\right.
\]
for $n\in\mathbb{N}$. \\
$\Rightarrow\quad f_n(x)=f_n(c)+(x-c)g_n(x)$ for all $n\in\mathbb{N}$ and $x\in(a, b)$. \\
Claim : $g_n$ converges uniformly on $(a, b)$ for all $c\in(a, b)$. \\
Let $n, m\in\mathbb{N}$ and $x\in(a, b)$ with $x\ne c$. \\
W.L.O.G., suppose that $x<c$. \\
Since each $f_n$ is differentiable on $(a, b)$ \\
$\Rightarrow\quad f_n, f_m$ and $g_n-g_m$ is continuous on $(x, c)$. \\
By the Mean Value Theorem, $\exists\ \xi\in(x, c)$ such that
\[
g_n(x)-g_m(x) = \frac{f_n(x)-f_m(x)-(f_n(c)-f_m(c))}{x-c} = f'_n(\xi)-f'_m(\xi).
\]
Since $f'_n$ converges uniformly on $(a, b)$ as $n\rightarrow\infty$ \\
$\Rightarrow\quad\exists\ N_1\in\mathbb{N}$ such that 
\[
n, m\ge N\quad\text{imply}\quad |f'_n(\xi)-f'_m(\xi)|=|g_n(x)-g_m(x)|<\varepsilon.
\] 
Since $g_n(x)=f'_n(x)$ where $x=c$ \\
$\Rightarrow\quad|g_n(c)-g_m(c)|=|f'_n(c)-f'_m(c)|<\varepsilon$ if $n, m\ge N_1$. \\
Therefore, by Lemma 7.11, $g_n$ converges uniformly on $(a, b)$. \\
Since $f_n$ converges at $x_0$ \\
$\Rightarrow\quad\exists\ f:(a, b)\rightarrow\mathbb{R}$ such that $f_n\rightarrow f$ at $x_0$, as $n\rightarrow\infty$ \\
$\Rightarrow\quad\exists\ N_2\in\mathbb{N}$ such that $n\ge N_2$ implies $|f_n(x_0)-f(x_0)|<\varepsilon/3$. \\
If $c=x_0$ \\
$\Rightarrow\quad f_n(x)=f_n(x_0)+(x-x_0)g_n(x)$ for all $n\in\mathbb{N}$ and $x\in(a, b)$. \\
Since $g_n$ converges uniformly on $(a, b)$ \\
$\Rightarrow\quad\exists\ N_3\in\mathbb{N}$ such that 
\[
n, m\ge N_3\quad\text{imply}\quad |g_n(x)-g_m(x)|<\frac{\varepsilon}{3(b-a)}
\]
for all $x\in(a, b)$. \\
It follows that
\begin{eqnarray*}
|f_n(x)-f_m(x)| 
    & = &   |f_n(x_0)-f_m(x_0)+(x-x_0)(g_n(x)-g_m(x))| \\
    &\le&   |f_n(x_0)-f|+|f-f_m(x_0)|+|x-x_0||g_n(x)-g_m(x)| \\
    & < &   \frac{\varepsilon}{3}+\frac{\varepsilon}{3}+\frac{\varepsilon}{3}=\varepsilon
\end{eqnarray*}
for $n, m\ge\max\{N_2, N_3\}$ and $x\in(a, b)$. \\
Hence $f_n$ converges uniformly on $(a, b)$ as $n\rightarrow\infty$. \\
Since $g_n$ converges uniformly on $(a, b)$ \\
$\Rightarrow\quad\exists\ g:(a, b)\rightarrow\mathbb{R}$ such that $g_n\rightarrow g$ on $(a, b)$, as $n\rightarrow\infty$. \\
Since $\lim_{x\rightarrow c}g_n(x)=f'_n(c)=g_n(c)$ for $n\in\mathbb{N}$ \\
$\Rightarrow\quad g_n$ is continuous at $c$ for $n\in\mathbb{N}$. \\
By Theorem 7.9, $g$ is continuous at $c$. \\
Since $g$ is continuous at $c$ and $g_n(c)=f'_n(c)$ for $n\in\mathbb{N}$ \\
$\Rightarrow\quad\lim_{n\rightarrow\infty}f'_n(c)
=\lim_{n\rightarrow\infty} g_n(c)
=g(c)
=\lim_{x\rightarrow c}g(x)$. \\
Since
\begin{eqnarray*}
f'(c)
    & = &   \lim_{x\rightarrow c}\frac{f(x)-f(c)}{x-c} \\
    & = &   \lim_{x\rightarrow c}\lim_{n\rightarrow\infty}\frac{f_n(x)-f_n(c)}{x-c} \\
    & = &   \lim_{x\rightarrow c}\lim_{n\rightarrow\infty}g_n(x) \\
    & = &   \lim_{x\rightarrow c}g(x) \\
\end{eqnarray*}
$\Rightarrow\quad\lim_{n\rightarrow\infty}f'_n(c)=\lim_{x\rightarrow c}g(x)=f'(c)$. \\
Hence the proof is complete. \\

\end{CJK}
\end{document}