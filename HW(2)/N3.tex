\documentclass[12pt]{book}
\usepackage[utf8]{inputenc}
\usepackage{color,soul,CJK,epic,tikz,array}
\usepackage{amsmath,amsthm,amssymb}
\setlength{\parindent}{0em}
\linespread{1.3}
\author{andersonwu2000}
\usepackage[margin=2cm]{geometry}
\pagestyle{empty}
\thispagestyle{empty} 

\newcounter{sect}

\newcounter{block}[sect]
\newenvironment{tblock}[1]
{\refstepcounter{block}\theblock.sim\begin{minipage}[t]{\dimexpr\linewidth}#1\\}
{\end{minipage}\\}

\newenvironment{comm}
{\makebox[12pt][l]{$\bullet$}\begin{minipage}[t]{\dimexpr\linewidth}}
{\end{minipage}}

\begin{document}
\begin{CJK}{UTF8}{bsmi}

\hfill 吳至堯 U10811023

Show that every positive integer has a unique Zeckendorf representation. \\
Let $P(n)$ is true iff $n\in\mathbb{N}$ and $n$ has a Zeckendorf representation. \\
Since $f_1=1, f_2=1, f_3=2\quad\Rightarrow\quad P(1), P(2)$ is true \\
Suppose that $n\in\mathbb{N}$ and $P(k)$ is true for all $2\le k<n,\ k\in\mathbb{N}$ \\
If $\exists\ m\in\mathbb{N}$ such that $f_m=n\quad\Rightarrow\quad P(n)$ is true \\
Else, $\exists\ m\in\mathbb{N}$ such that $f_m<n<f_{m+1}\quad\Rightarrow\quad\exists\ a\in\mathbb{N}$ such that $a=n-f_m$ \\
Since $2\le a<n\quad\Rightarrow\quad P(a)$ is true \\
Since $n=a+f_m$, $a=n-f_m<f_{m+1}-f_m=f_{m-1}$ and $P(a)$ is true $\quad\Rightarrow\quad P(n)$ is true \\
Hence, $\forall n\in\mathbb{N}$, $n$ has a Zeckendorf representation \\
Let two sets of non-consecutive Fibonacci numbers $R, S$ have the same sum, $R\ne\varnothing$, $S\ne\varnothing$ \\
Since $R/S=R-R\cap S$ and $S/R=S-R\cap S\quad\Rightarrow\quad S/R$ and $R/S$ has the same sum \\
Assume that $R\cap S\ne\varnothing$ \\
Let $\max R/S=f_r,\quad\max S/R=f_s$ \\
Since $R/S\cap S/R=\varnothing\quad\Rightarrow\quad f_r\ne f_s$ \\
W.L.O.G., If $f_r>f_s\quad\Rightarrow\quad$ sum of $S/R<f_{s+1}\le f_r\le$ sum of $R/S\rightarrow\leftarrow$ \\
Since $S/R$ and $R/S$ has the same sum$\quad\Rightarrow\quad S/R=R/S=\varnothing\quad\Rightarrow\quad R=S$ \\
Hence, $\forall n\in\mathbb{N}$, $n$ has a unique Zeckendorf representation \\\\

將學號後四碼用 Zeckendorf representation 表示 \\
Fibonacci numbers : 1, 1, 2, 3, 5, 8, 13, 21, 34, 55, 89, 144, 233, 377, 610, 987, 1597, ... \\
$1023 = 987+34+2$
\end{CJK}
\end{document}