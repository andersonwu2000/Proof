\documentclass[12pt]{book}
\usepackage[utf8]{inputenc}
\usepackage{color,soul,CJK,epic,tikz,array}
\usepackage{amsmath,amsthm,amssymb}
\usepackage{graphicx}
\usepackage{float}
\usepackage{subfigure}
\setlength{\parindent}{0em}
\linespread{1.3}
\author{andersonwu2000}
\usepackage[margin=1cm]{geometry}
\pagestyle{empty}
\thispagestyle{empty} 

\newcounter{sect}

\newcounter{block}[sect]
\newenvironment{tblock}[1]
{\refstepcounter{block}\theblock.sim\begin{minipage}[t]{\dimexpr\linewidth}#1\\}
{\end{minipage}\\}

\newenvironment{comm}
{\makebox[12pt][l]{$\bullet$}\begin{minipage}[t]{\dimexpr\linewidth}}
{\end{minipage}}

\begin{document}
\begin{CJK}{UTF8}{bsmi}

\hfill 2021/5/18 吳至堯 U10811023

11.49. Since $\displaystyle u^2+v^2+w^2=29$ and $\displaystyle\frac{u^2}{x^2}+\frac{v^2}{y^2}+\frac{w^2}{z^2}=17$ \\
$\displaystyle\Rightarrow\quad
  \frac{\partial}{\partial x}\left( u^2+v^2+w^2 \right)
= 2u\frac{\partial u}{\partial x}+2v\frac{\partial v}{\partial x}
= 0$,\quad $\displaystyle
\frac{\partial}{\partial y}\left( u^2+v^2+w^2 \right)
= 2u\frac{\partial u}{\partial y}+2v\frac{\partial v}{\partial y}
= 0$, \\
\hspace*{2.2em}$\displaystyle
  \frac{\partial}{\partial x}\left( \frac{u^2}{x^2}+\frac{v^2}{y^2}+\frac{w^2}{z^2} \right)
= 2ux^{-2}\frac{\partial u}{\partial x}-2u^2x^{-3}+2vy^{-2}\frac{\partial v}{\partial x}
= 0$, \\ 
\hspace*{2.2em}$\displaystyle
  \frac{\partial}{\partial y}\left( \frac{u^2}{x^2}+\frac{v^2}{y^2}+\frac{w^2}{z^2} \right)
= 2ux^{-2}\frac{\partial u}{\partial y}+2vy^{-2}\frac{\partial v}{\partial y}-2v^2y^{-3}
= 0$ \\[5pt]
Where $(u, v, x, y, z, w)=(4, 3, 2, 1, -1, -2)$ \\[5pt]
$\displaystyle\Rightarrow\quad 
  8\frac{\partial u}{\partial x}+6\frac{\partial v}{\partial x}
= 2\frac{\partial u}{\partial x}-4+6\frac{\partial v}{\partial x}
= 0
= 8\frac{\partial u}{\partial y}+6\frac{\partial v}{\partial y}
= 2\frac{\partial u}{\partial y}+6\frac{\partial v}{\partial y}-18$ \\
$\displaystyle\Rightarrow\quad\frac{\partial u}{\partial x}=-\frac{2}{3},\quad\frac{\partial v}{\partial x}=\frac{8}{9},\quad\frac{\partial u}{\partial y}=-3,\quad\frac{\partial v}{\partial y}=4$ \\[5pt]
Hence $\displaystyle\frac{\partial u}{\partial x}=-\frac{2}{3},\quad\frac{\partial v}{\partial x}=\frac{8}{9},\quad\frac{\partial u}{\partial y}=-3,\quad\frac{\partial v}{\partial y}=4$ at $(x, y, z, w)=(2, 1, -1, -2)$ \\\\

11.56. Let $V$ be open in $\mathbb{R}^n$, $\mathbf{a}\in V$, and suppose that $f:V\rightarrow\mathbb{R}$ satisfies $\nabla f(\mathbf{a})=\mathbf{0}$. \\
Suppose further that the second-order total differential of $f$ exists on $V$ and is continuous at $\mathbf{a}$. \\

11.56.i. If $D^{(2)}f(\mathbf{a};\mathbf{h})>0$ for all $\mathbf{h}\ne0$, then $f(\mathbf{a})$ is a local minimum of $f$. 

$Proof$. Since $\mathbf{a}\in V$ and $V$ is open $\quad\Rightarrow\quad\exists\ r_1>0$ s.t. $B_{r_1}(\mathbf{a})\subset V$ \\
Let $\varepsilon : B_{r_1}(\mathbf{0})\rightarrow\mathbb{R}$ be define by $\displaystyle\varepsilon(\mathbf{h})=\left\{\begin{matrix}
0,\quad\text{for\ }\mathbf{h}=\mathbf{0} \\
\displaystyle\frac{f(\mathbf{a}+\mathbf{h})-f(\mathbf{a})-\frac{1}{2}D^{(2)}f(\mathbf{a};\mathbf{h})}{\parallel\mathbf{h}\parallel^2},\quad\text{for\ }\mathbf{h}\in B_{r_1}(\mathbf{0})\setminus\{\mathbf{0}\}
\end{matrix}\right.$ \\
$\displaystyle\Rightarrow\quad f(\mathbf{a}+\mathbf{h})-f(\mathbf{a})=\frac{1}{2}D^{(2)}f(\mathbf{a};\mathbf{h})+\parallel\mathbf{h}\parallel^2\varepsilon(\mathbf{h}),\ \mathbf{h}\in B_{r_1}(\mathbf{0})$ \\
Let $\mathbf{h}_1=(h_1, h_2, \cdots, h_n)\in B_{r_1}(\mathbf{0})$, by Taylor's Formula, since $\nabla f(\mathbf{a})=\mathbf{0}$ \\
$\Rightarrow\quad\exists\ \mathbf{c}\in L(\mathbf{a};\mathbf{a}+\mathbf{h}_1)$ s.t. $\displaystyle f(\mathbf{a}+\mathbf{h}_1)=f(\mathbf{a})+\frac{1}{2}D^{(2)}f(\mathbf{c};\mathbf{h}_1)$ \\
$\displaystyle\Rightarrow\quad f(\mathbf{a}+\mathbf{h}_1)-f(\mathbf{a})-\frac{1}{2}D^{(2)}f(\mathbf{a};\mathbf{h}_1)=\frac{1}{2}\left(D^{(2)}f(\mathbf{c};\mathbf{h}_1)-D^{(2)}f(\mathbf{a};\mathbf{h}_1)\right)$ \\
\hspace*{2em}$\displaystyle=\frac{1}{2}\sum_{j=1}^n\sum_{k=1}^n\left(\frac{\partial^2 f}{\partial x_j\partial x_k}(\mathbf{c})-\frac{\partial^2 f}{\partial x_j\partial x_k}(\mathbf{a})\right)h_jh_k\ 
\le\ \frac{1}{2}\sum_{j=1}^n\sum_{k=1}^n\left(\frac{\partial^2 f}{\partial x_j\partial x_k}(\mathbf{c})-\frac{\partial^2 f}{\partial x_j\partial x_k}(\mathbf{a})\right)\parallel\mathbf{h}_1\parallel^2$ \\
Since the second-order total differential of $f$ is continuous at $\displaystyle\mathbf{a}$ \\
$\displaystyle\Rightarrow\quad\lim_{\mathbf{h}_1\rightarrow\mathbf{0}}\frac{1}{2}\sum_{j=1}^n\sum_{k=1}^n\left(\frac{\partial^2 f}{\partial x_j\partial x_k}(\mathbf{c})-\frac{\partial^2 f}{\partial x_j\partial x_k}(\mathbf{a})\right)=0$ \\
Since $\displaystyle0\le|\varepsilon(\mathbf{h})|\le\frac{1}{2}\sum_{j=1}^n\sum_{k=1}^n\left|\frac{\partial^2 f}{\partial x_j\partial x_k}(\mathbf{c})-\frac{\partial^2 f}{\partial x_j\partial x_k}(\mathbf{a})\right|\quad\Rightarrow\quad\lim_{\mathbf{h}\rightarrow\mathbf{0}}\varepsilon(\mathbf{h})=0$ \\
Since $ D^{(2)}f(\mathbf{a};\mathbf{h})>0$ for all $\mathbf{h}\ne\mathbf{0}$, by Lemma 11.55, $\exists\ m>0$ s.t. $D^{(2)}f(\mathbf{a};\mathbf{h})\ge m\parallel\mathbf{h}\parallel^2$ \\
$\displaystyle\Rightarrow\quad f(\mathbf{a}+\mathbf{h})-f(\mathbf{a})\ge\ \parallel \mathbf{h}\parallel^2(\frac{m}{2}+\varepsilon(\mathbf{h}))$ for all $\mathbf{h}\in B_{r_1}(\mathbf{0})$ \\
Since $\displaystyle\lim_{\mathbf{h}\rightarrow\mathbf{0}}\varepsilon(\mathbf{h})=0\quad\Rightarrow\quad\exists\ r_2>0$ s.t. $\displaystyle|\varepsilon(\mathbf{h})|<\frac{m}{2}$ for all $\mathbf{h}\in B_{r_2}(\mathbf{0})$ \\
Take $r=\min\{r_1, r_2\}$,\quad since $\displaystyle\parallel \mathbf{h}\parallel^2(\frac{m}{2}+\varepsilon(\mathbf{h}))\ne0$ for all $\mathbf{h}\ne\mathbf{0}\quad\Rightarrow\quad f(\mathbf{a}+\mathbf{h})-f(\mathbf{a})>0$,\ $\mathbf{h}\in B_r(\mathbf{0})\setminus\{\mathbf{0}\}$ \\
Hence $f(\mathbf{x})>f(\mathbf{a})$ for all $\mathbf{x}\in B_r(\mathbf{a})\setminus\{\mathbf{a}\}$ and $f(\mathbf{a})$ is a local minimum of $f$ \\

11.56.ii. If $D^{(2)}f(\mathbf{a};\mathbf{h})<0$ for all $\mathbf{h}\ne0$, then $f(\mathbf{a})$ is a local maximum of $f$.

$Proof$. Let $\displaystyle g = -f\quad\Rightarrow\quad D^{(2)}g(\mathbf{a};\mathbf{h})=\sum_{j=1}^n\sum_{i=1}^n\frac{\partial^2}{\partial x_j\partial x_i}(-f(\mathbf{a}))h_ih_j=-D^{(2)}f(\mathbf{a};\mathbf{h})$ \\
$\Rightarrow\quad D^{(2)}g(\mathbf{a};\mathbf{h})>0$ for all $\mathbf{h}\ne\mathbf{0}$ \\
By Thm 11.56.i, $g(\mathbf{a})$ is a local minimum of $g\quad\Rightarrow\quad\exists\ r>0$ s.t. $g(\mathbf{x})>g(\mathbf{a})$ for all $\mathbf{x}\in B_r(\mathbf{a})\setminus\{\mathbf{a}\}$ \\
$\Rightarrow\quad -f(\mathbf{x})>-f(\mathbf{a})$ for all $\mathbf{x}\in B_r(\mathbf{a})\setminus\{\mathbf{a}\}$ \\
Hence $f(\mathbf{x})<f(\mathbf{a})$ for all $\mathbf{x}\in B_r(\mathbf{a})\setminus\{\mathbf{a}\}$ and $f(\mathbf{a})$ is a local maximum of $f$ \\

11.56.iii. If $D^{(2)}f(\mathbf{a};\mathbf{h})$ takes on both positive and negative values for $\mathbf{h}\in\mathbb{R}^n$, then $\mathbf{a}$ is a saddle point of $f$. 

$Proof$. Since $\mathbf{a}\in V$ and $V$ is open $\quad\Rightarrow\quad\exists\ r_1>0$ s.t. $B_{r_1}(\mathbf{a})\subset V$ \\
Let $\varepsilon : B_{r_1}(\mathbf{0})\rightarrow\mathbb{R}$ be define by $\displaystyle\varepsilon(\mathbf{h})=\left\{\begin{matrix}
0,\quad\text{for\ }\mathbf{h}=\mathbf{0} \\
\displaystyle\frac{f(\mathbf{a}+\mathbf{h})-f(\mathbf{a})-\frac{1}{2}D^{(2)}f(\mathbf{a};\mathbf{h})}{\parallel\mathbf{h}\parallel^2},\quad\text{for\ }\mathbf{h}\in B_{r_1}(\mathbf{0})\setminus\{\mathbf{0}\}
\end{matrix}\right.$ \\
$\displaystyle\Rightarrow\quad f(\mathbf{a}+\mathbf{h})-f(\mathbf{a})=\frac{1}{2}D^{(2)}f(\mathbf{a};\mathbf{h})+\parallel\mathbf{h}\parallel^2\varepsilon(\mathbf{h})$ for $\mathbf{h}\in B_{r_1}(\mathbf{0})$ \\
According to part of the proof of Thm 11.56.i,\quad  $\displaystyle\lim_{\mathbf{h}\rightarrow\mathbf{0}}\varepsilon(\mathbf{h})=0$ \\
Let $\mathbf{h}_0\in B_{r_1}(\mathbf{0})\quad\Rightarrow\quad\exists\ r_2>0$ s.t. $\displaystyle|\varepsilon(\mathbf{h})|<|\frac{D^{(2)}f(\mathbf{a};\mathbf{h}_0)}{2\parallel\mathbf{h}_0\parallel^2}|$ for all $\mathbf{h}\in B_{r_2}(\mathbf{0})$ \\
Take $r=\min\{r_1, r_2\}$, let $\mathbf{h}_1, \mathbf{h}_2\in B_r(\mathbf{0})$ with $D^{(2)}f(\mathbf{a};\mathbf{h}_1)>0$ and $D^{(2)}f(\mathbf{a};\mathbf{h}_2)<0$ \\
$\Rightarrow\quad f(\mathbf{a}+\mathbf{h}_1)-f(\mathbf{a})>0$ and $f(\mathbf{a}+\mathbf{h}_2)-f(\mathbf{a})<0$ \\
Hence $\mathbf{a}$ is a saddle point of $f$.

\end{CJK}
\end{document}