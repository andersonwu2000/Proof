\documentclass[12pt]{book}
\usepackage[utf8]{inputenc}
\usepackage{color,soul,CJK,epic,tikz,array}
\usepackage{amsmath,amsthm,amssymb}
\usepackage{graphicx}
\usepackage{float}
\usepackage{subfigure}
\setlength{\parindent}{0em}
\linespread{1.3}
\author{andersonwu2000}
\usepackage[margin=1cm]{geometry}
\pagestyle{empty}
\thispagestyle{empty} 

\newcounter{sect}

\newcounter{block}[sect]
\newenvironment{tblock}[1]
{\refstepcounter{block}\theblock.sim\begin{minipage}[t]{\dimexpr\linewidth}#1\\}
{\end{minipage}\\}

\newenvironment{comm}
{\makebox[12pt][l]{$\bullet$}\begin{minipage}[t]{\dimexpr\linewidth}}
{\end{minipage}}

\begin{document}
\begin{CJK}{UTF8}{bsmi}

\hfill 章節 20 吳至堯 U10811023

20.10. Show that $x^{21}+2x^8+1$ does not have multiple zeros in any extension of $\mathbb{Z}_3$ \\
$Solution$. Since $f(x)=x^{21}+2x^8+1=x^7(x^{14}+2x)+1$ and $f'(x)=21x^{20}+16x^7=x^7$ \\
$\Rightarrow\quad\gcd(f(x), f'(x))=1$ \\
Hence $x^{21}+2x^8+1$ does not have multiple zeros in any extension of $\mathbb{Z}_3$ \\

20.11. Find the splitting field for $f(x)=(x^2+x+2)(x^2+2x+2)$ over $\mathbb{Z}_3\left[x\right]$. Write $f(x)$ as a product of linear factors. \\
$Solution$. Let $\alpha=x+<x^2+x+2>\quad\Rightarrow\quad f(x)=(x-\alpha)(x+(1+\alpha))(a+\alpha)(x+(1-\alpha))$ \\

21.1. Let $K=\mathbb{Q}(\sqrt{5}, \sqrt[3]{5}, \sqrt[4]{5}, \cdots)$ \\

21.1.a. Prove that $K$ is not a finite extension of $\mathbb{Q}$ \\
$Solution$. Since $K\supseteq\mathbb{Q}(\sqrt[n]{5})$ for all $n\in\mathbb{N}_{\ge2}$ \\
$\Rightarrow\quad\left [ K:\mathbb{Q} \right ]\ge\left [ \mathbb{Q}(\sqrt[n]{5}):\mathbb{Q} \right ] = n$ and $K$ is not a finite extension of $\mathbb{Q}$ \\

21.1.b. Prove that $K$ is an algebraic extension of $\mathbb{Q}$ \\
$Solution$. Let $a\in K\quad\Rightarrow\quad\exists\ k\mathbb{N}_{\ge2}$ s.t. $a\in\mathbb{Q}(\sqrt{5}, \sqrt[3]{5}, \sqrt[4]{5}, \cdots, \sqrt[k]{5})$ \\
Since $\left [ \mathbb{Q}(\sqrt{5}, \sqrt[3]{5}, \sqrt[4]{5}, \cdots, \sqrt[k]{5}):\mathbb{Q} \right ]$ is finite $\quad\Rightarrow\quad\mathbb{Q}(\sqrt{5}, \sqrt[3]{5}, \sqrt[4]{5}, \cdots, \sqrt[k]{5})$ is algebraic \\
Hence $a$ is algebraic over $K$ and $K$ is an algebraic extension of $\mathbb{Q}$ \\

\end{CJK}
\end{document}