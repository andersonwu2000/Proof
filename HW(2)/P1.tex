\documentclass[12pt]{book}
\usepackage[utf8]{inputenc}
\usepackage{color,soul,CJK,epic,tikz,array}
\usepackage{amsmath,amsthm,amssymb}
\usepackage{graphicx}
\usepackage{float}
\usepackage{subfigure}
\setlength{\parindent}{0em}
\linespread{1.3}
\author{andersonwu2000}
\usepackage[margin=1cm]{geometry}
\pagestyle{empty}
\thispagestyle{empty} 

\newcounter{sect}

\newcounter{block}[sect]
\newenvironment{tblock}[1]
{\refstepcounter{block}\theblock.sim\begin{minipage}[t]{\dimexpr\linewidth}#1\\}
{\end{minipage}\\}

\newenvironment{comm}
{\makebox[12pt][l]{$\bullet$}\begin{minipage}[t]{\dimexpr\linewidth}}
{\end{minipage}}

\begin{document}
\begin{CJK}{UTF8}{bsmi}

\hfill 章節 1,2 吳至堯 U10811023

1-5-4. \begin{minipage}[t]{\dimexpr\linewidth-2em}
Assume that an insurance company knows the following probabilities relating to automobile accidents (where the second column refers to the probability that the policyholder has at least one accident during the annual policy period): \\[5pt]
\begin{tabular}{c|c|c}
    Age of Driver & Probability of & Fraction of  \\ 
    & Accident & Company$'$s Insured Drivers \\\hline
    16–25 & 0.05 & $B_1$ : 0.10 \\
    26–50 & 0.02 & $B_2$ : 0.55 \\
    51–65 & 0.03 & $B_3$ : 0.20 \\
    66–90 & 0.04 & $B_4$ : 0.15 \\
\end{tabular}\hspace{2em}\begin{minipage}{16em}
A randomly selected driver from the company$'$s insured drivers has an accident. What is the conditional probability that the driver is in the 16–25 age group?
\end{minipage} \\[5pt]
Let $A=accident\ happened$ \\
$\displaystyle P(B_1\mid A)=\frac{B_1\cap A}{P(A)}=\frac{P(B_1)P(A\mid B_1)}{\sum_rP(B_r)P(A\mid B_r)}=\frac{0.05*0.1}{0.05*0.1+0.02*0.55+0.03*0.2+0.04*0.15}\approx 0.18$
\end{minipage}\\

1-5-10. \begin{minipage}[t]{\dimexpr\linewidth-2em}
Suppose we want to investigate the percentage of abused children in a certain population. To do this, doctors examine some of these children taken at random from that population. However, doctors are not perfect:
They sometimes classify an abused child $(A^+)$ as one not abused $(D^-)$ or they classify a nonabused child $(A^-)$ as one that is abused $(D^+)$. Suppose these error rates are $P(D^- | A^+) = 0.08$ and $P(D^+ | A^-) = 0.05$, respectively; thus, $P(D^+ | A^+) = 0.92$ and $P(D^- | A^-) = 0.95$ are the probabilities of the correct decisions. Let us pretend that only $2\%$ of all children are abused; that is, $P(A^+) = 0.02$ and $P(A^-) = 0.98$. \\
(a) Select a child at random. What is the probability that the doctor classifies this child as abused? That is, compute $P(D^+) = P(A^+)P(D^+ | A^+) + P(A^-)P(D^+ | A^-)$. \\
$P(D^+) = P(A^+)P(D^+ | A^+) + P(A^-)P(D^+ | A^-)=0.02*0.92+0.98*0.05=0.0674$ \\
(b) Compute $P(A^- | D^+)$ and $P(A^+ | D^+)$. \\
$\displaystyle P(A^- | D^+)=\frac{P(D^+\cap A^-)}{P(D^+)}=\frac{P(A^-)P(D^+ | A^-)}{P(D^+)}=\frac{0.98*0.05}{0.0674}\approx0.727$ \\
$\displaystyle P(A^+ | D^+)=\frac{P(D^+\cap A^+)}{P(D^+)}=\frac{P(A^+)P(D^+ | A^+)}{P(D^+)}=\frac{0.02*0.92}{0.0674}\approx0.273$ \\
(c) Compute $P(A^- | D^-)$ and $P(A^+ | D^-)$. \\
$P(D^-) = P(A^+)P(D^- | A^+) + P(A^-)P(D^- | A^-)=0.02*0.08+0.98*0.95=0.9326$ \\
$\displaystyle P(A^- | D^-)=\frac{P(D^-\cap A^-)}{P(D^-)}=\frac{P(A^-)P(D^- | A^-)}{P(D^-)}=\frac{0.98*0.95}{0.9326}\approx0.9983$ \\
$\displaystyle P(A^+ | D-)=\frac{P(D^-\cap A^+)}{P(D^-)}=\frac{P(A^+)P(D^- | A^+)}{P(D^-)}=\frac{0.02*0.08}{0.9326}\approx0.0017$ \\
(d) Are the probabilities in (b) and (c) alarming? This happens because the error rates of 0.08 and 0.05 are high relative to the fraction 0.02 of abused children in the population. \\
是,無受虐而被認為受虐的兒童比例很高 (0.727)
\end{minipage}\\

1-5-14. \begin{minipage}[t]{\dimexpr\linewidth-2em}
Two processes of a company produce rolls of materials: The rolls of Process I are 3\% defective and the rolls of Process II are 1\% defective. Process I produces 60\% of the company’s output, Process II 40\%. A roll is selected at random from the total output. Given that this roll is defective, what is the conditional probability that it is from Process I? \\
$\displaystyle P(defective)=P(I)*P(I\ and\ defective)+P(II)*P(II\ and\ defective)=0.6*0.03+0.4*0.01=0.022$ \\
$\displaystyle P(I\mid defective)=\frac{P(I\ and\ defective)}{P(defective)}=\frac{P(I)*P(defective\mid I)}{P(defective)}=\frac{0.6*0.03}{0.022}\approx0.8182$
\end{minipage}\\

2-1-3. \begin{minipage}[t]{\dimexpr\linewidth-2em}
For each of the following, determine the constant $c$ so that $f(x)$ satisfies the conditions of being a pmf for a random variable $X$, and then depict each pmf as a line graph:
\end{minipage}\\

2-1-3-c. \begin{minipage}[t]{\dimexpr\linewidth-2em}
$f(x)=c(1/4)^x,\quad x=1,2,3,...$ \\
Since $\displaystyle\sum_{x=1}^\infty c(1/4)^x=c\frac{1/4}{1-1/4}=\frac{c}{3}=1\quad\Rightarrow\quad c=3$
\end{minipage}\\

2-1-3-d. \begin{minipage}[t]{\dimexpr\linewidth-2em}
$f(x)=c(x+1)^2,\quad x=0,1,2,3$ \\
Since $\displaystyle\sum_{x=0}^3 c(x+1)^2=c+4c+9c+16c=30c=1\quad\Rightarrow\quad c=\frac{1}{30}$
\end{minipage}
\begin{figure}[H] 
\centering 
\subfigure[2-1-3-c]{
\includegraphics[width=0.4\textwidth]{HW(2)/c.png}}
\centering 
\subfigure[2-1-3-d]{
\includegraphics[width=0.4\textwidth]{HW(2)/d.png}}
\end{figure}

2-2-2. \begin{minipage}[t]{\dimexpr\linewidth-2em}
Let the random variable $X$ have the pmf $\displaystyle f(x)=\frac{(|x|+1)^2}{9},\quad x=-1,0,1$. \\
Compute $E(X),\ E(X^2)$, and $E(3X^2-2X+4)$. \\
$\displaystyle E(X)=\sum_{x=-1}^1 xf(x)=-1\cdot\frac{4}{9}+0\cdot\frac{1}{9}+1\cdot\frac{4}{9}=0$ \\
$\displaystyle E(X^2)=\sum_{x=-1}^1x^2f(x)=1\cdot\frac{4}{9}+0\cdot\frac{1}{9}+1\cdot\frac{4}{9}=\frac{8}{9}$ \\
$\displaystyle E(3X^2-2X+4)=\sum_{x=-1}^1(3x^2-2x+4)f(x)=9\cdot\frac{4}{9}+4\cdot\frac{1}{9}+5\cdot\frac{4}{9}=\frac{20}{3}$ \\
\end{minipage}\\

2-2-7. \begin{minipage}[t]{\dimexpr\linewidth-2em}
In the gambling game chuck-a-luck, for a \$1 bet it is possible to win \$1, \$2, or \$3 with respective probabilities 75/216, 15/216, and 1/216. One dollar is lost with probability 125/216. Let $X$ equal the payoff for this game and find $E(X)$. Note that when a bet is won, the \$1 that was bet, in addition to the \$1, \$2, or \$3 that is won, is returned to the bettor. \\
$\displaystyle E(X)=(-1)\cdot\frac{125}{216}+1\cdot\frac{75}{216}+2\cdot\frac{15}{216}+3\cdot\frac{1}{216}=\frac{-17}{216}$
\end{minipage}\\

\clearpage

2-3-8. \begin{minipage}[t]{\dimexpr\linewidth-2em}
Let $X$ equal the larger outcome when a pair of fair four-sided dice is rolled. The pmf of $X$ is $\displaystyle f(x)=\frac{2x-1}{16},\quad x=1,2,3,4$. Find the mean, variance, and standard deviation of $X$. \\
mean $\displaystyle=E(X)=\sum_{x=1}^4x\frac{2x-1}{16}=1\cdot\frac{1}{16}+2\cdot\frac{3}{16}+3\cdot\frac{5}{16}+4\cdot\frac{7}{16}=\frac{25}{8}$ \\
$\displaystyle Var(X)=E(X^2)-(E(X))^2=1\cdot\frac{1}{16}+4\cdot\frac{3}{16}+9\cdot\frac{5}{16}+16\cdot\frac{7}{16}-\left(\frac{25}{8}\right)^2=\frac{170}{16}-\frac{625}{64}=\frac{55}{64}$ \\
$\displaystyle\sigma=\left(Var(X)\right)^{0.5}=\sqrt{\frac{55}{64}}=\frac{\sqrt{55}}{8}$
\end{minipage}\\

2-3-11. \begin{minipage}[t]{\dimexpr\linewidth-2em}
If the moment-generating function of $X$ is $\displaystyle M(t)=\frac{2}{5}e^t+\frac{1}{5}e^{2t}+\frac{2}{5}e^{3t}$, \\
find the mean, variance, and pmf of $X$. \\
pmf : $f(x)=\frac{|2-x|+1}{5},\quad x=1,2,3$ \\
mean $\displaystyle=E(X)=xf(x)=1\cdot\frac{2}{5}+2\cdot\frac{1}{5}+3\cdot\frac{2}{5}=2$ \\
$Var(X)=E(X^2)-(E(X))^2=1\cdot\frac{2}{5}+4\cdot\frac{1}{5}+9\cdot\frac{2}{5}-2^2=\frac{4}{5}$ \\
\end{minipage}\\

2-3-14. \begin{minipage}[t]{\dimexpr\linewidth-2em}
The probability that a machine produces a defective item is 0.01. Each item is checked as it is produced. Assume that these are independent trials, and compute the probability that at least 100 items must be checked to find one that is defective. \\
$(1-0.01)^{100-1}\approx0.37$
\end{minipage}\\

\end{CJK}
\end{document}